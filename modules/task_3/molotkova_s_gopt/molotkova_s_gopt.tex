\documentclass{report}

\usepackage[T2A]{fontenc}
\usepackage[utf8]{luainputenc}
\usepackage[english, russian]{babel}
\usepackage[pdftex]{hyperref}
\usepackage[14pt]{extsizes}
\usepackage{listings}
\usepackage{color}
\usepackage{geometry}
\usepackage{enumitem}
\usepackage{multirow}
\usepackage{graphicx}
\usepackage{indentfirst}

\geometry{a4paper,top=2cm,bottom=3cm,left=2cm,right=1.5cm}
\setlength{\parskip}{0.5cm}
\setlist{nolistsep, itemsep=0.3cm,parsep=0pt}

\lstset{language=C++,
		basicstyle=\footnotesize,
		keywordstyle=\color{blue}\ttfamily,
		stringstyle=\color{red}\ttfamily,
		commentstyle=\color{green}\ttfamily,
		morecomment=[l][\color{magenta}]{\#}, 
		tabsize=4,
		breaklines=true,
  		breakatwhitespace=true,
  		title=\lstname,       
}

\makeatletter
\renewcommand\@biblabel[1]{#1.\hfil}
\makeatother

\begin{document}

\begin{titlepage}

\begin{center}
Министерство науки и высшего образования Российской Федерации
\end{center}

\begin{center}
Федеральное государственное автономное образовательное учреждение высшего образования \\
Национальный исследовательский Нижегородский государственный университет им. Н.И. Лобачевского
\end{center}

\begin{center}
Институт информационных технологий, математики и механики
\end{center}

\vspace{4em}

\begin{center}
\textbf{\LargeОтчет по лабораторной работе} \\
\end{center}
\begin{center}
\textbf{\Large«Многошаговая схема решения двумерных задач глобальной оптимизации. Распараллеливание путем разделения области поиска»} \\
\end{center}

\vspace{4em}

\newbox{\lbox}
\savebox{\lbox}{\hbox{text}}
\newlength{\maxl}
\setlength{\maxl}{\wd\lbox}
\hfill\parbox{7cm}{
\hspace*{5cm}\hspace*{-5cm}\textbf{Выполнил:} \\ студент группы 381806-1 \\ Молоткова С. С.\\
\\
\hspace*{5cm}\hspace*{-5cm}\textbf{Проверил:}\\ доцент кафедры МОСТ, \\ кандидат технических наук \\ Сысоев А. В.\\
}
\vspace{\fill}

\begin{center} Нижний Новгород \\ 2020 \end{center}

\end{titlepage}

\setcounter{page}{2}

% Содержание
\tableofcontents
\newpage

% Введение
\section*{Введение}
\addcontentsline{toc}{section}{Введение}
Часто для изучения свойств какой-либо системы строят её математическую модель, чтобы предсказать поведение объекта.
\par
В общем случае модель объекта определяется «чёрным ящиком», позволяющим определить значение вектора характеристик для заданного вектора параметров. Улучшение характеристик, интерпретируемое как уменьшение их значений, может осуществляться за счёт подбора вектора параметров. Возникает задача оптимизации.
\par
Задача оптимизации состоит в минимизации критерия в области поиска, в которой параметры могут изменяться. Подклассом этой задачи является задача глобальной оптимизации, в которой необходимо найти глобальный минимум в заданной области.
\par
В данной работе будет рассмотрен один из методов(метод Стронгина) решения задачи глобальной оптимизации и разработан его последовательный и параллельный вариант.

\newpage

% Постановка задачи
\section*{Постановка задачи}
\addcontentsline{toc}{section}{Постановка задачи}
В данной работе необходимо реализовать метод Стронгина решения задачи глобальной оптимизации. Данный метод должен содержать последовательную и параллельную версии. Распараллеливание должно происходить путем разделения области поиска.
\par
На основе разработанной программы необходимо провести вычислительные эксперименты, сравнив время работы параллельного и последовательного алгоритмов на разных входных данных и разном количестве процессов, затем сделать вывод об эффективности параллельнной версии метода.
\par
Также следует экспериментально проверить корректность разработанных версий алгоритма с помощью функционального тестирования с использованием Google тестов.
\newpage

% Метод решения
\section*{Метод решения}
\addcontentsline{toc}{section}{Метод решения}
Формальная постановка задачи выглядит следующим образом:
$f({x}^{\ast}) = min \{f(x):x \in [a,b]\}$
\par При условии, что исследуемая функция удовлетворяет условию Липшица, будем использовать метод Стронгина решения задачи глобальной оптимизации. Данный метод разработан в рамках информационно-статистического подхода и заключается в построении неравномерной сетки испытаний на заданном отрезке.
\par Пусть $x$ – вектор, представляющий сетку испытаний. При старте алгоритма $x_0 = a, x_1 = b$. Итерация построения нового звена сетки выглядит следующим образом:
\begin{enumerate}
	\item Отсортировать массив $x$ в порядке возрастания координат.
	\item Вычислить текущую оценку константы Липшица по следующим формулам: \par
	$M = \max_{0<j<n}\frac{|y_i - y_{i-1} |}{x_i - x_{i-1}},$\par
	$
	m =\left\{
	\begin{array}{rcl}
		r \times{} M&,&M > 0,\\
		1& , &M = 0\\
	\end{array}
	\right.
	$ \par
	Где $y_i = f(x_i), r < 1$ параметр надежности метода для выполнения условия сходимости метода:$m > 2L$, где $L$ константа Липшица.
	\item Для каждого интервала сетки $(x_{i-1},x_i), 1 \leq i \leq k$ вычислить их характеристики: \par
	$R(1) = 2\times{}(x_1 - x_0) - 4\frac{y_1}{m}$ \par
	$R(k) = 2\times{}(x_1 - x_0) - 4\frac{y_1}{m}$ \par
	$R(i) = x_i - x_{i-1} + \frac{({y_i - y_{i-1}})^2}{m^2(x_i-x_{-1})}, 0 < i < k$ \par
	Данные характеристики рассматриваются как некоторые меры вероятности локализации в данном интервале точки глобального минимума.
	\item Найти интервал с максимальной характеристикой$R(t)$.
	\item Добавить новое звено в неравномерную сетку испытаний:\par
	$x_{k+1} = \frac{x_t - x_{t -1}}{2} - \frac{y_i - y_{i - 1}}{2m}$
	\item Если $x_t - x_{t -1} < \epsilon$, где  $\epsilon$ - заданная точность, то выполнение алгоритма прекращается, а искомой точкой является $x_1$ . В противном случае необходимо вернуться на п.1.
\end{enumerate}
\newpage

% Схема распараллеливания
\section*{Схема распараллеливания}
\addcontentsline{toc}{section}{Схема распараллеливания}
Параллельный вариант метода Стронгина предусматривает разбиение области поиска на части, запуске на каждой из подобластей последовательного варианта алгоритма и выбор из всех локальных решений той точки, в которой значение функции минимально.
\par Исходя из того, что алгоритм нахождения глобального минимума предусматривает построение неравномерной сетки испытаний на заданном отрезке, деление $[a,b]$ на равные части является нецелесообразным, поскольку нагрузка на процессы будет ложиться неравномерно. Предлагается следующая схема распараллеливания:\par
\begin{enumerate}
	\item На нулевом процессе провести 64 итерации последовательного метода Стронгина.
	\item В случае, если решение не найдено с заданной точностью, вычислить количество отрезков, на которых каждый процесс будет запускать последовательный алгоритм, по следующей формуле:\par
	$h = \frac{64}{N}$, где $N$ - количество процессов
	\item Границы интервала, на котором будет запущен последовательный вариант алгоритма, будут выглядеть следующим образом:\par
	$a = x_{i\times{}h}, b = x_{(i+1)\times{}h}, i = \overline{0,N} $
	\item Запустить последовательный вариант алгоритма на отрезках, границы которых вычислены в п.3, получив локальный результат на каждом процессе.
	\item Собрав на нулевом процессе локальные результаты со всех процессов, выбрать из них точку с наименьшим значением функции. Данная точка будет являться результатом выполнения параллельного алгоритма.
\end{enumerate}
\par Стоит отметить, что распараллеливание путем разделения области поиска в целом является не самым эффективным методом распараллеливания. Исходя из смысла характеристик интервалов, который заключается в вероятности нахождения на данном интервале глобального оптимума, гораздо эффективнее отдавать процессам интервалы с максимальной характеристикой, а не делить область на части.
\newpage

% Описание программной реализации
\section*{Описание программной реализации}
\addcontentsline{toc}{section}{Описание программной реализации}
В программе реализован класс StronginMethod, представляющий задачу глобальной оптимизации. Класс имеет следующие поля со спецификатором доступа «protected»:\par
\begin{itemize}
	\item Левая и правая границы области поиска\par
	\begin{lstlisting}
		double left_border;
	\end{lstlisting}\par
	\begin{lstlisting}
		double right_border;
	\end{lstlisting}\par
	 \item Указатель на функцию, глобальный минимум которой необходимо найти \par
	\begin{lstlisting}
		std::function<double(double*)> Given_Function;
	\end{lstlisting}\par
	\item Точность, с которой задача должна быть решена \par
	\begin{lstlisting}
		double precision;
	\end{lstlisting}\par
\end{itemize}

	 Вспомогательные методы со спецификатором доступа «protected»:\par
\begin{itemize}	 
	\item Вычисление значения новой точки для добавления в неравномерную сетку \par
	\begin{lstlisting}
		double Point(int t, double Lconst2, const std::vector<double>& array);
	\end{lstlisting}\par
    \item Вычисление оценки константы Липшица заданного интервала; \par
	\begin{lstlisting}
		double Lipsh_Const1(int index, const std::vector<double>& array);
	\end{lstlisting}\par
		\begin{lstlisting}
		double Lipsh_Const2(double Lconst1, double r);
	\end{lstlisting}\par
	\item Вычисление характеристики заданного интервала \par
	\begin{lstlisting}
		double Interval_characteristic(int index, double Lconst1, const std::vector<double>& array);
	\end{lstlisting}\par
	\item Значение заданной функции от аргумента $x$ \par
	\begin{lstlisting}
		double Value(double x);
	\end{lstlisting}\par
\end{itemize}

    Методы класса со спецификатором доступа «public», предоставляющие решение задачи глобальной оптимизации:\par
    \begin{itemize}
	\item Последовательная реализация метода Стронгина \par	
	\begin{lstlisting}
		double Find_Sequential(int count_It = 1000);
	\end{lstlisting}\par
	\item Параллельная реализация метода Стронгина \par	
	\begin{lstlisting}
		double Find_Parallel(int count_It = 1000);
	\end{lstlisting}\par
\end{itemize}
\newpage

% Подтверждение корректности
\section*{Подтверждение корректности}
\addcontentsline{toc}{section}{Подтверждение корректности}
С целью подтверждения корректности разработанного алгоритма в программе реализованы тесты с использованием Google C++ Testing Framework – библиотеки для модульного тестирования. Всего было реализовано 5 тестов. 
\par Tесты подтверждают эквивалентность последовательного и параллельного методов на примере нескольких функций.
\newpage

% Результаты экспериментов
\section*{Результаты экспериментов}
\addcontentsline{toc}{section}{Результаты экспериментов}
Для проведения экспериментов использовался компьютер со следующими аппаратными характеристиками:
\begin{itemize}
\item Процессор: Intel Core i3-3220, 3300 MHz, ядер: 2;
\item Оперативная память: 8 Gb, 1600 MHz;
\item ОС: Microsoft Windows 10 Pro;
\item Среда разработки: Visual Studio 2019.
\end{itemize}

\par Функция, поиск глобального минимума которой необходимо найти, выглядит следующим образом:
 $f(x) = \sin(9x + 4)\times{}\cos(15x -2) + 1.4 $
\par Результаты экспериментов представлены в Таблице 1.

\begin{table}[!h]
\caption{Результаты проведения вычислительных экспериментов}
\centering
\begin{tabular}{|l|l|l|l|l|}
\hline
Точность & Последовательно & Параллельно 2 процесса & Параллельно 4 процесса  \\ \hline
$10^{-6}$        & 0,369646        & 0,259118     & 0,226895      \\
$10^{-4}$        & 3,54994         & 2,9212     & 2,371086       \\
\hline
\end{tabular}
\end{table}

\par По данным, полученным в результате проведения экспериментов, можно сделать вывод о том, что параллельный вариант метода Стронгина работает быстрее последовательного. Так, ускорение на 2-х процессах при точности поиска $10^{-6}$ составляет $1.4$, на 4-х процессах - $1.63$. При точности поиска $10^{-7}$ ускорение на 2-х процессах составляет 1,22, на 4-х – 1,49. Однако при большом количестве процессов появляются регрессии. Это объясняется переключениями между процессами и обменом данными между ними при коллективных операциях.Также некоторые процессы будут ожидать своей очереди на выполнение операций из-за недостатка вычислительной мощности.
\par Основываясь на результатах экспериментов, можно сделать вывод, что при данной аппаратной конфигурации  эффективным будет использование 4-5 процессов для данного алгоритма.

% Заключение
\section*{Заключение}
\addcontentsline{toc}{section}{Заключение}
В ходе проделанной работы был успешно реализован метод Стронгина решения задачи глобальной оптимизации в последовательном и параллельном вариантах. Работоспособность программы была проверена с помощью библиотеки для модульного тестирования.
\par Вычислительный эксперимент показал, что параллельный вариант алгоритма превосходит в производительности последовательный вариант. На примере было показано, что метод, созданный с помощью программного интерфейса MPI, является предпочтительным в использовании.
\par Также было выяснено, что распараллеливание метода Стронгина путём разделения области поиска является не самым оптимальным вариантом распараллеливания, поскольку по причине неравномерности сетки проведённых испытаний нагрузка на процессы ложится неравномерно.
\newpage

% Список литературы
\begin{thebibliography}{1}
\addcontentsline{toc}{section}{Список литературы}
\bibitem{1} Баркалов К.А., Сысоев А.В., Гергель В.П. «Решение задач глобальной оптимизации на гетерогенных
кластерных системах». Нижний Новгород, 2015. 
\bibitem{2} Гришагин В.А., Исрафилов Р.А. «Параллельная реализация адаптивной многошаговой схемы
редукции размерности для задач глобальной оптимизации». Нижний Новгород, 2017.
\bibitem{3} Стронгин Р.Г. Гергель В.П., Баркалов К.А «ПАРАЛЛЕЛЬНЫЕ МЕТОДЫ РЕШЕНИЯ ЗАДАЧ
ГЛОБАЛЬНОЙ ОПТИМИЗАЦИИ». Нижний Новгород, 2009.
\bibitem{4} Центр суперкопьютерных технологий [Электронный ресурс] // URL: \url {http://www.hpcc.unn.ru/?dir=893} (дата обращения: 9.12.2020)
\end{thebibliography}
\newpage

% Приложение
\section*{Приложение}
\addcontentsline{toc}{section}{Приложение}
В данном разделе находится листинг кода, созданного в рамках лабораторной работы. \par
\texttt{Заголовочный файл gopt.h}
\begin{lstlisting}
// Copyright 2020 Molotkova Svetlana
#ifndef MODULES_TASK_3_MOLOTKOVA_S_GOPT_GOPT_H_
#define MODULES_TASK_3_MOLOTKOVA_S_GOPT_GOPT_H_
#include <functional>
#include <vector>

class StronginMethod {
	double left_border;
	double right_border;
	std::function<double(double*)> Given_Function;
	double precision;
	
	double Point(int t, double Lconst2, const std::vector<double>& array);
	double Interval_characteristic(int index, double Lconst1, const std::vector<double>& array);
	double Lipsh_Const1(int index, const std::vector<double>& array);
	double Lipsh_Const2(double Lconst1, double r);
	double Value(double x);
	public :
	StronginMethod(double _left_border, double _right_border,
	std::function<double(double*)> _Given_Function, double precision);
	double Find_Sequential(int count_It = 1000);
	double Find_Parallel(int count_It = 1000);
};

#endif  // MODULES_TASK_3_MOLOTKOVA_S_GOPT_GOPT_H_

\end{lstlisting}
\texttt{Файл с реализацией функций gopt.cpp}
\begin{lstlisting}
// Copyright 2020 Molotkova Svetlana
#include <mpi.h>
#include <utility>
#include <list>
#include <algorithm>
#include <vector>
#include <iostream>
#include <cmath>
#include "../../../modules/task_3/molotkova_s_gopt/gopt.h"

StronginMethod::StronginMethod(double _left_border, double _right_border,
std::function<double(double*)> _Given_Function, double _precision) {
	left_border = _left_border;
	right_border = _right_border;
	Given_Function = _Given_Function;
	precision = _precision;
}

double StronginMethod::Value(double x) {
	double* arg = &x;
	return Given_Function(arg);
}

double Sign(double x) {
	if (x >= 0)
	return x;
	else
	return -x;
}

double StronginMethod::Lipsh_Const1(int index, const std::vector<double>& array) {
	return Sign(Value(array[index]) - Value(array[index - 1])) / (array[index] - array[index - 1]);
}

double StronginMethod::Lipsh_Const2(double Lconst1, double r) {
	if (Lconst1 == 0)
	return 1;
	else
	return r * Lconst1;
}

double StronginMethod::Interval_characteristic(int index, double Lconst1, const std::vector<double>& array) {
	double x_diff = array[index] - array[index - 1];
	if (index == 1) {
		return 2 * x_diff - 4 * Value(array[index]) / Lconst1;
	} else {
		double y_diff_in_pow = pow((Value(array[index]) - Value(array[index - 1])), 2);
		double interval_characteristic = x_diff + y_diff_in_pow / (Lconst1 * Lconst1 * x_diff) -
		2 * (Value(array[index]) + Value(array[index - 1])) / Lconst1;
		return  interval_characteristic;
	}
}

double StronginMethod::Point(int index, double Lconst2, const std::vector<double>& array) {
	double arg = (Value(array[index]) - Value(array[index - 1])) / (2 * Lconst2);
	double point = (array[index] + array[index - 1]) / 2 - arg;
	return point;
}

double StronginMethod::Find_Sequential(int count_It) {
	std::vector<double> x(count_It + 1, -1e+300);
	int nIteration = 1;
	int t = 1;
	double r = 2;
	double R, tmp;
	x[0] = left_border;
	x[1] = right_border;
	double Lconst1 = Lipsh_Const1(1, x);
	double Lconst2 = Lipsh_Const2(Lconst1, r);
	x[2] = Point(1, Lconst2, x);
	++nIteration;
	while (nIteration < count_It) {
		sort(x.begin(), x.begin() + nIteration + 1);
		Lconst1 = Lipsh_Const1(1, x);
		for (int i = 2; i <= nIteration; ++i) {
			Lconst1 = std::max(Lconst1, Lipsh_Const1(i, x));
		}
		Lconst2 = Lipsh_Const2(Lconst1, r);
		R = Interval_characteristic(1, Lconst1, x);
		t = 1;
		for (int i = 2; i < nIteration; ++i) {
			tmp = Interval_characteristic(i, Lconst1, x);
			if (R < tmp) {
				R = tmp;
				t = i;
			}
		}
		tmp = 2 * (x[nIteration] - x[nIteration - 1]) - 4 * Value(x[nIteration - 1]) / Lconst1;
		if (R < tmp) {
			R = tmp;
			t = nIteration;
		}
		x[nIteration + 1] = Point(t, Lconst2, x);
		++nIteration;
		if (x[t] - x[t - 1] < precision) {
			break;
		}
	}
	return x[t];
}
double StronginMethod::Find_Parallel(int count_It) {
	int procNum;
	int porog = 64;
	MPI_Comm_size(MPI_COMM_WORLD, &procNum);
	if (procNum == 1) {
		return Find_Sequential(count_It);
	}
	int procRank;
	MPI_Comm_rank(MPI_COMM_WORLD, &procRank);
	double localRes;
	std::vector<double> result(procNum);
	std::vector<double> x(porog + 1, -1e+300);
	int nIteration = 1;
	int t = 1;
	double r = 2;
	double R, tmp;
	
	x[0] = left_border;
	x[1] = right_border;
	double M = Lipsh_Const1(1, x);
	double m = Lipsh_Const2(M, r);
	x[2] = Point(1, m, x);
	++nIteration;
	
	while (nIteration < porog) {
		sort(x.begin(), x.begin() + nIteration + 1);
		
		M = Lipsh_Const1(1, x);
		for (int i = 2; i <= nIteration; ++i) {
			M = std::max(M, Lipsh_Const1(i, x));
		}
		m = Lipsh_Const2(M, r);
		R = Interval_characteristic(1, M, x);
		t = 1;
		
		for (int i = 2; i < nIteration; ++i) {
			tmp = Interval_characteristic(i, M, x);
			if (R < tmp) {
				R = tmp;
				t = i;
			}
		}
		tmp = 2 * (x[nIteration] - x[nIteration - 1]) - 4 * Value(x[nIteration - 1]) / M;
		if (R < tmp) {
			R = tmp;
			t = nIteration;
		}
		
		x[nIteration + 1] = Point(t, m, x);
		++nIteration;
		if (x[t] - x[t - 1] < precision) {
			break;
		}
	}
	sort(x.begin(), x.begin() + nIteration + 1);
	double h = porog / procNum;
	double locA, locB;
	if (procRank != procNum - 1) {
		locA = x[procRank * h];
		locB = x[(procRank + 1) * h];
	} else {
		locA = x[procRank * h];
		locB = x[porog - 1];
	}
	StronginMethod opt(locA, locB, Given_Function, precision);
	localRes = opt.Find_Sequential(count_It);
	MPI_Gather(&localRes, 1, MPI_DOUBLE, &result[0], 1, MPI_DOUBLE, 0, MPI_COMM_WORLD);
	if (procRank == 0) {
		for (int i = 1; i < procNum; ++i) {
			if (Value(result[i]) < Value(result[0])) {
				std::swap(result[i], result[0]);
			}
		}
	}
	return result[0];
}

\end{lstlisting}
\texttt{Файл с реализацией тестов main.cpp}
\begin{lstlisting}
// Copyright 2020 Molotkova Svetlana
#include <gtest-mpi-listener.hpp>
#include <gtest/gtest.h>
#include <iostream>
#include <cmath>
#include "./gopt.h"


double f(double* x) {
	return  *x * *x;
}
double testF(double* _x) {
	double x = *_x;
	return sin(9 * x + 4) * cos(15 * x - 2) + 1.4;
}
double f1(double* x) {
	return exp(-0.5 * *x) * sin(6 * *x - 1.5) + 0.1 * *x;
}
double f2(double* _x) {
	double x = *_x;
	return sin(x + 5) * (x + 5);
}

double f3(double* _x) {
	double x = *_x;
	return exp(-0.9 * x);
}

TEST(global_optimization, ts1) {
	const int count_It = 1000;
	int myrank;
	MPI_Comm_rank(MPI_COMM_WORLD, &myrank);
	StronginMethod ts(5, 10, f1, 1e-5);
	double time1 = MPI_Wtime();
	double result = ts.Find_Parallel(count_It);
	double time2 = MPI_Wtime();
	if (myrank == 0) {
		double stime1 = MPI_Wtime();
		double sresult = ts.Find_Sequential(count_It);
		double stime2 = MPI_Wtime();
		double part = time2 - time1;
		double st = stime2 - stime1;
		if (part > st)
		std::cout << "NOT OK " << part - st << std::endl;
		else
		std::cout << " OK" << std::endl;
		ASSERT_NEAR(result, sresult, 1e-2);
	}
}

TEST(global_optimization, ts2) {
	const int count_It = 1000;
	int myrank;
	MPI_Comm_rank(MPI_COMM_WORLD, &myrank);
	StronginMethod ts(0.6, 2.2, testF, 1e-5);
	double time1 = MPI_Wtime();
	double result = ts.Find_Parallel(count_It);
	double time2 = MPI_Wtime();
	if (myrank == 0) {
		double stime1 = MPI_Wtime();
		double sresult = ts.Find_Sequential(count_It);
		double stime2 = MPI_Wtime();
		double part = time2 - time1;
		double st = stime2 - stime1;
		if (part > st)
		std::cout << "NOT OK" << std::endl;
		else
		std::cout << " OK" << std::endl;
		ASSERT_NEAR(result, sresult, 1e-2);
	}
}

TEST(global_optimization, ts3) {
	const int count_It = 1000;
	int myrank;
	MPI_Comm_rank(MPI_COMM_WORLD, &myrank);
	StronginMethod ts(-6, 10, f, 1e-5);
	double time1 = MPI_Wtime();
	double result = ts.Find_Parallel(count_It);
	double time2 = MPI_Wtime();
	if (myrank == 0) {
		double stime1 = MPI_Wtime();
		double sresult = ts.Find_Sequential(count_It);
		double stime2 = MPI_Wtime();
		double part = time2 - time1;
		double st = stime2 - stime1;
		if (part > st)
		std::cout << "NOT OK" <<std::endl;
		else
		std::cout << " OK" << std::endl;
		ASSERT_NEAR(result, sresult, 1e-2);
	}
}

TEST(global_optimization, ts4) {
	const int count_It = 1000;
	int myrank;
	MPI_Comm_rank(MPI_COMM_WORLD, &myrank);
	StronginMethod ts(5, 10, f2, 1e-5);
	double time1 = MPI_Wtime();
	double result = ts.Find_Parallel(count_It);
	double time2 = MPI_Wtime();
	if (myrank == 0) {
		double stime1 = MPI_Wtime();
		double sresult = ts.Find_Sequential(count_It);
		double stime2 = MPI_Wtime();
		double part = time2 - time1;
		double st = stime2 - stime1;
		if (part > st)
		std::cout << "NOT OK" << part - st << std::endl;
		else
		std::cout << " OK" << std::endl;
		ASSERT_NEAR(result, sresult, 1e-2);
	}
}

TEST(global_optimization, ts5) {
	const int count_It = 1000;
	int myrank;
	MPI_Comm_rank(MPI_COMM_WORLD, &myrank);
	StronginMethod ts(5, 10, f3, 1e-5);
	double time1 = MPI_Wtime();
	double result = ts.Find_Parallel(count_It);
	double time2 = MPI_Wtime();
	if (myrank == 0) {
		double stime1 = MPI_Wtime();
		double sresult = ts.Find_Sequential(count_It);
		double stime2 = MPI_Wtime();
		double part = time2 - time1;
		double st = stime2 - stime1;
		if (part > st)
		std::cout << "NOT OK" << part - st << std::endl;
		else
		std::cout << " OK" << std::endl;
		ASSERT_NEAR(result, sresult, 1e-2);
	}
}

int main(int argc, char** argv) {
	::testing::InitGoogleTest(&argc, argv);
	MPI_Init(&argc, &argv);
	
	::testing::AddGlobalTestEnvironment(new GTestMPIListener::MPIEnvironment);
	::testing::TestEventListeners& listeners =
	::testing::UnitTest::GetInstance()->listeners();
	
	listeners.Release(listeners.default_result_printer());
	listeners.Release(listeners.default_xml_generator());
	
	listeners.Append(new GTestMPIListener::MPIMinimalistPrinter);
	return RUN_ALL_TESTS();
}

\end{lstlisting}

\end{document}
