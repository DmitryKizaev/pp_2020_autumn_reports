\documentclass{report}

\usepackage[T2A]{fontenc}
\usepackage[utf8]{luainputenc}
\usepackage[english, russian]{babel}
\usepackage[pdftex]{hyperref}
\usepackage[14pt]{extsizes}
\usepackage{listings}
\usepackage{color}
\usepackage{geometry}
\usepackage{enumitem}
\usepackage{multirow}
\usepackage{graphicx}
\usepackage{indentfirst}

\geometry{a4paper,top=2cm,bottom=3cm,left=2cm,right=1.5cm}
\setlength{\parskip}{0.5cm}
\setlist{nolistsep, itemsep=0.3cm,parsep=0pt}

\lstset{language=C++,
		basicstyle=\footnotesize,
		keywordstyle=\color{blue}\ttfamily,
		stringstyle=\color{red}\ttfamily,
		commentstyle=\color{green}\ttfamily,
		morecomment=[l][\color{magenta}]{\#}, 
		tabsize=4,
		breaklines=true,
  		breakatwhitespace=true,
  		title=\lstname,       
}

\makeatletter
\renewcommand\@biblabel[1]{#1.\hfil}
\makeatother

\begin{document}

\begin{titlepage}

\begin{center}
Министерство науки и высшего образования Российской Федерации
\end{center}

\begin{center}
Федеральное государственное автономное образовательное учреждение высшего образования \\
Национальный исследовательский Нижегородский государственный университет им. Н.И. Лобачевского
\end{center}

\begin{center}
Институт информационных технологий, математики и механики
\end{center}

\vspace{4em}

\begin{center}
\textbf{\LargeОтчет по лабораторной работе} \\
\end{center}
\begin{center}
\textbf{\Large «Быстрая сортировка с четно-нечетным слиянием Бэтчера.»}
\vspace{4em}
\end{center}
\newbox{\lbox}
\savebox{\lbox}{\hbox{text}}
\newlength{\maxl}
\setlength{\maxl}{\wd\lbox}
\hfill\parbox{7cm}{
\hspace*{5cm}\hspace*{-5cm}\textbf{Выполнил:} \\ студент группы 381806-2 \\ Пестреев Д. С.\\
\\
\hspace*{5cm}\hspace*{-5cm}\textbf{Проверил:}\\ доцент кафедры МОСТ, \\ кандидат технических наук \\ Сысоев А. В.\\
}
\vspace{\fill}

\begin{center} Нижний Новгород \\ 2020 \end{center}

\end{titlepage}

\setcounter{page}{2}

% Содержание
\tableofcontents
\newpage

% Введение
\section*{Введение}
\addcontentsline{toc}{section}{Введение}
Сортировкой называют процесс перегруппировки заданного множества объектов в некотором определенном порядке. 
Цель сортировки — облегчение последующего поиска элементов в отсортированном множестве во внутренней или внешней памяти. В последнем случае она особенно важна, если имеется только последовательный доступ к файлам (появляется возможность приемлемой замены прямого доступа).
\par Сортировка в той или иной мере присутствуют во всех приложениях. При обработке больших объемов данных эффективность именно этих операций определяет продуктивность, а иногда и работоспособность всей системы.
\par Одним из самых быстрых алгоритмов сортировки является сортировка Хоара. Среднее время работы данного алгоритма O(nlogn).  Но при увеличении объема данных алгоритмы увеличивается и время обработки может быть нерпиемлемо большим, в том числе и для алгоритма Хоара. Чтобы решить данную проблему, алгоритм сортировки можно сделать параллельным. Для этого исходные данные делятся на несколько массивов, сортируются последовательно и сливаются в один отсортированный массив. 
\newpage

% Постановка задачи
\section*{Постановка задачи}
\addcontentsline{toc}{section}{Постановка задачи}
Целью данной лабораторной работы является реализовать последовательный алгоритм быстрой сортировки и параллельный алгоритм быстрой сортировки с помощью четно-нечетного слияния Бэтчера.  Необходимо провести замеры времени работы алгоритмов и сделать выводы по полученным результатам.
\par Для параллельной реализации будут использоваться средства MPI. Проверка алгоритмов на корректность будут осуществляться через Google C++ Testing Framework.
\newpage

% Описание метода решения
\section*{Описание метода решения}
\addcontentsline{toc}{section}{Описание метода решения}
Алгоритм быстрой сортировки основан на операциях сравнения и перестановки. Последовательность действий выглядит следующим образом:
\begin{enumerate}
\item	Необходимо выбрать опорный элемент. Стать им может каждый элемент в массиве. От выбора опорного элемента может зависеть производительность. В данной работе в качестве опорного алгоритма будет использоваться элемент по середине.
\item Выполнить сранение оставшихся элементов с опорным и перераспределить так, чтобы было 3 набора: больше опорного, меньше опорного и равные опорному.
\item Для наборов, меньших и больших опорного, повторить алгоритм рекурсивно до тех пор, пока эти подмассивы больше 1.
\end{enumerate}
\par Худший случай может прозойти, если в качестве опорного элемента выбирается наибольший или наименьший элемент массива.
\newpage

% Распараллеливание
\section*{Распараллеливание}
\addcontentsline{toc}{section}{Распараллеливание}
Для параллельной реализации необходимо разделить исходный массив на части, количество которых равно количеству процессов. В случае, если невозможно поделить на равные сегменты, добавляем к массиву фиктивные переменные, которые больше или меньше всех элементов, то есть можно взять максимальное или минимальное значение для данного типа данных. Потом распререляем их по процессам.
\par Далее каждая часть сортируется быстрой сортировкой. Затем строится сеть обменной сортировки Бэтчера. Данный алгоритм принимает на вход массив номеров процессов и заполняет массив компараторов, которые хранят пары номеров процессов. Алгоритм построения сети работает следующим образом:
\begin{enumerate}
\item Массив номеров процессов делится на 2 части;
\item Данные сегменты рекурсивно сливаются  по четным и нечетным элементам;
\item Записываем элементы в массив компараторов;
\item Рекурсивное повторение до тех пор, пока длина сегмента больше 1.
\end{enumerate}
\par После построения сети Бэтчера берем пары процессов, записанные в массиве компараторов. Каждый процесс из пары посылает свой фрагмент и посылает чужой. Далее они сливаются так, что в одном процессе сохраняется меньший массив, а в другом больший. В итоге все сегменты отправляются в нулевой процесс и компонуются в массив-результат в порядке номера процесса.

\newpage

% Описание программной реализации
\section*{Описание программной реализации}
\addcontentsline{toc}{section}{Описание программной реализации}
В течение лабораторной работы были написаны следующие функции:
\begin{lstlisting}
std::vector<int> getRandomVector(int size);
\end{lstlisting}
\par Данная функция возвращает случайно сгенерированный вектор заданной длины.
\begin{lstlisting}
void qsort(int* vec, int left, int right);
\end{lstlisting}
\par Данная функция в качестве параметров принимает одномерный массив значений, начальный и конечный индекс массива. В качестве опорного элемента берется элемент, индекс которого равен среднему арифметическому начального и конечного индекса массива.
\begin{lstlisting}
std::vector<int> quickSortV(const std::vector<int>& vec);
\end{lstlisting}
\par Данная функция принимает константную ссылку на вектор и возвращает отсортированный результат. Так как значения входного вектора берутся по константной ссылке, невозможно реализовать для него рекуррентный алгоритм быстрой сортировки. Поэтому значения вектора записываются в одномерный динамический массив и сортируются функцией qsort. Отсортированный массив передается в результат в виде вектора.
\begin{lstlisting}
void recur_merge(const std::vector<int>& left, const std::vector<int>& right);
\end{lstlisting}
\par Данная функция принимает ссылки на векторы, которые нужно слить по четным и нечетным индексам и заполняет массив компараторов. Вызывается в функции networking.
\begin{lstlisting}
void networking(const std::vector<int>& arr);
\end{lstlisting}
\par Данная функция принимает константную сслылку на вектор, содержащий номера процессов. Далее происходит деление вектора пополам. Потом происходит рекурсивный вызов и функция recur\_merge для данных половин.
\begin{lstlisting}
void batchers_network(int proc_size);
\end{lstlisting}
\par Функция принимает количество процессов, создает вектор, заполненный номерами процессов и вызывает функцию networking.
\begin{lstlisting}
std::vector<int> parallel_sorting(const std::vector<int>& vec);
\end{lstlisting}
\par Функция принимает массив в виде вектора, который нужно отсортировать. В конце работы возвращает уже отсортированный массив. В данной функции реализована параллельная часть программы. Данные разделяются между процессами и сортируются алгоритмом сортировки Хоара. Далее строится сеть Бэтчера. После постройки берутся пары процессов из массива компараторов и сливаются части, хранящиеся в данных парах. После слияния все части собираются в один массив, который является результатом.  

\newpage

% Подтверждение корректности
\section*{Подтверждение корректности}
\addcontentsline{toc}{section}{Подтверждение корректности}
Для подтверждения корректности в программе представлен набор тестов, разработанных с помощью использования Google C++ Testing Framework.
\par Данные тесты проверяют корректность работы алгоритма(результаты работы сортировки из стандартной библиотеки С++  и параллельного алгоритма сравниваются друг с другом) и эффективность параллельного алгоритма по сравнению с последовательным(сравнение по времени).
\par Успешное прохождение всех тестов доказывает корректность работы всей программы.
\newpage

% Результаты экспериментов
\section*{Результаты экспериментов}
\addcontentsline{toc}{section}{Результаты экспериментов}
Вычислительные эксперименты для оценки эффективности параллельного алгоритма быстрой сортировки четно-нечетным слиянием Бэтчера производились на оборудовании со следующей аппаратной конфигурацией:

\begin{itemize}
\item Процессор: Intel(R) Core(TM) i3-6006U CPU @ 2.00 GHz, 2.00 GHz, 2 ядра, 4 потока;
\item Оперативная память: 16384 МБ (DDR4);
\item ОС: Microsoft Windows 10 Pro. 
\end{itemize}

\par Для эксперимента будем исспользовать вектор размера 18523140, заполненный случайными целыми числами. 
\par Результаты экспериментов представлены в таблице
\begin{table}[!h]
\caption{Результаты вычислительных экспериментов}
\centering
\begin{tabular}{lllll}
Кол-во процессов  & Последовательно & Параллельно \\
1      & 4.04227         & 4.20516  \\
2     & 4.03881         & 2.78243  \\
3     & 3.72178         & 2.17409 \\
4     & 3.66811        & 2.01871 \\
5     &3.72273         & 2.19946  \\
6     &3.77097         & 2.35688  

\end{tabular}
\end{table}

\par Полученные результаты говорят о том, что параллельный алгоритм работает быстрее последовательного. Но с увеличением количества процессов эффектвность программы начинает падать. Это связано с большим количеством сравнений и издержками коллективных операциях приема и передачи данных.
\par Для данной системы наиболее эффективным будет использование 4 процессов.
\newpage

% Вывод
\section*{Вывод}
\addcontentsline{toc}{section}{Вывод}
В результате выполнения лабораторной работы созданы последовательный алгоритм быстрой сортировки и параллельный с четно-нечетным слиянием Бэтчера.
\par Задача состояла в разработке параллельного алгоритма, который будет эффективнее последовательного. Результаты экспериментов говорят о том, что задача выполнена.
\par Также результаты экспериментов позволили сравнить время работы алгоритмов. Для подтверждения корректности работы программы был разработан набор тестов Google C++
Testing Framework, которые были успешно пройдены.

\newpage

% Список литературы
\begin{thebibliography}{1}
\addcontentsline{toc}{section}{Список литературы}
\bibitem{Hoare} 
Hoare C. A. R. Quicksort //The Computer Journal. – 1962. – Т. 5. – №. 1. – С. 10-16.
\bibitem{Gergel}
Гергель В. П. Теория и практика параллельных вычислений. – 2007. 
\bibitem{Gergel}
Гергель В. П., Стронгин Р. Г. Основы параллельных вычислений для многопроцессорных вычислительных систем. – 2003.
\bibitem{Habr} Сеть обменной сортировки со слиянием Бэтчера [Электронный ресурс] // URL: \url{https://habr.com/ru/post/275889/}
\end{thebibliography}
\newpage

% Приложение
\section*{Приложение}
\addcontentsline{toc}{section}{Приложение}
В данном разделе находится листинг всего кода, написанного в рамках лабораторной работы.
\begin{lstlisting}
// quick_sort_even_odd_merge.h

// Copyright 2020 Pestreev Daniil
#ifndef MODULES_TASK_3_PESTREEV_D_QUICK_SORT_EVEN_ODD_QUICK_SORT_EVEN_ODD_MERGE_H_
#define MODULES_TASK_3_PESTREEV_D_QUICK_SORT_EVEN_ODD_QUICK_SORT_EVEN_ODD_MERGE_H_

#include <vector>
#include <string>

std::vector<int> getRandomVector(int size);
void Swap(int a, int b);
void qsort(int* vec, int left, int right);
std::vector<int> quickSortV(const std::vector<int>& vec);
void recur_merge(const std::vector<int>& left, const std::vector<int>& right);
void networking(const std::vector<int>& arr);
void batchers_network(int proc_size);
std::vector<int> parallel_sorting(const std::vector<int>& vec);


#endif  // MODULES_TASK_3_PESTREEV_D_QUICK_SORT_EVEN_ODD_QUICK_SORT_EVEN_ODD_MERGE_H_

\end{lstlisting}
\begin{lstlisting}
// quick_sort_even_odd_merge.cpp

// Copyright 2020 Pestreev Daniil
#include <mpi.h>
#include <iostream>
#include <vector>
#include <random>
#include <limits>
#include <utility>
#include <ctime>
#include "../../../modules/task_3/pestreev_d_quick_sort_even_odd/quick_sort_even_odd_merge.h"

std::vector<std::pair<int, int>> comps;

std::vector<int> getRandomVector(int size) {
    std::mt19937 gen;
    gen.seed(time(0));
    std::vector<int> a(size);

    for (int i = 0; i < size; i++) {
        a[i] = gen();
    }
    return a;
}

void qsort(int* vec, int left, int right) {
    int mid;
    int l = left;
    int r = right;
    mid = vec[(l + r) / 2];
    do {
        while (vec[l] < mid) l++;
        while (vec[r] > mid) r--;
        if (l <= r) {
            std::swap(vec[l], vec[r]);
            l++;
            r--;
        }
    } while (l < r);
    if (left < r) qsort(vec, left, r);
    if (l < right) qsort(vec, l, right);
}

std::vector<int> quickSortV(const std::vector<int>& vec) {
    int size = vec.size();
    int* arr = new int[size];
    for (int i = 0; i < size; i++) {
        arr[i] = vec[i];
    }
    qsort(arr, 0, size - 1);
    std::vector<int> tmp;
    for (int i = 0; i < size; i++) {
        tmp.push_back(arr[i]);
    }
    return tmp;
}

void recur_merge(const std::vector<int>& left, const std::vector<int>& right) {
    int array_count = left.size() + right.size();
    if (array_count <= 1) {
        return;
    }
    if (array_count == 2) {
        comps.push_back(std::pair<int, int>(left[0], right[0]));
        return;
    }
    std::vector<int> left_odd;
    std::vector<int> left_even;
    std::vector<int> right_odd;
    std::vector<int> right_even;
    int leftsize = left.size();
    for (int i = 0; i < leftsize; i++) {
        if (i % 2) {
            left_even.push_back(left[i]);
        } else {
            left_odd.push_back(left[i]);
        }
    }
    int rightsize = right.size();
    for (int i = 0; i < rightsize; i++) {
        if (i % 2) {
            right_even.push_back(right[i]);
        } else {
            right_odd.push_back(right[i]);
        }
    }

    recur_merge(left_odd, right_odd);
    recur_merge(left_even, right_even);

    std::vector<int> res;
    for (int i = 0; i < leftsize; i++) {
        res.push_back(left[i]);
    }
    for (int i = 0; i < rightsize; i++) {
        res.push_back(right[i]);
    }

    for (int i = 1; i + 1 < array_count; i += 2) {
        comps.push_back(std::pair<int, int>(res[i], res[i + 1]));
    }
}

void networking(const std::vector<int>& arr) {
    int arr_size = arr.size();

    if (arr_size <= 1) {
        return;
    }

    std::vector<int> l(arr.begin(), arr.begin() + arr_size / 2);
    std::vector<int> r(arr.begin() + arr_size / 2, arr.begin() + arr_size);
    networking(l);
    networking(r);
    recur_merge(l, r);
}

void batchers_network(int proc_size) {
    if (proc_size <= 1)
        return;
    std::vector<int> procs;
    for (int i = 0; i < proc_size; i++) {
        procs.push_back(i);
    }

    networking(procs);
}

std::vector<int> parallel_sorting(const std::vector<int>& vec) {
    int proc_rank, proc_size;
    MPI_Comm_size(MPI_COMM_WORLD, &proc_size);
    MPI_Comm_rank(MPI_COMM_WORLD, &proc_rank);

    std::vector<int> globalV = vec;
    int vecsizeG = globalV.size();

    if (proc_size < 0) {
        return globalV;
    }

    if (proc_size >= vecsizeG || proc_size == 1) {
        if (proc_rank == 0) {
            globalV = quickSortV(vec);
        }
        return globalV;
    }
    int exitcircle = -1;
    int vecsizechange = 0;
    if (proc_rank == 0) {
        while (exitcircle < 0) {
            exitcircle = -1;
            if (globalV.size() % proc_size) {
                int o = std::numeric_limits<int>::max();
                vecsizechange++;
                globalV.push_back(o);
            } else {
                exitcircle = 1;
            }
        }
    }
    vecsizeG = globalV.size();

    if (proc_rank == 0) {
        for (int i = 1; i < proc_size; i++) {
            MPI_Send(&vecsizeG, 1, MPI_INT, i, 0, MPI_COMM_WORLD);
        }
    } else {
        MPI_Recv(&vecsizeG, 1, MPI_INT, 0, 0, MPI_COMM_WORLD, MPI_STATUS_IGNORE);
    }


    int vecsizeL = vecsizeG / proc_size;

    std::vector<int> localV(vecsizeL);
    std::vector<int> neighboringV(vecsizeL);
    std::vector<int> tmpV(vecsizeL);
    MPI_Scatter(&globalV[0], vecsizeL, MPI_INT, &localV[0], vecsizeL, MPI_INT, 0, MPI_COMM_WORLD);

    batchers_network(proc_size);
    const std::vector<int> v = localV;
    localV = quickSortV(v);
    int comps_size = comps.size();
    for (int i = 0; i < comps_size; i++) {
        if (proc_rank == comps[i].first) {
            MPI_Send(&localV[0], vecsizeL, MPI_INT, comps[i].second, 0, MPI_COMM_WORLD);
            MPI_Recv(&neighboringV[0], vecsizeL, MPI_INT, comps[i].second, 0, MPI_COMM_WORLD, MPI_STATUS_IGNORE);
            int localidx = 0;
            int neighboridx = 0;
            for (int tmp_index = 0; tmp_index < vecsizeL; tmp_index++) {
                int local = localV[localidx];
                int neighbor = neighboringV[neighboridx];
                if (local < neighbor) {
                    tmpV[tmp_index] = local;
                    localidx++;
                } else {
                    tmpV[tmp_index] = neighbor;
                    neighboridx++;
                }
            }
            localV = tmpV;
        } else if (proc_rank == comps[i].second) {
            MPI_Recv(&neighboringV[0], vecsizeL, MPI_INT, comps[i].first, 0, MPI_COMM_WORLD, MPI_STATUS_IGNORE);
            MPI_Send(&localV[0], vecsizeL, MPI_INT, comps[i].first, 0, MPI_COMM_WORLD);
            int start = vecsizeL - 1;
            int localidx = start;
            int neighboridx = start;
            for (int tmp_index = start; tmp_index >= 0; tmp_index--) {
                int local = localV[localidx];
                int neighbor = neighboringV[neighboridx];
                if (local > neighbor) {
                    tmpV[tmp_index] = local;
                    localidx--;
                } else {
                    tmpV[tmp_index] = neighbor;
                    neighboridx--;
                }
            }
            localV = tmpV;
        }
    }
    MPI_Gather(&localV[0], vecsizeL, MPI_INT, &globalV[0], vecsizeL, MPI_INT, 0, MPI_COMM_WORLD);

    if (proc_rank == 0 && vecsizechange > 0)
        globalV.erase(globalV.begin() + vecsizeG - vecsizechange, globalV.begin() + vecsizeG);
    return globalV;
}

\end{lstlisting}
\begin{lstlisting}
// main.cpp
// Copyright 2020 Pestreev Daniil
#include <mpi.h>
#include <gtest/gtest.h>
#include <gtest-mpi-listener.hpp>
#include <math.h>
#include <vector>
#include <random>
#include <ctime>
#include <algorithm>

#include "../../../modules/task_3/pestreev_d_quick_sort_even_odd/quick_sort_even_odd_merge.h"

TEST(TestQuickSort, IntroducedVector) {
    int size, rank;
    MPI_Comm_size(MPI_COMM_WORLD, &size);
    MPI_Comm_rank(MPI_COMM_WORLD, &rank);
    std::vector<int> vec = {-8098, 1009, -160, -1, 179796, 2,
        -1603, 166, -7396, -199, -18348, -15, 659, 7, 82519};
    double startTime = MPI_Wtime();
    std::vector<int> resP = parallel_sorting(vec);
    double finishTime = MPI_Wtime();
    double paralleltime = finishTime - startTime;
    if (rank == 0) {
        std::vector<int> resS = vec;
        double startTime = MPI_Wtime();
        std::sort(resS.begin(), resS.end());
        double sequentialtime = MPI_Wtime() - startTime;
        std::cout << "Sequential sort:" << sequentialtime << std::endl;
        std::cout << "Parallel sort:" << paralleltime << std::endl;
        ASSERT_EQ(resS, resP);
    }
}

TEST(TestQuickSort, SameNumber1000) {
    int size, rank;
    MPI_Comm_size(MPI_COMM_WORLD, &size);
    MPI_Comm_rank(MPI_COMM_WORLD, &rank);
    std::vector<int> vec(1000);
    std::fill(vec.begin(), vec.end(), 8);
    double startTime = MPI_Wtime();
    std::vector<int> resP = parallel_sorting(vec);
    double finishTime = MPI_Wtime();
    double paralleltime = finishTime - startTime;
    if (rank == 0) {
        std::vector<int> resS = vec;
        double startTime = MPI_Wtime();
        std::sort(resS.begin(), resS.end());
        double sequentialtime = MPI_Wtime() - startTime;
        std::cout << "Sequential sort:" << sequentialtime << std::endl;
        std::cout << "Parallel sort:" << paralleltime << std::endl;
        ASSERT_EQ(resS, resP);
    }
}

TEST(TestQuickSort, randomVector3) {
    int size, rank;
    MPI_Comm_size(MPI_COMM_WORLD, &size);
    MPI_Comm_rank(MPI_COMM_WORLD, &rank);
    int vecsize = 3;
    std::vector<int> vec = getRandomVector(vecsize);
    double startTime = MPI_Wtime();
    std::vector<int> resP = parallel_sorting(vec);
    double finishTime = MPI_Wtime();
    double paralleltime = finishTime - startTime;
    if (rank == 0) {
        std::vector<int> resS = vec;
        double startTime = MPI_Wtime();
        std::sort(resS.begin(), resS.end());
        double sequentialtime = MPI_Wtime() - startTime;
        std::cout << "Sequential sort:" << sequentialtime << std::endl;
        std::cout << "Parallel sort:" << paralleltime << std::endl;
        ASSERT_EQ(resS, resP);
    }
}

TEST(TestQuickSort, randomVector131) {
    int size, rank;
    MPI_Comm_size(MPI_COMM_WORLD, &size);
    MPI_Comm_rank(MPI_COMM_WORLD, &rank);
    int vecsize = 131;
    std::vector<int> vec = getRandomVector(vecsize);
    double startTime = MPI_Wtime();
    std::vector<int> resP = parallel_sorting(vec);
    double finishTime = MPI_Wtime();
    double paralleltime = finishTime - startTime;
    if (rank == 0) {
        std::vector<int> resS = vec;
        double startTime = MPI_Wtime();
        std::sort(resS.begin(), resS.end());
        double sequentialtime = MPI_Wtime() - startTime;
        std::cout << "Sequential sort:" << sequentialtime << std::endl;
        std::cout << "Parallel sort:" << paralleltime << std::endl;
        ASSERT_EQ(resS, resP);
    }
}

TEST(TestQuickSort, randomVector100100) {
    int size, rank;
    MPI_Comm_size(MPI_COMM_WORLD, &size);
    MPI_Comm_rank(MPI_COMM_WORLD, &rank);
    int vecsize = 100100;
    std::vector<int> vec = getRandomVector(vecsize);
    double startTime = MPI_Wtime();
    std::vector<int> resP = parallel_sorting(vec);
    double finishTime = MPI_Wtime();
    double paralleltime = finishTime - startTime;
    if (rank == 0) {
        std::vector<int> resS = vec;
        double startTime = MPI_Wtime();
        std::sort(resS.begin(), resS.end());
        double sequentialtime = MPI_Wtime() - startTime;
        std::cout << "Sequential sort:" << sequentialtime << std::endl;
        std::cout << "Parallel sort:" << paralleltime << std::endl;
        ASSERT_EQ(resS, resP);
    }
}

TEST(TestQuickSort, randomVector18523140) {
    int size, rank;
    MPI_Comm_size(MPI_COMM_WORLD, &size);
    MPI_Comm_rank(MPI_COMM_WORLD, &rank);
    int vecsize = 18523140;
    std::vector<int> vec = getRandomVector(vecsize);
    double startTime = MPI_Wtime();
    std::vector<int> resP = parallel_sorting(vec);
    double finishTime = MPI_Wtime();
    double paralleltime = finishTime - startTime;
    if (rank == 0) {
        std::vector<int> resS = vec;
        double startTime = MPI_Wtime();
        std::sort(resS.begin(), resS.end());
        double sequentialtime = MPI_Wtime() - startTime;
        std::cout << "Sequential sort:" << sequentialtime << std::endl;
        std::cout << "Parallel sort:" << paralleltime << std::endl;
        ASSERT_EQ(resS, resP);
    }
}


int main(int argc, char** argv) {
  ::testing::InitGoogleTest(&argc, argv);
  MPI_Init(&argc, &argv);

  ::testing::AddGlobalTestEnvironment(new GTestMPIListener::MPIEnvironment);
  ::testing::TestEventListeners& listeners =
    ::testing::UnitTest::GetInstance()->listeners();

  listeners.Release(listeners.default_result_printer());
  listeners.Release(listeners.default_xml_generator());

  listeners.Append(new GTestMPIListener::MPIMinimalistPrinter);
  return RUN_ALL_TESTS();
}


\end{lstlisting}

\end{document}