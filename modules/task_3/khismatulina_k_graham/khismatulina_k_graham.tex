\documentclass{report}

\usepackage[T2A]{fontenc}
\usepackage[utf8]{luainputenc}
\usepackage[english, russian]{babel}
\usepackage[pdftex]{hyperref}
\usepackage[14pt]{extsizes}
\usepackage{listings}
\usepackage{color}
\usepackage{geometry}
\usepackage{enumitem}
\usepackage{multirow}
\usepackage{graphicx}
\usepackage{indentfirst}

\geometry{a4paper,top=2cm,bottom=3cm,left=2cm,right=1.5cm}
\setlength{\parskip}{0.5cm}
\setlist{nolistsep, itemsep=0.3cm,parsep=0pt}

\lstset{language=C++,
basicstyle=\footnotesize,
keywordstyle=\color{blue}\ttfamily,
stringstyle=\color{red}\ttfamily,
commentstyle=\color{green}\ttfamily,
morecomment=[l][\color{magenta}]{\#},
tabsize=4,
breaklines=true,
breakatwhitespace=true,
title=\lstname,
}

\makeatletter
\renewcommand\@biblabel[1]{#1.\hfil}
\makeatother

\begin{document}

\begin{titlepage}

\begin{center}
Министерство науки и высшего образования Российской Федерации
\end{center}

\begin{center}
Федеральное государственное автономное образовательное учреждение высшего образования \\
Национальный исследовательский Нижегородский государственный университет им. Н.И. Лобачевского
\end{center}

\begin{center}
Институт информационных технологий, математики и механики
\end{center}

\vspace{4em}

\begin{center}
\textbf{\LargeОтчет по лабораторной работе} \\
\end{center}
\begin{center}
\textbf{\Large«Построение выпуклой оболочки. Проход Грэхэма.»} \\
\end{center}

\vspace{4em}

\newbox{\lbox}
\savebox{\lbox}{\hbox{text}}
\newlength{\maxl}
\setlength{\maxl}{\wd\lbox}
\hfill\parbox{7cm}{
\hspace*{5cm}\hspace*{-5cm}\textbf{Выполнил:} \\ студент группы 381808-2 \\ Хисматулина Карина\\
\\
\hspace*{5cm}\hspace*{-5cm}\textbf{Проверил:}\\ доцент кафедры МОСТ, \\ кандидат технических наук \\ Сысоев А.В.\\
}
\vspace{\fill}

\begin{center} Нижний Новгород \\ 2020 \end{center}

\end{titlepage}

\setcounter{page}{2}

\newpage

\section*{Введение}
\addcontentsline{toc}{section}{Введение}
Задача построения выпуклых оболочек, является одной из центральных задач вычислительной геометрии. Важность этой задачи происходит не только из-за огромного количества приложений (в распознавании образов, обработке изображений, базах данных, в задаче раскроя и компоновки материалов, математической статистике), но также и из-за полезности выпуклой оболочки как инструмента решения множества задач вычислительной геометрии. Очень широко алгоритмы построения выпуклой оболочки используются в геоинформатике и геоинформационных системах. Выпуклую оболочку объекта можно строить различными аналитическими способами, которые в основном являются сложными. Ниже предлагается один из способов построения выпуклой оболочки на языке C++ с использованием технологий MPI.

\newpage

\section*{Постановка задачи}
\addcontentsline{toc}{section}{Постановка задачи}
В данной лабораторной работе необходимо реализовать построение минимальной выпуклой оболочки по алгоритму Грэхэма. Программа должна быть написана с возможностью запуска на нескольких процессах.
Задача построения выпуклой оболочки заключается в том, что необходимо  получить некоторый вектор координат точек на плоскости и построить выпуклую область, которая включает все полученные точки и занимает минимальную возможную площадь.
Результат — упорядоченное множество вершин выпуклой оболочки.
Программа должна также уметь генерировать случайные множества точек и оценивать время выполнения.

\par Цель – получение навыков распараллеливания задач компьютерной графики средствами библиотеки MPI.
\par Критерии эффективности разработки:
•	результаты тестирования корректности работы алгоритма (сравнение результатов работы линейной и параллельной версии программы)
•	производительность параллельной версии программы выше, чем линейной(не достигнуто)
\par Требования:
•	Должна использоваться библиотека MPI
•	Схема распараллеливания не зависит от количества предоставленных ресурсов

\newpage

\section*{Метод решения и описание алгоритма}
\addcontentsline{toc}{section}{Описание алгоритма}
\par
Если получено не более двух точек, выпуклая оболочка является точкой или отрезком и может быть построена сразу по полученным точкам. Иначе для нахождения выпуклой оболочки используется алгоритм Грэхэма.

\par 
Алгоритм Грэхэма:
\par 
	Находим точку P0 нашего множества с самой маленькой у-координатой (если таких несколько, берем самую правую из них), добавляем в ответ.
\par 
	Сортируем все остальные точки по полярному углу относительно P0 .
\par 
	Добавляем в ответ P1 - самую первую из отсортированных точек.
\par 
	Берем следующую по счету точку t. Пока t и две последних точки в текущей оболочке Pi и P(i-1) образуют неправый поворот (вектора Pi t и P(i-1) Pi), удаляем из оболочки Pi.
	Добавляем в оболочку t.
\par 
\par 
	Делаем п.5, пока не закончатся точки.
\par 
В программе выполняется вычисление косинусов углов, а не самих углов.
Косинус вычисляется как частное скалярного произведения векторов (значения вектора — разности координат точек) и произведения длин тех же векторов.
\newpage




% Схема распараллеливания
\section*{Схема распараллеливания}
\addcontentsline{toc}{section}{Схема распараллеливания}
В алгоритме Грэхэма для поиска точки выпуклой оболочки необходимо, чтобы были известны две предыдущие точки оболочки, потому между процессами постоянно должен осуществляться обмен данными.
\par 
\par 
В начале программы равномерно распределим векторы координат точек между всеми процессами.
\par 
Чтобы найти нижнюю левую точку p1, найдем на каждом процессе крайнюю левую точку из множества точек, которое известно процессу, соберем результаты на нулевом процессе и найдем в полученном множестве точек крайнюю левую точку снова, эта точка будет результатом поиска.
\par 
Чтобы, зная точки pi и pi+1 , найти точку pi+2 с минимальным углом между прямыми  pi pi+1 и  pi+1 pi+2 , найдем на каждом процессе такую точку из известного процессу множества точек, соберем результаты на нулевом процессе и повторим в полученном множестве поиск необходимой точки.
\par 
Когда мы завершаем построение выпуклой оболочки, необходимо оповестить все процессы об этом.

\section*{Описание программной реализации}
\addcontentsline{toc}{section}{Описание программной реализации}
\par 
Программа реализует линейную и параллельную версии алгоритма построение минимальной выпуклой оболочки.
\par 
Для начала работы с программой необходимо установить библиотеки MPI.
\par 
Запуск программы возможен только из интерфейса командной строки посредством введения в неё команды: [pathToMPI]/mpiexec.exe [–np [нужное кол-во процессов]] [путь к скомпилированному файлу].
Код программы можно просмотреть в разделе «Приложение».
\newpage

\section*{Подтверждение корректности}
\addcontentsline{toc}{section}{Подтверждение корректности}
Для подтверждения корректности в программе предусмотрена автоматическая проверка полного соответствия результата работы линейной версии алгоритма результату работы параллельной версии алгоритма.
\newpage

\section*{Результаты экспериментов}
\addcontentsline{toc}{section}{Результаты экспериментов}
Эксперименты проводились на 4-х ядерном компьютере с операционной системой – Windows 10, который имеет 8 логических процессоров. Результаты приведены в таблице:

\begin{table}[!h]
\caption{Результаты вычислительных экспериментов}
\centering
\begin{tabular}{llll}
Элементы на входе & Паралл. время & Послед. время \\
10(4 процесса) & 3.70999e-05 & 0.000156\\
100(4 процесса) & 0.000811 & 0.001458\\
1000(4 процесса) & 0.058932 & 0.105002\\
10(8 процессов) & 4.05e-05 & 0.000955\\
100(8 процессов) & 0.000668 & 0.001676\\
1000(8 процессов) & 0.05913	 & 0.09285\\
\end{tabular}
\end{table}
\par

Из-за несовершенства параллельной части ускорение не достигается, однако при увеличении числа поступающих на вход элементов заметно уменьшается разница в скорости выполнения алгоритмов(до двух раз при 1000 элементах).
\newpage


\section*{Заключение}
\addcontentsline{toc}{section}{Заключение}
Мной был изучен и реализован алгоритм Грэхэма для поиска выпуклой оболочки.
\par 
Реализация программы эффективна только при больших объемах данных, потому что при работе программы на нескольких процессах возникают накладные расходы.
\newpage


\begin{thebibliography}{1}
\addcontentsline{toc}{section}{Список литературы}
\bibitem{Gergel}
Гергель В. П., Стронгин Р. Г. Основы параллельных вычислений для многопроцессорных вычислительных систем. – 2003.
\bibitem{parallel} Parallel: Лаборатория Параллельных информационных технологий НИВЦ МГУ [Электронный ресурс] // URL: \url {https://parallel.ru/vvv/mpi.html} (дата обращения: 28.11.2020)
\bibitem{Algolist} Algolist: Алгоритмы, методы, исходники [Электронный ресурс] // URL: \url {http://algolist.manual.ru/sort/radix_sort.php} (дата обращения: 28.11.2020)
\end{thebibliography}
\newpage

% Приложение
\section*{Приложение}
\addcontentsline{toc}{section}{Приложение}
\begin{lstlisting}
// Copyright 2020 Khismatulina Karina
#include <algorithm>
#include <vector>

#ifndef MODULES_TASK_3_KHISMATULINA_K_GRAHAM_GRAHAM_H_
#define MODULES_TASK_3_KHISMATULINA_K_GRAHAM_GRAHAM_H_

class Point {
 public:
    double x;
    double y;
    Point() { x = 0; y = 0; }
    Point(double _x, double _y) : x(_x), y(_y) {}
    Point& operator=(const Point& A) {
        if (this != &A) {
            x = A.x;
            y = A.y;
        }
        return *this;
    }
    bool operator==(const Point& p) {
        return ((fabs(x - p.x) < 1e-7)) && (fabs(y - p.y) < 1e-7);
    }
    bool operator!=(const Point& p) {
        return ((fabs(x - p.x) > 1e-7)) && (fabs(y - p.y) < 1e-7);
    }
    Point min(const Point& A, const Point& B) {
        if (A.y < B.y) {
            return A;
        } else {
            if ((fabs(A.y - B.y) < 1e-7) && (fabs(A.x - B.x) < 1e-7))
                return A;
        }
        return B;
    }
    ~Point() { x = 0; y = 0; }
};

class Vector {
 public:
    double X;
    double Y;
    Vector() { X = 0; Y = 0; }
    Vector(Point A, Point B) { X = B.x - A.x; Y = B.y - A.y; }
    Vector(double X_, double Y_) { X = X_; Y = Y_; }
    ~Vector() { X = Y = 0; }
};


double modul(const Vector& Vec);

double skalyar(const Vector& Vec1, const Vector& Vec2);

double cos(const Vector& A, const Vector& B);

bool cw(double x1, double y1, double x2, double y2, double x3, double y3);

std::vector<Point> GrahamPar(const std::vector<Point>& Points);

std::vector<Point> GrahamSeq(const std::vector<Point>& Points);

Point SearchMinPoint(const std::vector<Point>& Points);

std::vector<Point> Sort(const std::vector<Point>& Points);

std::vector<Point> GenPoints(int n);

#endif  // MODULES_TASK_3_KHISMATULINA_K_GRAHAM_GRAHAM_H_

\end{lstlisting}

\begin{lstlisting}
// Copyright 2020 Khismatulina Karina
#include <mpi.h>
#include <vector>
#include <cmath>
#include <ctime>
#include <random>
#include "../../../modules/task_3/khismatulina_k_graham/graham.h"
//#include <opencv2/opencv.hpp>

double modul(const Vector& Vec) {
    return (sqrt(Vec.X * Vec.X + Vec.Y * Vec.Y));
}

double skalyar(const Vector& Vec1, const Vector& Vec2) {
    return (Vec1.X * Vec2.X + Vec1.Y * Vec2.Y);
}

double cos(const Vector& A, const Vector& B) {
    return (skalyar(A, B) / (modul(A) * modul(B)));
}

Point SearchMinPoint(const std::vector<Point>& Points) {
    Point min_point = Points[0];
    size_t i = 0;
    while (i < Points.size()) {
        min_point = min_point.min(min_point, Points[i]);
        i += 1;
    }
    return min_point;
}

std::vector<Point> GenPoints(int n) {
    if (n < 1)
        throw "slishkom malo!!!";
    std::vector<Point> p(n);
    std::mt19937 gen;
    for (int i = 0; i < n; i++) {
        p[i].x = gen();
        p[i].y = gen();
    }
    return p;
}

std::vector<Point> Sort(const std::vector<Point>& Points) {
    std::vector<Point> P = Points;
    Point min_point = SearchMinPoint(P);
    std::vector<double> Cos(P.size());
    Vector A(1, 0);
    for (size_t i = 0; i < P.size(); i++) {
        Vector B(min_point, Points[i]);
        if (min_point == Points[i]) {
            Cos[i] = 2;
        } else {
            Cos[i] = cos(A, B);
        }
    }
    for (size_t i = 0; i < P.size() - 1; ++i) {
        for (size_t j = 0; j < P.size() - i - 1; ++j) {
            if (Cos[j + 1] > Cos[j]) {
                Point tmp = P[j + 1];
                P[j + 1] = P[j];
                P[j] = tmp;
                double tmp_cos = Cos[j + 1];
                Cos[j + 1] = Cos[j];
                Cos[j] = tmp_cos;
            }
        }
    }
    for (size_t i = 0; i < P.size() - 1; ++i) {
        if (fabs(Cos[i + 1] - Cos[i]) < 0.000001) {
            Vector Vec1(min_point, P[i]);
            Vector Vec2(min_point, P[i + 1]);
            if (modul(Vec1) > modul(Vec2)) {
                Point temp = P[i + 1];
                P[i + 1] = P[i];
                P[i] = temp;
            }
        }
    }
    return P;
}

bool cw(double x1, double y1, double x2, double y2, double x3, double y3) {
    return (x2 - x1) * (y3 - y1) - (y2 - y1) * (x3 - x1) < 0;
}

std::vector<Point> GrahamSeq(const std::vector<Point>& P) {
    std::vector<Point> Points = Sort(P);
    if (Points.size() < 2) {
        return Points;
    }
    std::vector<double> Res;
    Res.push_back(Points[0].x);
    Res.push_back(Points[0].y);
    Res.push_back(Points[1].x);
    Res.push_back(Points[1].y);
    if (Points.size() > 2) {
        for (size_t i = 2; i < Points.size(); ++i) {
            while (cw(Res[Res.size() - 4], Res[Res.size() - 3], Res[Res.size() - 2],
                Res[Res.size() - 1], Points[i].x, Points[i].y)) {
                Res.pop_back();
                Res.pop_back();
            }
            Res.push_back(Points[i].x);
            Res.push_back(Points[i].y);
        }
    }
    std::vector<Point> resPoint(Res.size() / 2);
    for (size_t i = 0; i < Res.size(); i += 2) {
        resPoint[i / 2].x = Res[i];
        resPoint[i / 2].y = Res[i + 1];
    }
    return resPoint;
}

std::vector<Point> GrahamPar(const std::vector<Point>& Points) {
    std::vector<double> Res;
    std::vector<Point> P = Sort(Points);
    int points_amount = P.size();
    MPI_Status status;
    int rank, size;
    MPI_Comm_size(MPI_COMM_WORLD, &size);
    MPI_Comm_rank(MPI_COMM_WORLD, &rank);
    int delta = points_amount / size;
    if (points_amount >= size) {
        std::vector<double> Vec;
        for (size_t i = 0; i < P.size(); ++i) {
            Vec.push_back(P[i].x);
            Vec.push_back(P[i].y);
        }
        std::vector<double> local_vec(2 * delta);
        MPI_Scatter(&Vec[0], 2 * delta, MPI_DOUBLE, &local_vec[0], 2 * delta, MPI_DOUBLE, 0, MPI_COMM_WORLD);
        if (rank == (size - 1)) {
            size_t i = size * delta;
            while (i < P.size()) {
                local_vec.push_back(P[i].x);
                local_vec.push_back(P[i].y);
                i += 1;
            }
        }
        std::vector<Point> l_P(local_vec.size() / 2);
        for (size_t i = 0; i < local_vec.size(); i += 2) {
            l_P[i / 2].x = local_vec[i];
            l_P[i / 2].y = local_vec[i + 1];
        }
        std::vector<Point> local_res = GrahamSeq(l_P);
        local_vec.clear();
        for (size_t i = 0; i < local_res.size(); i++) {
            local_vec.push_back(local_res[i].x);
            local_vec.push_back(local_res[i].y);
        }
        int vec_n = local_vec.size();
        std::vector<int> vec_size(size);
        int N = 0;
        MPI_Gather(&vec_n, 1, MPI_INT, &vec_size[0], 1, MPI_INT, 0, MPI_COMM_WORLD);
        MPI_Reduce(&vec_n, &N, 1, MPI_INT, MPI_SUM, 0, MPI_COMM_WORLD);
        std::vector<double> global_vec(N);
        if (rank == 0) {
            std::vector<int> address(size);
            address[0] = 0;
            for (int i = 1; i < size; i++) {
                address[i] = vec_size[i - 1];
            }
            for (int i = 2; i < size; i++) {
                address[i] += address[i - 1];
            }
            for (int i = 0; i < vec_n; i++) {
                global_vec[i] = local_vec[i];
            }
            for (int i = 1; i < size; i++) {
                MPI_Recv(&global_vec[address[i]], vec_size[i], MPI_DOUBLE, i, MPI_ANY_TAG, MPI_COMM_WORLD, &status);
            }
        } else {
            MPI_Send(&local_vec[0], vec_n, MPI_DOUBLE, 0, 1, MPI_COMM_WORLD);
        }
        if (rank == 0) {
            std::vector<Point> global_res(global_vec.size() / 2);
            size_t i = 0;
            while (i < global_vec.size()) {
                global_res[i / 2].x = global_vec[i];
                global_res[i / 2].y = global_vec[i + 1];
                i += 2;
            }
            std::vector<Point> res_point = GrahamSeq(global_res);
            return res_point;
        }
    } else {
        std::vector<Point> res_point = GrahamSeq(P);
        return res_point;
    }
    return P;
}

\end{lstlisting}

\begin{lstlisting}
// Copyright 2020 Khismatulina Karina

#include <gtest-mpi-listener.hpp>
#include <gtest/gtest.h>
#include <math.h>
#include <stack>
#include <vector>
#include <iostream>
#include "../../../modules/task_3/khismatulina_k_graham/graham.h"
//#include <opencv2/opencv.hpp>

TEST(khismatulina_task_3, test_1_error) {
    int rank;
    MPI_Comm_rank(MPI_COMM_WORLD, &rank);
    if (rank == 0) {
        ASSERT_ANY_THROW(GenPoints(-1));
    }
}

TEST(khismatulina_task_3, test_2_par_10) {
    int rank;
    MPI_Comm_rank(MPI_COMM_WORLD, &rank);
    std::vector<Point> Points{ {0, 0}, {1, 5}, {3, 6}, {-1, -4}, {2, 2}, {1, 2.5}, {1, -1}, {0, 1}, {0, 3}, {4, -1} };
    std::vector<Point> P = Sort(Points);
    std::vector<Point> res = GrahamPar(P);
    std::vector<Point> check{ {-1, 0}, {4, 5}, {3, 6}, {1, -4}, {0, 2} };
    bool a = 1;
    if (rank == 0) {
        for (size_t i = 0; i < res.size(); ++i)
            if (res[i] != check[i])
                a = 0;
        EXPECT_EQ(a, 1);
    }
}

TEST(khismatulina_task_3, test_3_par_100) {
    int rank;
    MPI_Comm_rank(MPI_COMM_WORLD, &rank);
    std::vector<Point> P = GenPoints(100);
    std::vector<Point> res = GrahamPar(P);
    std::vector<Point> check = GrahamSeq(P);
    bool a = 1;
    if (rank == 0) {
        for (size_t i = 0; i < res.size(); ++i)
            if (res[i] != check[i])
                a = 0;
        EXPECT_EQ(a, 1);
    }
}

TEST(khismatulina_task_3, test_4_par_200) {
    int rank;
    MPI_Comm_rank(MPI_COMM_WORLD, &rank);
    std::vector<Point> P = GenPoints(200);
    std::vector<Point> res = GrahamPar(P);
    std::vector<Point> check = GrahamSeq(P);
    bool a = 1;
    if (rank == 0) {
        for (size_t i = 0; i < res.size(); ++i)
            if (res[i] != check[i])
                a = 0;
        EXPECT_EQ(a, 1);
    }
}

TEST(khismatulina_task_3, test_5_par_500) {
    int rank;
    MPI_Comm_rank(MPI_COMM_WORLD, &rank);
    std::vector<Point> P = GenPoints(500);
    std::vector<Point> res = GrahamPar(P);
    std::vector<Point> check = GrahamSeq(P);
    bool a = 1;
    if (rank == 0) {
        for (size_t i = 0; i < res.size(); ++i)
            if (res[i] != check[i])
                a = 0;
        EXPECT_EQ(a, 1);
    }
}

TEST(khismatulina_task_3, test_for_time) {
    int rank;
    MPI_Comm_rank(MPI_COMM_WORLD, &rank);
    std::vector<Point> P = GenPoints(1000);
    double startTime, endTime;
    startTime = MPI_Wtime();
    std::vector<Point> res = GrahamPar(P);
    if (rank == 0) {
        endTime = MPI_Wtime();
        std::cout << "par: " << endTime - startTime << std::endl;
    }
    if (rank == 0) {
        bool a = 1;
        startTime = MPI_Wtime();
        std::vector<Point> check = GrahamSeq(P);
        endTime = MPI_Wtime();
        std::cout << "seq: " << endTime - startTime << std::endl;
        EXPECT_EQ(a, 1);
    }
}


int main(int argc, char** argv) {
    ::testing::InitGoogleTest(&argc, argv);
    MPI_Init(&argc, &argv);

    ::testing::AddGlobalTestEnvironment(new GTestMPIListener::MPIEnvironment);
    ::testing::TestEventListeners& listeners =
        ::testing::UnitTest::GetInstance()->listeners();

    listeners.Release(listeners.default_result_printer());
    listeners.Release(listeners.default_xml_generator());

    listeners.Append(new GTestMPIListener::MPIMinimalistPrinter);

    return RUN_ALL_TESTS();
}

\end{lstlisting}
\end{document}

