\documentclass{report}

\usepackage[T2A]{fontenc}
\usepackage[utf8]{luainputenc}
\usepackage[english, russian]{babel}
\usepackage[pdftex]{hyperref}
\usepackage[14pt]{extsizes}
\usepackage{listings}
\usepackage{color}
\usepackage{geometry}
\usepackage{enumitem}
\usepackage{multirow}
\usepackage{graphicx}
\usepackage{indentfirst}
\usepackage{amsmath}
\usepackage{esint}

\geometry{a4paper,top=2cm,bottom=3cm,left=2cm,right=1.5cm}
\setlength{\parskip}{0.5cm}
\setlist{nolistsep, itemsep=0.3cm,parsep=0pt}

\lstset{language=C++,
		basicstyle=\footnotesize,
		keywordstyle=\color{blue}\ttfamily,
		stringstyle=\color{red}\ttfamily,
		commentstyle=\color{green}\ttfamily,
		morecomment=[l][\color{magenta}]{\#}, 
		tabsize=4,
		breaklines=true,
  		breakatwhitespace=true,
  		title=\lstname,       
}

\makeatletter
\renewcommand\@biblabel[1]{#1.\hfil}
\makeatother

\begin{document}

\begin{titlepage}

\begin{center}
Министерство науки и высшего образования Российской Федерации
\end{center}

\begin{center}
Федеральное государственное автономное образовательное учреждение высшего образования \\
Национальный исследовательский Нижегородский государственный университет им. Н.И. Лобачевского
\end{center}

\begin{center}
Институт информационных технологий, математики и механики
\end{center}

\vspace{4em}

\begin{center}
\textbf{\LargeОтчет по лабораторной работе} \\
\end{center}
\begin{center}
\textbf{\Large«Вычисление многомерных интегралов с использованием многошаговой схемы (метод прямоугольников)»} \\
\end{center}

\vspace{4em}

\newbox{\lbox}
\savebox{\lbox}{\hbox{text}}
\newlength{\maxl}
\setlength{\maxl}{\wd\lbox}
\hfill\parbox{7cm}{
\hspace*{5cm}\hspace*{-5cm}\textbf{Выполнил:} \\ студент группы 381806-1 \\ Бакаева М.И.\\
\\
\hspace*{5cm}\hspace*{-5cm}\textbf{Проверил:}\\ доцент кафедры МОСТ, \\ Нестеров А.Ю.\\
}
\vspace{\fill}

\begin{center} Нижний Новгород \\ 2020 \end{center}

\end{titlepage}

\setcounter{page}{2}

% Содержание
\tableofcontents
\newpage

% Введение
\section*{Введение}
\addcontentsline{toc}{section}{Введение}
Задача вычисления определенного интеграла I для некоторой заданной на отрезке [a, b] функции f(x) является классической задачей математического анализа. Известно, что для функций, имеющих на [a, b] конечное число точек разрыва первого рода, такое значение существует, единственно и может быть формально получено по определению как\\
\begin{gather*}
\idotsint_V\ f(x_1,\dots,x_k) \,dx_1 \dots dx_k=  h_i\[ \sum_{n=1}^{\infty} f(p_i)\]
\end{gather*}
\par В данном лабораторной работе будет рассмотрен метод прямоугольников. В ходе работы будет представлен подход к распараллеливанию задачи.
\newpage

% Постановка задачи
\section*{Постановка задачи}
\addcontentsline{toc}{section}{Постановка задачи}
В лабораторной работе необходимо реализовать последовательную и параллельную реализации метода прямоугольников для вычисления многомерных интегралов, проверить корректность работы алгоритмов, провести тесты для оценки времени. По полученным результатам сделать выводы.
\par Для реализации параллельной версии будут использоваться средства MPI. Для проверки корректности работы алгоритмов будут использзоваться Google C++ Testing Framework.
\newpage

% Метод решения
\section*{Метод решения}
\addcontentsline{toc}{section}{Метод решения}
Алгоритм вычисления интегралов методом прямоугольников:
\begin{itemize}
\item Весь участок [a, b] делим на n равных частей с шагом h = (b - a)/ n.
\item Определяем значение y_i подынтегральной функции f(x) в каждой части деления, т.е y_i = f(x_i), i\in[0, n].
\item
\item Для каждой части деления определяем площадь  частичного прямоугольника.
\item Суммируем эти площади. Приближенное значение интеграла I равно сумме площадей частичных прямоугольников.
\end{itemize}
\newpage

% Схема распараллеливания
\section*{Схема распараллеливания}
\addcontentsline{toc}{section}{Схема распараллеливания базового алгоритма}
Один из наиболее простых подходов - геометрическая декомпозиция данных. Данный подход предлагает разделение данных на части и применение к ним одного и того же алгоритма. В численном интегрировании мы имеем дело с прямоугольной сеткой, в каждом узле которой вычисляется функция и умножается на квадрат шага. Вычисление функции в одном узле сетки не зависит от соседних узлов, таким образом, поделив сетку между потоками, можно получить параллельную версию.
\\ Сетку интегрирования будем разбивать по строкам. То есть каждый процессор будет вычислять равное количество сумм наших интегралов. После все результаты будут суммироваться и умножаться на шаги разбиения отрезков интегрирования.
\newpage

% Описание программной реализации
\section*{Описание программной реализации}
\addcontentsline{toc}{section}{Описание программной реализации}
Алгоритм последовательного вычисления многомерных интегралов вызывается функцией:
\begin{lstlisting}
double getSequentialIntegrals(const int n, vector<pair<double, double> > a_b, double (*F)(vector<double>));
\end{lstlisting}
\par В качестве входных параметров принимает - шаг разбиения, вектор отрезков интегрирования, функцию для вычисления. Возвращает результат вычисления интеграла.
\par Алгоритм параллельного вычисления многомерных интегралов вызывается функцией:
\begin{lstlisting}
double getParallelIntegrals(const int n, vector<pair<double, double> > a_b, double (*F)(vector<double>));
\end{lstlisting}
\par В качестве входных параметров принимает - шаг разбиения, вектор отрезков интегрирования, функцию для вычисления. Возвращает результат вычисления интеграла.

\newpage

% Подтверждение корректности
\section*{Подтверждение корректности}
\addcontentsline{toc}{section}{Подтверждение корректности}
Для подтверждения корректности в программе существует набор тестов, разработанных с использованием Google C++ Testing Framework.
\par 
Представленные тесты проверяют корректность вычислений, а именно сравниваются результаты линейной и параллельной реализаций. Также проверяется эффективность с помощью измерения времени, полученные результаты выводятся в консоль.
\par Успешное прохождение всех тестов доказывает корректность работы программы.
\newpage

% Результаты экспериментов
\section*{Результаты экспериментов}
\addcontentsline{toc}{section}{Результаты экспериментов}
Вычислительные эксперименты для оценки эффективности параллельного вычисления многомерных интегралов методом прямоугольников проводились на оборудовании со следующей аппаратной конфигурацией:

\begin{itemize}
\item Процессор: Intel Core i3-7130U, 2700 MHz, ядер: 2;
\item Оперативная память: 7880 МБ (DDR3), 1600 MHz;
\item ОС: Microsoft Windows 10 Home, версия 1903 сборка 18362.535.
\end{itemize}

\par Для проведения экспериментов производилось вычисление тройного интеграла с шагом разбиения 1500.
\par Результаты экспериментов представлены в Таблице 1.

\begin{table}[!h]
\caption{Результаты вычислительных экспериментов}
\centering
\begin{tabular}{lllll}
Процессы & Последовательно & Параллельно & Ускорение  \\
1        & 419.917          & 366,117     & 0.87       \\
2        & 364.567         & 221.947     & 1.64       \\
3        & 367.293         & 182.732     & 2.01       \\
4        & 384,111         & 163,564     & 2.34       \\
5        & 368.306         & 157.9     & 2.33       \\
6        & 375.097         & 161.539     & 2.32
\end{tabular}
\end{table}

\par По результатам экспериментов видно, что параллельная реализация работает быстрее линейной. Но при шести процесах появляются регрессии. Происходит это по причине переключения между процессами и коллективного обмена данными. Приходим к выводу, что оптимальным числом процессов будет 4.
\newpage

% Заключение
\section*{Заключение}
\addcontentsline{toc}{section}{Заключение}
В результате лабораторной работы были разработаны последовательная и параллельная реализации вычисления многомерных интегралов методом прямоугольников.
\par Основная задача лабораторной работы была достигнута. Параллельная версия работает быстрее, что было доказано в ходе экспериментов.
\par Тесты созданные с помощью Google C++ Testing Framework подтвердили корректность работы программы и помогли в проведении экспериментов.
\newpage

% Список литературы
\begin{thebibliography}{1}
\addcontentsline{toc}{section}{Список литературы}
\bibitem{Sidnev} Козинов Е.А., Мееров И.Б., Сысоев А.В. «Лабораторная работа №2. Вычисление определенного интеграла». Нижний Новгород, 2010, 38 с. 
\bibitem{parallel} Clever Students:[Электронный ресурс] // URL: \url {http://www.cleverstudents.ru/integral/method_of_rectangles.html}
\end{thebibliography}
\newpage

% Приложение
\section*{Приложение}
\addcontentsline{toc}{section}{Приложение}
Листинг кода лабораторной работы.
\begin{lstlisting}
// integrals_rectangles_method.h
// Copyright 2020 Bakaeva Maria
#ifndef MODULES_TASK_3_BAKAEVA_M_INTEGRALS_RECTANGLES_METHOD_INTEGRALS_RECTANGLES_METHOD_H_
#define MODULES_TASK_3_BAKAEVA_M_INTEGRALS_RECTANGLES_METHOD_INTEGRALS_RECTANGLES_METHOD_H_
#include <vector>
#include <algorithm>
#include <utility>

using std::vector;
using std::pair;

double getSequentialIntegrals(const int n, vector<pair<double, double> > a_b, double (*F)(vector<double>));
double getParallelIntegrals(const int n, vector<pair<double, double> > a_b, double (*F)(vector<double>));
#endif  // MODULES_TASK_3_BAKAEVA_M_INTEGRALS_RECTANGLES_METHOD_INTEGRALS_RECTANGLES_METHOD_H_
\end{lstlisting}
\begin{lstlisting}
// integrals_rectangles_method.cpp
// Copyright 2020 Bakaeva Maria
#include "../../../modules/task_3/bakaeva_m_integrals_rectangles_method/integrals_rectangles_method.h"
#include <mpi.h>
#include <stddef.h>
#include <math.h>
#include <ctime>
#include <random>
#include <vector>
#include <algorithm>
#include <iostream>
#include <utility>

using std::vector;
using std::pair;


double getSequentialIntegrals(const int n, vector<pair<double, double> > a_b, double (*F)(vector<double>)) {
    // a_b - массив пределов интегрирования
    // n - количество отрезков интегрирования
    // h = (b - a) / n - шаг разбиения отрезка [a, b]
    // Integral = h * Summ(f(xi))
    // countIntegrals - количество интегралов

    int countIntegrals = static_cast<int>(a_b.size());
    size_t size = 1;
    vector<double> h(countIntegrals);
    for (int i = 0; i < countIntegrals; i++) {
        h[i] = (a_b[i].second - a_b[i].first) / n;
        size = size * n;
    }

    double result = 0.0;
    vector<double> forCalculateF(countIntegrals);
    for (size_t j = 0; j < size; j++) {
    for (int i = 0; i < countIntegrals; i++) {
        forCalculateF[i] = a_b[i].first + (j % n) * h[i] + h[i] * 0.5;
    }
    result += F(forCalculateF);
    }

    for (int i = 0; i < countIntegrals; i++) {
        result *= h[i];
    }

    return result;
}

double getParallelIntegrals(const int n, vector<pair<double, double> > a_b, double (*F)(vector<double>)) {
    int size, rank;

    //Число задействованных процессов
    MPI_Comm_size(MPI_COMM_WORLD, &size);
    //Получение номера текущего процесса в рамках коммуникатора
    MPI_Comm_rank(MPI_COMM_WORLD, &rank);

    int countIntegrals = static_cast<int>(a_b.size());
    vector<double> h(n);
    vector<pair<double, double>> ab(countIntegrals);
    size_t countElements;  // Количество всех одночленов

    // Находим шаги разбиения для каждого отрезка [a,b]
    if (rank == 0) {
        countElements = 1;
        for (int i = 0; i < countIntegrals; i++) {
            h[i] = (a_b[i].second - a_b[i].first) / n;
            ab[i] = a_b[i];
        }
        int j = 0;
        while (j != countIntegrals - 1) {
            countElements *= n;
            j++;
        }
    }

    // Рассылаем данные всем процессам
    int length = n;  // Количество отрезков интегрирования
    MPI_Bcast(&countElements, 1, MPI_UNSIGNED, 0, MPI_COMM_WORLD);
    MPI_Bcast(&length, countIntegrals, MPI_INT, 0, MPI_COMM_WORLD);
    MPI_Bcast(&h[0], countIntegrals, MPI_DOUBLE, 0, MPI_COMM_WORLD);
    MPI_Bcast(&ab[0], 2 * countIntegrals, MPI_DOUBLE, 0, MPI_COMM_WORLD);

    size_t delta = static_cast<size_t>(countElements / size);
    int remainder = static_cast<int>(countElements % size);
    if (rank < remainder) {
        delta += 1;
    }

    int tmp = 0;
    if (rank < remainder) {
        tmp = (rank * delta);
    } else {
        tmp = (remainder + rank * delta);
    }

    // Каждый процесс вычисляет свое количество sum(f(x..))
    vector<vector<double>> forCalculateF(delta);
    for (size_t i = 0; i < delta; i++) {
        int number = static_cast<int>(tmp + i);
        for (int j = 0; j < countIntegrals - 1; j++) {
            forCalculateF[i].push_back(ab[j].first + h[j] * (number % n) + h[j] * 0.5);
        }
    }

    double result = 0.0;
    for (size_t i = 0; i < delta; i++) {
        for (int j = 0; j < n; j++) {
            forCalculateF[i].push_back(ab[countIntegrals - 1].first + (j + 0.5) * h[countIntegrals - 1]);
            result += F(forCalculateF[i]);
            forCalculateF[i].pop_back();
        }
    }

    // Суммирование всех полученых результатов
    double Integral = 0.0;
    MPI_Reduce(&result, &Integral, 1, MPI_DOUBLE, MPI_SUM, 0, MPI_COMM_WORLD);
    // Умножение на h по формуле
    for (int i = 0; i < countIntegrals; i++) {
        Integral *= h[i];
    }

    return Integral;
}

\end{lstlisting}
\begin{lstlisting}
// main.cpp
// Copyright 2020 Bakaeva Maria
#include <gtest-mpi-listener.hpp>
#include <gtest/gtest.h>
#include <math.h>
#include <utility>
#include <vector>
#include <ctime>
#include <iostream>
#include "./integrals_rectangles_method.h"


    double f1(vector<double> vec) {
        int x = vec[0];
        int y = vec[1];
        return x + y * y;
    }

    double f2(vector<double> vec) {
        int x = vec[0];
        int y = vec[1];
        return x + y;
    }

    double f3(vector<double> vec) {
        int x = vec[0];
        int y = vec[1];
        return sin(3 * x) + cos(y);
    }

    double f4(vector<double> vec) {
        int x = vec[0];
        int y = vec[1];
        int z = vec[2];
        return 8 * y * y * z * exp(2 * x * y * z);
    }

    double f5(vector<double> vec) {
        int x = vec[0];
        int y = vec[1];
        int z = vec[2];
        return x * x * z * sin(x * y * z);
    }

TEST(IntegralsRectanglesMethod, time_test) {
    int rank;
    MPI_Comm_rank(MPI_COMM_WORLD, &rank);

    double t1, t2;

    vector<pair<double, double>> a_b(3);
    a_b = {{0, 2}, {0, 1}, {0, 3.14}};
    int n = 1500;
    double resLinear;

    if (rank == 0) {
        t1 = MPI_Wtime();
        resLinear = getSequentialIntegrals(n, a_b, f5);
        t2 = MPI_Wtime();
        std::cout << "Linear time: " << t2 - t1 << std::endl;
    }

    t1 = MPI_Wtime();
    double resParallel = getParallelIntegrals(n, a_b, f5);
    t2 = MPI_Wtime();

    if (rank == 0)
    std::cout << "Parallel time: " <<t2 - t1 << std::endl;

    if (rank == 0) {
        ASSERT_NEAR(resLinear, resParallel, 0.001);
    }
}

TEST(IntegralsRectanglesMethod, testCalculate_f1) {
    int rank;
    MPI_Comm_rank(MPI_COMM_WORLD, &rank);

    vector<pair<double, double>> a_b(2);
    a_b = {{1, 5}, {0, 3}};

    double resParallel = getParallelIntegrals(100, a_b, f1);

    if (rank == 0) {
        double resLinear = getSequentialIntegrals(100, a_b, f1);
            ASSERT_NEAR(resLinear, resParallel, 0.001);
    }
}

TEST(IntegralsRectanglesMethod, testCalculate_f2) {
    int rank;
    MPI_Comm_rank(MPI_COMM_WORLD, &rank);

    vector<pair<double, double>> a_b(2);
    a_b = {{1, 5}, {0, 3}};

    double resParallel = getParallelIntegrals(100, a_b, f2);

    if (rank == 0) {
        double resLinear = getSequentialIntegrals(100, a_b, f2);
            ASSERT_NEAR(resLinear, resParallel, 0.001);
    }
}

TEST(IntegralsRectanglesMethod, testCalculate_f3) {
    int rank;
    MPI_Comm_rank(MPI_COMM_WORLD, &rank);

    vector<pair<double, double>> a_b(2);
    a_b = {{1, 5}, {0, 3}};

    double resParallel = getParallelIntegrals(100, a_b, f3);

    if (rank == 0) {
        double resLinear = getSequentialIntegrals(100, a_b, f3);
            ASSERT_NEAR(resLinear, resParallel, 0.001);
    }
}

TEST(IntegralsRectanglesMethod, testCalculate_f4) {
    int rank;
    MPI_Comm_rank(MPI_COMM_WORLD, &rank);

    vector<pair<double, double>> a_b(3);
    a_b = {{-1, 0}, {0, 2}, {0, 1}};

    double resParallel = getParallelIntegrals(100, a_b, f4);

    if (rank == 0) {
        double resLinear = getSequentialIntegrals(100, a_b, f4);
            ASSERT_NEAR(resLinear, resParallel, 0.001);
    }
}

TEST(IntegralsRectanglesMethod, testCalculate_f5) {
    int rank;
    MPI_Comm_rank(MPI_COMM_WORLD, &rank);

    vector<pair<double, double>> a_b(3);
    a_b = {{0, 2}, {0, 1}, {0, 3.14}};

    double resParallel = getParallelIntegrals(100, a_b, f5);

    if (rank == 0) {
        double resLinear = getSequentialIntegrals(100, a_b, f5);
            ASSERT_NEAR(resLinear, resParallel, 0.001);
    }
}

int main(int argc, char** argv) {
    ::testing::InitGoogleTest(&argc, argv);
    MPI_Init(&argc, &argv);

    ::testing::AddGlobalTestEnvironment(new GTestMPIListener::MPIEnvironment);
    ::testing::TestEventListeners& listeners =
        ::testing::UnitTest::GetInstance()->listeners();

    listeners.Release(listeners.default_result_printer());
    listeners.Release(listeners.default_xml_generator());

    listeners.Append(new GTestMPIListener::MPIMinimalistPrinter);
    return RUN_ALL_TESTS();
}

\end{lstlisting}

\end{document}