\documentclass{report}

\usepackage[T2A]{fontenc}
\usepackage[utf8]{luainputenc}
\usepackage[english, russian]{babel}
\usepackage[pdftex]{hyperref}
\usepackage[14pt]{extsizes}
\usepackage{listings}
\usepackage{color}
\usepackage{geometry}
\usepackage{enumitem}
\usepackage{multirow}
\usepackage{graphicx}
\usepackage{indentfirst}

\geometry{a4paper,top=2cm,bottom=3cm,left=2cm,right=1.5cm}
\setlength{\parskip}{0.5cm}
\setlist{nolistsep, itemsep=0.3cm,parsep=0pt}

\lstset{language=C++,
		basicstyle=\footnotesize,
		keywordstyle=\color{blue}\ttfamily,
		stringstyle=\color{red}\ttfamily,
		commentstyle=\color{green}\ttfamily,
		morecomment=[l][\color{magenta}]{\#}, 
		tabsize=4,
		breaklines=true,
  		breakatwhitespace=true,
  		title=\lstname,       
}

\makeatletter
\renewcommand\@biblabel[1]{#1.\hfil}
\makeatother

\begin{document}

\begin{titlepage}

\begin{center}
Министерство науки и высшего образования Российской Федерации
\end{center}

\begin{center}
Федеральное государственное автономное образовательное учреждение высшего образования \\
Национальный исследовательский Нижегородский государственный университет им. Н.И. Лобачевского
\end{center}

\begin{center}
Институт информационных технологий, математики и механики
\end{center}

\vspace{4em}

\begin{center}
\textbf{\LargeОтчет по лабораторной работе} \\
\end{center}
\begin{center}
\textbf{\Large«Поиск кратчайших путей из одной вершины (алгоритм Дейкстры).»} \\
\end{center}

\vspace{4em}

\newbox{\lbox}
\savebox{\lbox}{\hbox{text}}
\newlength{\maxl}
\setlength{\maxl}{\wd\lbox}
\hfill\parbox{7cm}{
\hspace*{5cm}\hspace*{-5cm}\textbf{Выполнил:} \\ студент группы 381806-1 \\ Рустаов А.З.\\
\\
\hspace*{5cm}\hspace*{-5cm}\textbf{Проверил:}\\ доцент кафедры МОСТ, \\ кандидат технических наук \\ Сысоев А. В.\\
}
\vspace{\fill}

\begin{center} Нижний Новгород \\ 2020 \end{center}

\end{titlepage}

\setcounter{page}{2}

% Содержание
\tableofcontents
\newpage

% Введение
\section*{Введение}
\addcontentsline{toc}{section}{Введение}
Задача о поиске кратчайших путей является одной из важнейших классических задач теории графов.
\par Одним из возможных алгоритмов решения задачи поиска кратчайшего пути на графе является алгоритм Дейкстры.
\par Алгоритм Дейкстры имеет сложность порядка  $O(n^2)$, следовательно, его сложность быстро растёт с увеличением числа вершин. Поэтому возникает необходимость распараллеливания вычислений. 
\par В данной лабораторной работе будет реализован алгоритм Дейкстры.
\newpage

% Постановка задачи
\section*{Постановка задачи}
\addcontentsline{toc}{section}{Постановка задачи}
В рамках лабораторной работы необходимо реализовать последовательную и параллельную реализации алгоритма Дейкстры, проверить корректность работы алгоритмов, провести эксперименты для оценки эффективности параллелизации. По полученным результатам сделать выводы.
\par Для реализации параллельной версии необходимо использовать библиотеку межпроцессорного взаимодействия MPI. Для проверки корректности работы алгоритмов используется Google Testing Framework.
\newpage

% Метод решения
\section*{Метод решения}
\addcontentsline{toc}{section}{Метод решения}
Даны граф {\itshape A}, представленный в виде матрицы смежности и вершина-источник, для которой и будет строиться дерево кратчайших путей.
\par Первым шагом помечается вершина-источник, и расстояние до ней записывается равным нулю. Далее, ищется наименьшее ребро, соединяющее помеченную и непомеченную вершину. Если такое ребро есть, то такое ребро добавляется к дереву кратчайших путей, непомеченная вершина помечается, а расстояние до этой вершины складывается из веса добавленного ребра и расстояния до вершины, с которой соединяет это ребро. Если ребра не нашлось, то, либо в графе не осталось непомеченных вершин, либо не существует ребер, соединяющих помеченную вершину с непомеченной. В любом случае, алгоритм завершает свою работу.
\newpage

% Схема распараллеливания
\section*{Схема распараллеливания}
\addcontentsline{toc}{section}{Схема распараллеливания}
Для распараллеливания алгоритма Дейкстры необходимо равномерно распределить вершины на процессы. При использовании матрицы смежности, процессам передаются строки матрицы. Затем процессы, содержащий помеченные вершины, ищут ребро минимального веса, соединяющее помеченную вершину с непомеченной. Затем, веса этих ребер собираются в корневом процессе, который находит среди них минимум и рассылает всем процессам, в каком процессе этот минимум был найден. Далее, подходящий процесс передаёт, какие вершины соединяло это ребро. Это ребро добавляется в дерево кратчайших путей, помечается добавленная вершина и вычисляется расстояние до этой вершины. Затем, процесс повторяется.
\newpage

% Описание программной реализации
\section*{Описание программной реализации}
\addcontentsline{toc}{section}{Описание программной реализации}
В программе используются функции, представленные ниже:
\par Последовательный алгоритм Дейкстры реализован в следующей функции:
\begin{lstlisting}
Matrix SequentialDijkstraAlgorithm(Matrix graph, int verts, int source_vertex);
\end{lstlisting}
\par Входными данными функции является матрица смежности графа, количество вершин в графе и вершина-источник, для которой и будет построено дерево кратчайших путей. Выходными данными является построенное дерево кратчайших путей.
\par Параллельный алгоритм Дейкстры реализован в следующей функции:
\begin{lstlisting}
Matrix ParallelDijkstraAlgorithm(Matrix graph, int verts, int source_vertex);
\end{lstlisting}
\par Входными параметрами функции является матрица смежности графа, количество вершин в графе и вершина-источник, для которой и будет построено дерево кратчайших путей. Выходными данными является построенное дерево кратчайших путей.
\par При тестировании используются следующие функции:
\begin{itemize}
    \item Функция для генерации матрицы смежности
    \begin{lstlisting}
    Matrix RandomGraph(int verts, int edges, bool oriented)
    \end{lstlisting}
    \par Входными параметрами функции является чило вершин в графе, число ребер в графе и значение типа \verb|bool|, определяющее ориентированный ли граф или неориентированный. Выходыми данными является сгенерированная матрица смежности.
    \item Функция для генерации матрицы смежности разреженного графа
    \begin{lstlisting}
    Matrix RandomGraphSparce(int verts, int edges, bool oriented)
    \end{lstlisting}
    \par Входными параметрами функции является чило вершин в графе, число ребер в графе и значение типа \verb|bool|, определяющее ориентированный ли граф или неориентированный. Вызывается из функции RandomGraph(), если число ребер меньше, чем половина максимального числа ребер для данного графа. Выходыми данными является сгенерированная матрица смежности.
    \item Функция для генерации матрицы смежности плотного графа
    \begin{lstlisting}
    Matrix RandomGraphDence(int verts, int edges, bool oriented)
    \end{lstlisting}
    \par Входными параметрами функции является чило вершин в графе, число ребер в графе и значение типа \verb|bool|, определяющее ориентированный ли граф или неориентированный. Вызывается из функции RandomGraph(), если число ребер больше, чем половина максимального числа ребер для данного графа. Выходыми данными является сгенерированная матрица смежности.
\end{itemize}
\newpage

% Подтверждение корректности
\section*{Подтверждение корректности}
\addcontentsline{toc}{section}{Подтверждение корректности}
Для подтверждения корректности в программе представлен набор тестов, разработанных с Google Testing Framework.
\par Набор представляет из себя тесты, которые проверяют корректность корректность вычислений (сравнивается дерево кратчайших путей, полученное при параллельной реализации алгоритма, с дерево кратчайших путей, полученным с помощью последовательного алгоритма), а также эффективность (вычисление занимаемого последовательного и параллельного алгоритмов времени и сравнение полученных данных).
\par Успешное прохождение всех тестов подтверждает корректность работы написанной программы.
\newpage

% Результаты экспериментов
\section*{Результаты экспериментов}
\addcontentsline{toc}{section}{Результаты экспериментов}
Эксперименты для оценки эффективности проводились на ПК со следующими характеристиками:

\begin{itemize}
\item Процессор: AMD A10-9620P RADEON R5, 10 COMPUTE CORES 4C+6G, 2.50 GHz
\item Оперативная память: 8 ГБ;
\item ОС: Microsoft Windows 10 Pro Build 19042.685.
\end{itemize}

\par В рамках эксперимента было вычислено время работы параллельного и последовательно алгоритмов Дейкстры поиска кратчайших путей из одной вершины. Размер матриц смежности \verb|500| , \verb|1000| и \verb|2000|.
\par Результаты экспериментов представлены ниже.

\begin{table}[!h]
\caption{Результаты экспериментов для неориентированных графов с 500 вершинами и 70000 ребрами}
\centering
\begin{tabular}{lllll}
Процессы & Последовательно & Параллельно & Ускорение  \\
2        & 2.526570            & 1.354200       & 1.865729       \\
4        & 2.476640            & 0.835555       & 2.964065       \\
6        & 2.727910            & 3.045080       & 0.895841
\end{tabular}
\end{table}

\begin{table}[!h]
\caption{Результаты экспериментов для ориентированных графов с 500 вершинами и 220000 ребрами}
\centering
\begin{tabular}{lllll}
Процессы & Последовательно & Параллельно & Ускорение            \\
2        & 2.526570            & 1.397360        & 1.808102     \\
4        & 2.64650            & 0.871001        & 3.038458      \\
6        & 2.72791            & 3.208920        & 0.850102
\end{tabular}
\end{table}

\begin{table}[!h]
\caption{Результаты экспериментов для ориентированных графов с 1000 вершинами и 998000 ребрами}
\centering
\begin{tabular}{lllll}
Процессы & Последовательно & Параллельно & Ускорение            \\
2        & 20.7845            & 11.3295        & 1.834547       \\
4        & 22.5343            & 7.72283        & 2.917881       \\
6        & 21.7523            & 14.485        & 1.501712
\end{tabular}
\end{table}

\par По данным, полученным в результате экспериментов, можно сделать вывод о том, что параллельный случай работает действительно быстрее, чем последовательный.
\par Если увеличить количество процессов до 6, то ускорение наблюдается только для графов с большим количеством вершин. Так происходит из-за увеличения накладных расходов: происходит больше пересылок данных между процессами, увеличивается количество операций сравнения между локальными минимумами.
\newpage

% Заключение
\section*{Заключение}
\addcontentsline{toc}{section}{Заключение}
В ходе выполнения лабораторной работы были разработаны последовательная реализация алгоритма Дейкстры поиска кратчайших путей из одной вершины.
\par Результаты проведенных экспериментов показывают, что параллельная реализация работает быстрее, чем последовательная.
\par Для подтверждения корректности работы программы написаны тесты с использованием Google Testing Framework.
\newpage

% Список литературы
\begin{thebibliography}{1}
\addcontentsline{toc}{section}{Список литературы}
\bibitem{wikishell} Википедия: Алгоритм Дейкстры [Электронный ресурс] // URL: \url {https://en.wikipedia.org/wiki/Dijkstra%27s_algorithm} (дата обращения: 26.12.2020)
\bibitem{wikishell} Википедия: Задача о кратчайшем пути [Электронный ресурс] // URL: \url {https://en.wikipedia.org/wiki/Shortest_path_problem} (дата обращения: 26.12.2020)
\end{thebibliography}
\newpage

% Приложение
\section*{Приложение}
\addcontentsline{toc}{section}{Приложение}
В этом разделе находится листинг всего кода, написанного в рамках лабораторной работы.
\begin{lstlisting}
// dijkstra_algorithm.h
// Copyright 2020 Rustamov Azer
#ifndef MODULES_TASK_3_RUSTAMOV_A_DIJKSTRA_ALGORITHM_DIJKSTRA_ALGORITHM_H_
#define MODULES_TASK_3_RUSTAMOV_A_DIJKSTRA_ALGORITHM_DIJKSTRA_ALGORITHM_H_
#include <vector>

using Matrix = std::vector<double>;

Matrix RandomGraph(int verts, int edges, bool oriented = false);
Matrix RandomGraphSparce(int verts, int edges, bool oriented = false);
Matrix RandomGraphDence(int verts, int edges, bool oriented = false);

Matrix SequentialDijkstraAlgorithm(Matrix graph, int verts, int source_vertex);
Matrix ParallelDijkstraAlgorithm(Matrix graph, int verts, int source_vertex);

#endif  // MODULES_TASK_3_RUSTAMOV_A_DIJKSTRA_ALGORITHM_DIJKSTRA_ALGORITHM_H_
\end{lstlisting}
\begin{lstlisting}
// dijkstra_algorithm.cpp
// Copyright 2020 Rustamov Azer
#include <mpi.h>
#include <iostream>
#include <vector>
#include <random>
#include <limits>
#include "../../../modules/task_3/rustamov_a_dijkstra_algorithm/dijkstra_algorithm.h"

Matrix RandomGraph(int verts, int edges, bool oriented) {
    int max_edges = verts * (verts - 1) / 2;
    if (oriented) {
        max_edges *= 2;
    }
    if ((verts == 0) || (edges > max_edges)) {
        throw "Incorrect graph";
    }
    if (edges > max_edges / 2) {
        return RandomGraphDence(verts, edges, oriented);
    } else {
        return RandomGraphSparce(verts, edges, oriented);
    }
}

Matrix RandomGraphSparce(int verts, int edges, bool oriented) {
    int max_edges = verts * (verts - 1) / 2;
    if (oriented) {
        max_edges *= 2;
    }
    std::random_device rd;
    std::mt19937 mersenne(rd());
    std::uniform_real_distribution<> weight(0, 30);
    std::uniform_int_distribution<> pos(0, verts * verts - 1);
    double inf = std::numeric_limits<double>::infinity();
    Matrix matrix(verts * verts);
    std::fill(matrix.begin(), matrix.end(), inf);
    double edge_weight;
    int vert1, vert2, position;
    for (int edge = 0; edge < edges; edge++) {
        position = static_cast<int>(pos(mersenne));
        vert1 = position / verts;
        vert2 = position % verts;
        if ((vert1 == vert2) || (matrix[vert1 * verts + vert2] != inf)) {
            edge--;
        } else {
            edge_weight = static_cast<double>(weight(mersenne));
            matrix[vert1 * verts + vert2] = edge_weight;
            if (!oriented) {
                matrix[vert2 * verts + vert1] = edge_weight;
            }
        }
    }
    return matrix;
}

Matrix RandomGraphDence(int verts, int edges, bool oriented) {
    int max_edges = verts * (verts - 1) / 2;
    if (oriented) {
        max_edges *= 2;
    }
    std::random_device rd;
    std::mt19937 mersenne(rd());
    std::uniform_real_distribution<> weight(0, 30);
    std::uniform_int_distribution<> pos(0, verts * verts - 1);
    double inf = std::numeric_limits<double>::infinity();
    Matrix matrix(verts * verts);
    double edge_weight;
    for (int i = 0; i < verts; i++) {
        for (int j = 0; j < verts; j++) {
            if (i == j) {
                matrix[i * verts + j] = inf;
            } else {
                edge_weight = static_cast<double>(weight(mersenne));
                matrix[i * verts + j] = edge_weight;
            }
        }
    }
    int vert1, vert2, position;
    for (int cut = 0; cut < max_edges - edges; cut++) {
        position = static_cast<int>(pos(mersenne));
        vert1 = position / verts;
        vert2 = position % verts;
        if (matrix[vert1 * verts + vert2] == inf) {
            cut--;
        } else {
            matrix[vert1 * verts + vert2] = inf;
            if (!oriented) {
                matrix[vert2 * verts + vert1] = inf;
            }
        }
    }
    return matrix;
}

Matrix SequentialDijkstraAlgorithm(Matrix graph, int verts, int source_vertex) {
    if (graph.size() != verts * verts) {
        throw "Incorrect graph";
    }
    if (source_vertex >= verts) {
        throw "Incorrecr source vertex";
    }
    Matrix shortest_path_tree(verts * verts);
    double inf = std::numeric_limits<double>::infinity();
    std::fill(shortest_path_tree.begin(), shortest_path_tree.end(), inf);
    Matrix distance_to_verex(verts);
    std::fill(distance_to_verex.begin(), distance_to_verex.end(), inf);
    distance_to_verex[source_vertex] = 0.0;
    int closest_vert_in, closest_vert_out;
    double shortest_path;
    while (true) {
        shortest_path = inf;
        for (int vert_in = 0; vert_in < verts; vert_in++) {
            if (distance_to_verex[vert_in] != inf) {
                for (int vert_out = 0; vert_out < verts; vert_out++) {
                    if ((distance_to_verex[vert_out] == inf) &&
                        (graph[vert_in * verts + vert_out] < shortest_path)) {
                        closest_vert_in = vert_in;
                        closest_vert_out = vert_out;
                        shortest_path = graph[vert_in * verts + vert_out];
                    }
                }
            }
        }
        if (shortest_path == inf) {
            break;
        }
        shortest_path_tree[closest_vert_in * verts + closest_vert_out] = shortest_path;
        distance_to_verex[closest_vert_out] = distance_to_verex[closest_vert_in] +
           shortest_path;
    }
    return shortest_path_tree;
}


Matrix ParallelDijkstraAlgorithm(Matrix graph, int verts, int source_vertex) {
    if (graph.size() != verts * verts)
        throw "Incorrect graph";
    if (source_vertex >= verts)
        throw "Incorrecr source vertex";
    double inf = std::numeric_limits<double>::infinity();
    int procNum, procRank;
    MPI_Comm_size(MPI_COMM_WORLD, &procNum);
    MPI_Comm_rank(MPI_COMM_WORLD, &procRank);
    if (procNum == 1)
        return SequentialDijkstraAlgorithm(graph, verts, source_vertex);
    Matrix shortest_path_tree(verts * verts);
    std::fill(shortest_path_tree.begin(), shortest_path_tree.end(), inf);
    Matrix distance_to_verex(verts);
    std::fill(distance_to_verex.begin(), distance_to_verex.end(), inf);
    distance_to_verex[source_vertex] = 0.0;
    MPI_Status status;
    const int delta = verts / procNum;
    const int remain = verts % procNum;
    const int remain_for_proc = procRank < remain ? 1 : 0;
    Matrix local_graph((delta + remain_for_proc) * verts);
    std::fill(local_graph.begin(), local_graph.end(), inf);
    if (procRank == 0) {
        for (int row = 0; row < verts; row++) {
            if (row % procNum != 0) {
                MPI_Send(graph.data() + row * verts, verts, MPI_DOUBLE, row % procNum, 0, MPI_COMM_WORLD);
            } else {
                for (int vert = 0; vert < verts; vert++) {
                    local_graph[(row / procNum) * verts + vert] = graph[row * verts + vert];
                }
            }
        }
    } else {
        for (int local_vert = 0; local_vert < delta + remain_for_proc; local_vert++)
            MPI_Recv(local_graph.data() + local_vert * verts, verts, MPI_DOUBLE, 0, 0, MPI_COMM_WORLD, &status);
    }
    double local_shortest_path;
    Matrix global_shortest_paths(procNum);
    int local_closest_vert_in, local_closest_vert_out;
    bool done = false;
    int result_in, result_out;
    double result_path;
    while (true) {
        result_in = result_out = -1;
        result_path = inf;
        local_closest_vert_in = local_closest_vert_out = -1;
        local_shortest_path = inf;
        if (procRank == 0) {
            std::fill(global_shortest_paths.begin(), global_shortest_paths.end(), inf);
        }
        for (int vert_in = 0; vert_in < delta + remain_for_proc; vert_in++) {
            if (distance_to_verex[vert_in * procNum + procRank] != inf) {
                for (int vert_out = 0; vert_out < verts; vert_out++) {
                    if ((distance_to_verex[vert_out] == inf) &&
                    (local_graph[vert_in * verts + vert_out] < local_shortest_path)) {
                        local_closest_vert_in = vert_in * procNum + procRank;
                        local_closest_vert_out = vert_out;
                        local_shortest_path = local_graph[vert_in * verts + vert_out];
                    }
                }
            }
        }
        int curr_proc;
        if (procRank != 0) {
            MPI_Send(&local_shortest_path, 1, MPI_DOUBLE, 0, 0, MPI_COMM_WORLD);
            MPI_Bcast(&done, 1, MPI_C_BOOL, 0, MPI_COMM_WORLD);
        } else {
            global_shortest_paths[0] = local_shortest_path;
            for (int proc = 1; proc < procNum; proc++) {
                MPI_Recv(global_shortest_paths.data() + proc, 1, MPI_DOUBLE, proc, 0, MPI_COMM_WORLD, &status);
            }
            for (int proc = 0; proc < procNum; proc++) {
                if (result_path > global_shortest_paths[proc]) {
                    result_path = global_shortest_paths[proc];
                    curr_proc = proc;
                }
            }
            if (result_path == inf) {
                done = true;
            }
            MPI_Bcast(&done, 1, MPI_C_BOOL, 0, MPI_COMM_WORLD);
        }
        if (done) {
            break;
        }
        MPI_Bcast(&curr_proc, 1, MPI_INT, 0, MPI_COMM_WORLD);
        if (curr_proc != 0) {
            if (curr_proc == procRank) {
                MPI_Send(&local_closest_vert_in, 1, MPI_INT, 0, 0, MPI_COMM_WORLD);
                MPI_Send(&local_closest_vert_out, 1, MPI_INT, 0, 0, MPI_COMM_WORLD);
            }
            if (procRank == 0) {
                MPI_Recv(&result_in, 1, MPI_INT, curr_proc, 0, MPI_COMM_WORLD, &status);
                MPI_Recv(&result_out, 1, MPI_INT, curr_proc, 0, MPI_COMM_WORLD, &status);
            }
        } else {
            result_in = local_closest_vert_in;
            result_out = local_closest_vert_out;
        }
        if (procRank == 0) {
            distance_to_verex[result_out] = distance_to_verex[result_in] + result_path;
            shortest_path_tree[result_in * verts + result_out] = result_path;
        }
        MPI_Bcast(distance_to_verex.data(), verts, MPI_DOUBLE, 0, MPI_COMM_WORLD);
    }
    return shortest_path_tree;
}
\end{lstlisting}
\begin{lstlisting}
// main.cpp
// Copyright 2020 Rustamov Azer
#include <mpi.h>
#include <gtest-mpi-listener.hpp>
#include <gtest/gtest.h>
#include <vector>
#include <limits>
#include "./dijkstra_algorithm.h"

#define EPSILON 0.000001

TEST(Dijkstra_Algorithm, Incorrect_Graph) {
    int procNum, procRank;
    MPI_Comm_size(MPI_COMM_WORLD, &procNum);
    MPI_Comm_rank(MPI_COMM_WORLD, &procRank);
    int verts = 2, edges = 10;
    Matrix graph;
    if (procRank == 0) {
        ASSERT_ANY_THROW(RandomGraph(verts, edges));
    }
}

TEST(Dijkstra_Algorithm, Correct_Answer_Unoriented_5_Seq) {
    int procRank;
    double inf = std::numeric_limits<double>::infinity();
    MPI_Comm_rank(MPI_COMM_WORLD, &procRank);
    int verts = 5;
    Matrix graph = {inf, inf, 7,   3,   9,
                    inf, inf, 2,   15,  4,
                    7,   2,   inf, 10,  inf,
                    3,   15,  10,  inf, inf,
                    9,   4,   inf, inf, inf };
    if (procRank == 0) {
        Matrix spt = SequentialDijkstraAlgorithm(graph, verts, 0);
        Matrix exspt = {inf, inf, 7,   3,   inf,
                        inf, inf, inf, inf, 4,
                        inf, 2,   inf, inf, inf,
                        inf, inf, inf, inf, inf,
                        inf, inf, inf, inf, inf };
        for (int pos = 0; pos < verts * verts; pos++) {
            if ((spt[pos] != inf) || (exspt[pos] != inf)) {
                ASSERT_NEAR(spt[pos], exspt[pos], EPSILON);
            }
        }
    }
}

TEST(Dijkstra_Algorithm, Correct_Answer_Unoriented_5_Par) {
    int procRank;
    double inf = std::numeric_limits<double>::infinity();
    MPI_Comm_rank(MPI_COMM_WORLD, &procRank);
    int verts = 5;
    Matrix graph = {inf, inf, 7,   3,   9,
                    inf, inf, 2,   15,  4,
                    7,   2,   inf, 10,  inf,
                    3,   15,  10,  inf, inf,
                    9,   4,   inf, inf, inf };
    Matrix spt = ParallelDijkstraAlgorithm(graph, verts, 0);
    if (procRank == 0) {
        Matrix exspt = {inf, inf, 7,   3,   inf,
                        inf, inf, inf, inf, 4,
                        inf, 2,   inf, inf, inf,
                        inf, inf, inf, inf, inf,
                        inf, inf, inf, inf, inf };
        for (int pos = 0; pos < verts * verts; pos++) {
            if ((spt[pos] != inf) || (exspt[pos] != inf)) {
                ASSERT_NEAR(spt[pos], exspt[pos], EPSILON);
            }
        }
    }
}

TEST(Dijkstra_Algorithm, 5_9_Unoriented_Source_0) {
    int procRank;
    double inf = std::numeric_limits<double>::infinity();
    MPI_Comm_rank(MPI_COMM_WORLD, &procRank);
    int verts = 5;
    int edges = 9;
    bool is_oriented = false;
    Matrix graph = RandomGraph(verts, edges, is_oriented);
    Matrix spt = ParallelDijkstraAlgorithm(graph, verts, 0);
    if (procRank == 0) {
        Matrix seq_spt = SequentialDijkstraAlgorithm(graph, verts, 0);
        for (int pos = 0; pos < verts * verts; pos++) {
            if ((spt[pos] != inf) || (seq_spt[pos] != inf)) {
                ASSERT_NEAR(spt[pos], seq_spt[pos], EPSILON);
            }
        }
    }
}

TEST(Dijkstra_Algorithm, 5_9_Oriented_Source_0) {
    int procRank;
    double inf = std::numeric_limits<double>::infinity();
    MPI_Comm_rank(MPI_COMM_WORLD, &procRank);
    int verts = 5;
    int edges = 9;
    bool is_oriented = true;
    Matrix graph = RandomGraph(verts, edges, is_oriented);
    Matrix spt = ParallelDijkstraAlgorithm(graph, verts, 0);
    if (procRank == 0) {
        Matrix seq_spt = SequentialDijkstraAlgorithm(graph, verts, 0);
        for (int pos = 0; pos < verts * verts; pos++) {
            if ((spt[pos] != inf) || (seq_spt[pos] != inf)) {
                ASSERT_NEAR(spt[pos], seq_spt[pos], EPSILON);
            }
        }
    }
}


TEST(Dijkstra_Algorithm, 5_9_Unoriented_Source_1) {
    int procRank;
    double inf = std::numeric_limits<double>::infinity();
    MPI_Comm_rank(MPI_COMM_WORLD, &procRank);
    int verts = 5;
    int edges = 9;
    bool is_oriented = false;
    Matrix graph = RandomGraph(verts, edges, is_oriented);
    Matrix spt = ParallelDijkstraAlgorithm(graph, verts, 1);
    if (procRank == 0) {
        Matrix seq_spt = SequentialDijkstraAlgorithm(graph, verts, 1);
        for (int pos = 0; pos < verts * verts; pos++) {
            if ((spt[pos] != inf) || (seq_spt[pos] != inf)) {
                ASSERT_NEAR(spt[pos], seq_spt[pos], EPSILON);
            }
        }
    }
}

TEST(Dijkstra_Algorithm, 5_9_Oriented_Source_3) {
    int procRank;
    double inf = std::numeric_limits<double>::infinity();
    MPI_Comm_rank(MPI_COMM_WORLD, &procRank);
    int verts = 5;
    int edges = 9;
    bool is_oriented = true;
    Matrix graph = RandomGraph(verts, edges, is_oriented);
    Matrix spt = ParallelDijkstraAlgorithm(graph, verts, 3);
    if (procRank == 0) {
        Matrix seq_spt = SequentialDijkstraAlgorithm(graph, verts, 3);
        for (int pos = 0; pos < verts * verts; pos++) {
            if ((spt[pos] != inf) || (seq_spt[pos] != inf)) {
                ASSERT_NEAR(spt[pos], seq_spt[pos], EPSILON);
            }
        }
    }
}

TEST(Dijkstra_Algorithm, 10_30_Unoriented_Source_0) {
    int procRank;
    double inf = std::numeric_limits<double>::infinity();
    MPI_Comm_rank(MPI_COMM_WORLD, &procRank);
    int verts = 10;
    int edges = 30;
    bool is_oriented = false;
    Matrix graph = RandomGraph(verts, edges, is_oriented);
    double time_start_par = MPI_Wtime();
    Matrix spt = ParallelDijkstraAlgorithm(graph, verts, 0);
    if (procRank == 0) {
        double time_end_par = MPI_Wtime();
        std::cout << "PAR: " << time_end_par - time_start_par << std::endl;
    }
    if (procRank == 0) {
        double time_start_seq = MPI_Wtime();
        Matrix seq_spt = SequentialDijkstraAlgorithm(graph, verts, 0);
        double time_end_seq = MPI_Wtime();
        std::cout << "SEQ: " << time_end_seq - time_start_seq << std::endl;
        for (int pos = 0; pos < verts * verts; pos++) {
            if ((spt[pos] != inf) || (seq_spt[pos] != inf)) {
                ASSERT_NEAR(spt[pos], seq_spt[pos], EPSILON);
            }
        }
    }
}

TEST(Dijkstra_Algorithm, 100_400_Unoriented_Source_0) {
    int procRank;
    double inf = std::numeric_limits<double>::infinity();
    MPI_Comm_rank(MPI_COMM_WORLD, &procRank);
    int verts = 100;
    int edges = 400;
    bool is_oriented = false;
    Matrix graph = RandomGraph(verts, edges, is_oriented);
    double time_start_par = MPI_Wtime();
    Matrix spt = ParallelDijkstraAlgorithm(graph, verts, 0);
    if (procRank == 0) {
        double time_end_par = MPI_Wtime();
        std::cout << "PAR: " << time_end_par - time_start_par << std::endl;
    }
    if (procRank == 0) {
        double time_start_seq = MPI_Wtime();
        Matrix seq_spt = SequentialDijkstraAlgorithm(graph, verts, 0);
        double time_end_seq = MPI_Wtime();
        std::cout << "SEQ: " << time_end_seq - time_start_seq << std::endl;
        for (int pos = 0; pos < verts * verts; pos++) {
            if ((spt[pos] != inf) || (seq_spt[pos] != inf)) {
                ASSERT_NEAR(spt[pos], seq_spt[pos], EPSILON);
            }
        }
    }
}

TEST(Dijkstra_Algorithm, 200_400_Unoriented_Source_0) {
    int procRank;
    double inf = std::numeric_limits<double>::infinity();
    MPI_Comm_rank(MPI_COMM_WORLD, &procRank);
    int verts = 200;
    int edges = 400;
    bool is_oriented = false;
    Matrix graph = RandomGraph(verts, edges, is_oriented);
    double time_start_par = MPI_Wtime();
    Matrix spt = ParallelDijkstraAlgorithm(graph, verts, 0);
    if (procRank == 0) {
        double time_end_par = MPI_Wtime();
        std::cout << "PAR: " << time_end_par - time_start_par << std::endl;
    }
    if (procRank == 0) {
        double time_start_seq = MPI_Wtime();
        Matrix seq_spt = SequentialDijkstraAlgorithm(graph, verts, 0);
        double time_end_seq = MPI_Wtime();
        std::cout << "SEQ: " << time_end_seq - time_start_seq << std::endl;
        for (int pos = 0; pos < verts * verts; pos++) {
            if ((spt[pos] != inf) || (seq_spt[pos] != inf)) {
                ASSERT_NEAR(spt[pos], seq_spt[pos], EPSILON);
            }
        }
    }
}

TEST(Dijkstra_Algorithm, 500_500_Unoriented_Source_0) {
    int procRank;
    double inf = std::numeric_limits<double>::infinity();
    MPI_Comm_rank(MPI_COMM_WORLD, &procRank);
    int verts = 500;
    int edges = 500;
    bool is_oriented = false;
    Matrix graph = RandomGraph(verts, edges, is_oriented);
    double time_start_par = MPI_Wtime();
    Matrix spt = ParallelDijkstraAlgorithm(graph, verts, 0);
    if (procRank == 0) {
        double time_end_par = MPI_Wtime();
        std::cout << "PAR: " << time_end_par - time_start_par << std::endl;
    }
    if (procRank == 0) {
        double time_start_seq = MPI_Wtime();
        Matrix seq_spt = SequentialDijkstraAlgorithm(graph, verts, 0);
        double time_end_seq = MPI_Wtime();
        std::cout << "SEQ: " << time_end_seq - time_start_seq << std::endl;
        for (int pos = 0; pos < verts * verts; pos++) {
            if ((spt[pos] != inf) || (seq_spt[pos] != inf)) {
                ASSERT_NEAR(spt[pos], seq_spt[pos], EPSILON);
            }
        }
    }
}

TEST(Dijkstra_Algorithm, 500_70000_Unoriented_Source_0) {
    int procRank;
    double inf = std::numeric_limits<double>::infinity();
    MPI_Comm_rank(MPI_COMM_WORLD, &procRank);
    int verts = 500;
    int edges = 70000;
    bool is_oriented = false;
    Matrix graph = RandomGraph(verts, edges, is_oriented);
    double time_start_par = MPI_Wtime();
    Matrix spt = ParallelDijkstraAlgorithm(graph, verts, 0);
    if (procRank == 0) {
        double time_end_par = MPI_Wtime();
        std::cout << "PAR: " << time_end_par - time_start_par << std::endl;
    }
    if (procRank == 0) {
        double time_start_seq = MPI_Wtime();
        Matrix seq_spt = SequentialDijkstraAlgorithm(graph, verts, 0);
        double time_end_seq = MPI_Wtime();
        std::cout << "SEQ: " << time_end_seq - time_start_seq << std::endl;
        for (int pos = 0; pos < verts * verts; pos++) {
            if ((spt[pos] != inf) || (seq_spt[pos] != inf)) {
                ASSERT_NEAR(spt[pos], seq_spt[pos], EPSILON);
            }
        }
    }
}

TEST(Dijkstra_Algorithm, 500_220000_Oriented_Source_0) {
    int procRank;
    double inf = std::numeric_limits<double>::infinity();
    MPI_Comm_rank(MPI_COMM_WORLD, &procRank);
    int verts = 500;
    int edges = 220000;
    bool is_oriented = true;
    Matrix graph = RandomGraph(verts, edges, is_oriented);
    double time_start_par = MPI_Wtime();
    Matrix spt = ParallelDijkstraAlgorithm(graph, verts, 0);
    if (procRank == 0) {
        double time_end_par = MPI_Wtime();
        std::cout << "PAR: " << time_end_par - time_start_par << std::endl;
    }
    if (procRank == 0) {
        double time_start_seq = MPI_Wtime();
        Matrix seq_spt = SequentialDijkstraAlgorithm(graph, verts, 0);
        double time_end_seq = MPI_Wtime();
        std::cout << "SEQ: " << time_end_seq - time_start_seq << std::endl;
        for (int pos = 0; pos < verts * verts; pos++) {
            if ((spt[pos] != inf) || (seq_spt[pos] != inf)) {
                ASSERT_NEAR(spt[pos], seq_spt[pos], EPSILON);
            }
        }
    }
}

TEST(Dijkstra_Algorithm, 1000_998000_Oriented_Source_0) {
    int procRank;
    double inf = std::numeric_limits<double>::infinity();
    MPI_Comm_rank(MPI_COMM_WORLD, &procRank);
    int verts = 1000;
    int edges = 998000;
    bool is_oriented = true;
    Matrix graph = RandomGraph(verts, edges, is_oriented);
    double time_start_par = MPI_Wtime();
    Matrix spt = ParallelDijkstraAlgorithm(graph, verts, 0);
    if (procRank == 0) {
        double time_end_par = MPI_Wtime();
        std::cout << "PAR: " << time_end_par - time_start_par << std::endl;
    }
    if (procRank == 0) {
        double time_start_seq = MPI_Wtime();
        Matrix seq_spt = SequentialDijkstraAlgorithm(graph, verts, 0);
        double time_end_seq = MPI_Wtime();
        std::cout << "SEQ: " << time_end_seq - time_start_seq << std::endl;
        for (int pos = 0; pos < verts * verts; pos++) {
            if ((spt[pos] != inf) || (seq_spt[pos] != inf)) {
                ASSERT_NEAR(spt[pos], seq_spt[pos], EPSILON);
            }
        }
    }
}

int main(int argc, char* argv[]) {
    ::testing::InitGoogleTest(&argc, argv);
    MPI_Init(&argc, &argv);

    ::testing::AddGlobalTestEnvironment(new GTestMPIListener::MPIEnvironment);
    ::testing::TestEventListeners &listeners =
        ::testing::UnitTest::GetInstance()->listeners();

    listeners.Release(listeners.default_result_printer());
    listeners.Release(listeners.default_xml_generator());

    listeners.Append(new GTestMPIListener::MPIMinimalistPrinter);
    return RUN_ALL_TESTS();
}
\end{lstlisting}

\end{document}