\documentclass{report}

\usepackage[T2A]{fontenc}
\usepackage[utf8]{luainputenc}
\usepackage[english, russian]{babel}
\usepackage[pdftex]{hyperref}
\usepackage[14pt]{extsizes}
\usepackage{listings}
\usepackage{color}
\usepackage{geometry}
\usepackage{enumitem}
\usepackage{multirow}
\usepackage{graphicx}
\usepackage{indentfirst}

\geometry{a4paper,top=2cm,bottom=3cm,left=2cm,right=1.5cm}
\setlength{\parskip}{0.5cm}
\setlist{nolistsep, itemsep=0.3cm,parsep=0pt}

\lstset{language=C++,
		basicstyle=\footnotesize,
		keywordstyle=\color{blue}\ttfamily,
		stringstyle=\color{red}\ttfamily,
		commentstyle=\color{green}\ttfamily,
		morecomment=[l][\color{magenta}]{\#}, 
		tabsize=4,
		breaklines=true,
  		breakatwhitespace=true,
  		title=\lstname,       
}

\makeatletter
\renewcommand\@biblabel[1]{#1.\hfil}
\makeatother

\begin{document}

\begin{titlepage}

\begin{center}Министерство образования и науки Российской Федерации 
Федеральное государственное автономное образовательное учреждение 
высшего образования Национальный исследовательский 
Нижегородский государственный университет им. Н.И. Лобачевского 
Институт информационных технологий, математики и механики

\vspace{8em}
\textbf{\LargeОтчет по лабораторной работе "Умножение разреженных матриц"} \end{center}

\vspace{8em}

\begin{flushright}  \textbf{Выполнил:} \\ студент группы 381808-1  \\ Созинов А.П. \end{flushright}
\begin{flushright}  \textbf{Проверил:}  \\ доцент кафедры МОСТ, \\ кандидат технических наук \\ Сысоев А. В.\\ \end{flushright}

\vspace{\fill}
\begin{center} Нижний Новгород  2020г \end{center}
\end{titlepage}

\setcounter{page}{2}

% Содержание
\tableofcontents

\newpage
% Введение
\section*{Введение}
\addcontentsline{toc}{section}{Введение}
\par Алгебра разреженных матриц - важный раздел математики, имеющий практическое применение. Разреженные матрицы
встречаются при постановке и решении задач в различных научных и технических областях (при постановке и решении оптимизационных задач,
при численном решении дифференциальных уравненийв частных производных).
\par Целью лабораторной работы является изучение принципа хранения и алгоритмов обработки разреженных матриц. Реализовать алгоритм умножения разреженных матриц с использованием технологии MPI для параллельного вычисления.
\par В данной работе в качестве структуры хранения матрицы был выбран столбцовый вариант - CCS. 
\newpage

% Постановка задачи
\section*{Постановка задачи}
\addcontentsline{toc}{section}{Постановка задачи}
\par Пусть A и B –  матрицы размера M x N и N x K, в которых процент ненулевых элементов мал. Будем считать, что элементы матриц A и B – вещественные числа (далее в программной реализации будем использовать числа с плавающей точкой двойной точности).

\par Требуется найти матрицу C = A * B, где символ * соответствует матричному умножению.


\begin{table}[!h]
\caption{A}
\begin{tabular}{ | l | l | l | l | l | }
\hline
0 & 0 & 0 & -7 & 0 \\ \hline
0 & 6 & 0 & 0 & 0 \\ \hline
0 & 0 & 0 & 0 & 0 \\ \hline
0 & 1 & 0 & 0 & 5 \\
\hline
\end{tabular}
\end{table}

\begin{table}[!h]
 \caption{B}
\begin{tabular}{ | l | l | l | }
\hline
0 & 0 & 7 \\ \hline
0 & 3 & 0 \\ \hline
0 & 0 & 0 \\ \hline
33 & 0 & 0 \\ \hline
0 & 0 & 2 \\
\hline
\end{tabular}
\end{table}

\begin{table}[!h]
  \caption{C = A * B}
\begin{tabular}{ | l | l | l | }
\hline
-231 & 0 & 0 \\ \hline
0 & 18 & 0 \\ \hline
0 & 0 & 0 \\ \hline
0 & 0 & 10 \\
\hline
\end{tabular}
\end{table}


\par Для этого нужно выполнить следующие задачи:
\begin{enumerate}
\item Разработать структуру хранения разреженной матрицы;
\item Разработать алгоритм перемножения матрицы;
\item Проверить на работоспособность с помощью библиотеки модульного тестирования Google Testing Framework.
\end{enumerate}
\newpage

% Описание алгоритма
\section*{Описание алгоритма}
\addcontentsline{toc}{section}{Описание алгоритма}
\par Алгоритм умножения разреженных матриц выполним по определению (для каждой ячейки вычисляется скалярное произведение строки одной матрицы на столбцы другой матрицы). 
\par Так как для хранения матрицы был выбран столбцовый вариант, то и алгоритм будет проходить по столбцам результирующей матрицы для последовательного заполнения значений
\par Из недостатков можно отметить не удобное выделение элементов строки первой матрицы.

\newpage
% Схема распараллеливания
\section*{Схема распараллеливания}
\addcontentsline{toc}{section}{Схема распараллеливания}
\par Имеем матрицы A - M x N и В - N x K. 
\par Для распараллеливания алгоритма умножения CCS-матриц необходимо:
\begin{enumerate}
\item Убедиться, что количество столбцов  матрицы A равно количеству строк матрицы B
\item Распределить K столбцов между процессами
\item Для каждой матрицы M x $K_i$, где i - номер процесса, вычисляем значения ее ячеек.
\item Следующим шагом собираем данные в корневой процесс и устанавливаем корректные стартовые позиции для каждого столбца.
\end{enumerate}
\newpage

% Описание программной реализации
\section*{Описание программной реализации}
\addcontentsline{toc}{section}{Описание программной реализации}
В программе имеются следующие функции и структуры
\begin{enumerate}
\item Структура хранения
\begin{lstlisting}
	struct ComprColStorage {
		int countRow;
  		int countCol;
  		std::vector<double> value;
  		std::vector<int> startPos;
		std::vector<int> numberRow;
\end{lstlisting}

\item Функция генерации матрицы
\begin{lstlisting}
	ComprColStorage GenerateMatrix(int countRow, int countCol, double proc);
\end{lstlisting}

\item Последовательная реализация алгоритма
\begin{lstlisting}
	ComprColStorage GetSequentialSolution(ComprColStorage *first, ComprColStorage *second);
\end{lstlisting}

\item Параллельная реализация алгоритма
\begin{lstlisting}
	ComprColStorage GetParalSolution(ComprColStorage *first, ComprColStorage *second);
\end{lstlisting}

\item Функция нахождения значения для ячейки результирующей матрицы
\begin{lstlisting}
	double Sum(const ComprColStorage &first, const ComprColStorage &second, int curRow, int curCol, int size);
\end{lstlisting}

\item Функция вывода матрицы на экран
Стоит обратить внимание, что данный метод выводит ее в следующем формате:
кличество строк, количество столбцов\\ стартовые позиции\\строки, в которых находятся значения\\значения матрицы\\
\begin{lstlisting}
	void PrintMatr(const ComprColStorage &matr);
\end{lstlisting}

\end{enumerate}
\newpage

% Результаты экспериментов
\section*{Результаты экспериментов}
\addcontentsline{toc}{section}{Результаты экспериментов}
Эксперименты проводились на устройстве со следующей конфигурацией:
\begin{enumerate}
\item Процессор: Intel Core i5-2410M, 2300 МГц, ядер - 2;
\item Оперативная память:  4Гб, DDR3;
\item ОС: Windows 10;
\end{enumerate}
 Результаты экспериментов приведены в таблице:

\begin{table}[!h]
\caption{Результаты вычислительных экспериментов}
\centering
\begin{tabular}{ | l | l | l | l  | l | l | }
\hline
Процессы & Матрица А & Матрица В & \% заполнения & Последовательно & Параллельно \\ \hline
1 & 500x500 & 500x500 & 17 & 48.490 & 48.640 \\ \hline
2 & 500x500 & 500x500 & 17 & 50.250 & 25.423 \\ \hline
3 & 500x500 & 500x500 & 17 & 52.137 & 20.053 \\ \hline
4 & 500x500 & 500x500 & 17 & 53.283 & 15.235 \\ \hline
1 & 300x600 & 600x200 & 18.5 & 10.273 & 10.157 \\ \hline
2 & 300x600 & 600x200 & 18.5 & 11.471 & 5.415 \\ \hline
3 & 300x600 & 600x200 & 18.5 & 11.582 & 3.574 \\ \hline
4 & 300x600 & 600x200 & 18.5 & 10.125 & 2.854 \\ \hline
1 & 800x1000 & 1000x900 & 20 & 400.633 & 405.424 \\ \hline
2 & 800x1000 & 1000x900 & 20 & 405.383 & 207.365 \\ \hline
3 & 800x1000 & 1000x900 & 20 & 403.715 & 136.386 \\ \hline
4 & 800x1000 & 1000x900 & 20 & 400.152 & 110.150 \\ \hline
\end{tabular}
\end{table}

\par По полученным данным можно сделать вывод, что параллельная реализация алгоритма умножения разреженных матриц выполняется быстрее последовательного

\newpage
% Заключение
\section*{Заключение}
\addcontentsline{toc}{section}{Заключение}
\par В результате лабораторной работы было выполнено:
\begin{enumerate}
\item Разработана труктура хранения разреженной матрицы;
\item Разработан алгоритм перемножения матриц;
\item Проверено на работоспособность с помощью библиотеки модульного тестирования Google Testing Framework.
\end{enumerate}
\par Основной задачей данной лабораторной работы была реализация параллельной версии, которая должна была быть эффективнее последовательной. Эта задача была успешно достигнута, о чем говорят результаты экспериментов, проведенных в ходе работы. Они показывают, что параллельный вариант работает действительно быстрее, чем последовательный.
\newpage

% Литература
\begin{thebibliography}{1}
\addcontentsline{toc}{section}{Список литературы}
\bibitem{Intel} Intuit: Академия Intel: Intel Parallel Programming Professional (Introduction) [Электронный ресурс] // URL: \url { https://intuit.ru/studies/courses/4447/983/lecture/14931} (дата обращения: 11.12.2020)
\bibitem{Parallel} Parallel: Лаборатория Параллельных информационных технологий НИВЦ МГУ [Электронный ресурс] // URL: \url {https://parallel.ru/vvv/mpi.html} (дата обращения: 11.12.2020)

\end{thebibliography}

\newpage
% Приложение - код
\section*{Приложение}
\addcontentsline{toc}{section}{Приложение}
\begin{lstlisting}
// sparse_matrix_col.h


// Copyright 2020 Sozinov Alex
#ifndef MODULES_TASK_3_SOZINOV_A_SPARSE_MATRIX_COL_SPARSE_MATRIX_COL_H_
#define MODULES_TASK_3_SOZINOV_A_SPARSE_MATRIX_COL_SPARSE_MATRIX_COL_H_

#include <vector>
#include <iostream>

struct ComprColStorage {
  int countRow;
  int countCol;
  std::vector<double> value;
  std::vector<int> startPos;  // starting vector position (vector value) for i column
  std::vector<int> numberRow;  // number Row for element std::vector value

  bool operator==(const ComprColStorage& sec) const;
  bool operator!=(const ComprColStorage& sec) const;
};

ComprColStorage GenerateMatrix(int countRow, int countCol, double proc);
ComprColStorage GetSequentialSolution(ComprColStorage *first, ComprColStorage *second);
ComprColStorage GetParalSolution(ComprColStorage *first, ComprColStorage *second);
double Sum(const ComprColStorage &first, const ComprColStorage &second, int curRow, int curCol, int size);
void PrintMatr(const ComprColStorage &matr);
#endif  // MODULES_TASK_3_SOZINOV_A_SPARSE_MATRIX_COL_SPARSE_MATRIX_COL_H_
\end{lstlisting}


\begin{lstlisting}

// sparse_matrix_col.cpp


// Copyright 2020 Sozinov Alex

#include <mpi.h>
#include <algorithm>
#include <cmath>
#include <random>
#include <vector>
#include "../../modules/task_3/sozinov_a_sparse_matrix_col/sparse_matrix_col.h"

ComprColStorage GenerateMatrix(int countRow, int countCol, double proc) {
  int countElem = std::ceil(countRow * countCol * proc / 100);
  ComprColStorage Matr {
    countRow,
    countCol,
    std::vector<double>(countElem, 0),
    std::vector<int> (countCol + 1, 0),
    std::vector<int> (countElem, -1)
  };

  std::random_device dev;
  std::mt19937 ger(dev());
  std::uniform_real_distribution<> realDist(-100, 100);
  std::uniform_int_distribution<> intColDist(0, countCol - 1);
  std::uniform_int_distribution<> intRowDist(0, countRow - 1);

  // Generate elements
  int curElem = 0;
  while (curElem < countElem) {
    double elem = realDist(ger);
    if (elem != 0) {
      Matr.value[curElem] = elem;
      ++curElem;
    }
  }

  // Generate col
  curElem = 0;
  while (curElem < countElem) {
    int col = intColDist(ger);
    if (Matr.startPos[col + 1] - Matr.startPos[col] < countRow) {
      for (int j = col + 1; j < countCol + 1; ++j) {
        ++Matr.startPos[j];
      }
      ++curElem;
    }
  }

  // Generate row
  for (int curCol = 0; curCol < countCol; ++curCol) {
    int index = Matr.startPos[curCol];
    while (index < Matr.startPos[curCol + 1]) {
      int row = intRowDist(ger);
      bool F = true;
      for (int curPos = Matr.startPos[curCol]; curPos < index && F; ++curPos) {
        if (Matr.numberRow[curPos] == row)
          F = false;
      }
      if (F) {
        Matr.numberRow[index] = row;
        ++index;
      }
    }
    std::vector<int>::iterator startIt = Matr.numberRow.begin() + Matr.startPos[curCol];
    std::vector<int>::iterator endIt = Matr.numberRow.begin() + Matr.startPos[curCol + 1];
    std::sort(startIt, endIt);
  }
  MPI_Bcast(Matr.value.data(), Matr.value.size(), MPI_DOUBLE, 0, MPI_COMM_WORLD);
  MPI_Bcast(Matr.numberRow.data(), Matr.numberRow.size(), MPI_INT, 0, MPI_COMM_WORLD);
  MPI_Bcast(Matr.startPos.data(), Matr.startPos.size(), MPI_INT, 0, MPI_COMM_WORLD);
  return Matr;
}

ComprColStorage GetSequentialSolution(ComprColStorage *first, ComprColStorage *second) {
  if (first->countCol == 0 || first->countRow == 0 || second->countCol == 0 || second->countRow == 0)
    return ComprColStorage();
  if (first->countCol != second->countRow)
    return ComprColStorage();
  ComprColStorage resMatr{
    first->countRow,
    second->countCol,
    std::vector<double>(),
    std::vector<int>(second->countCol + 1, 0),
    std::vector<int>()
  };
  for (int curCol = 0; curCol < resMatr.countCol; ++curCol) {
    for (int curRow = 0; curRow < resMatr.countRow; ++curRow) {
      double curSum = Sum(*first, *second, curRow, curCol, first->countCol);
      if (curSum != 0) {
        resMatr.value.push_back(curSum);
        resMatr.numberRow.push_back(curRow);
        for (size_t i = curCol + 1; i < resMatr.startPos.size(); ++i) {
          ++resMatr.startPos[i];
        }
      }
    }
  }

  return resMatr;
}

ComprColStorage GetParalSolution(ComprColStorage *first, ComprColStorage *second) {
  if (first->countCol == 0 || first->countRow == 0 || second->countCol == 0 || second->countRow == 0)
    return ComprColStorage();
  if (first->countCol != second->countRow)
    return ComprColStorage();
  ComprColStorage resMatr {
    first->countRow,
    second->countCol,
    std::vector<double>(),
    std::vector<int>(second->countCol + 1, 0),
    std::vector<int>()
  };

  int ProcCount;
  int ProcRank;
  MPI_Comm_rank(MPI_COMM_WORLD, &ProcRank);
  MPI_Comm_size(MPI_COMM_WORLD, &ProcCount);
  int offset = resMatr.countCol / ProcCount;
  std::vector<int> countColInProc(ProcCount, offset);

  for (int i = 0; i < resMatr.countCol % ProcCount; ++i)
    ++countColInProc[i];
  if (resMatr.countCol % ProcCount)
    ++offset;
  std::vector<int> globCol(ProcCount, 0);
  for (int i = 1; i < ProcCount; ++i) {
    globCol[i] = globCol[i - 1] + countColInProc[i - 1];
  }
  ComprColStorage locMatr{
    resMatr.countRow,
    countColInProc[ProcRank],
    std::vector<double>(),
    std::vector<int>(locMatr.countCol + 1, 0),
    std::vector<int>()
  };
  int count = 0;
  for (int curCol = 0; curCol < locMatr.countCol; ++curCol) {
    for (int curRow = 0; curRow < locMatr.countRow; ++curRow) {
      double curSum = Sum(*first, *second, curRow, globCol[ProcRank] + curCol, first->countCol);
      if (curSum != 0) {
        ++count;
        locMatr.value.push_back(curSum);
        locMatr.numberRow.push_back(curRow);
        for (size_t i = curCol + 1; i < locMatr.startPos.size(); ++i)
          ++locMatr.startPos[i];
      }
    }
  }
  int locElem = locMatr.value.size();
  std::vector<int> countElem(ProcCount, 0);
  int totalCountElem;

  MPI_Allgather(&locElem, 1, MPI_INT, &countElem[0], 1, MPI_INT, MPI_COMM_WORLD);
  MPI_Allreduce(&locElem, &totalCountElem, 1, MPI_INT, MPI_SUM, MPI_COMM_WORLD);

  resMatr.numberRow = std::vector<int>(totalCountElem, 0);
  resMatr.value = std::vector<double>(totalCountElem, 0);
  std::vector<int> offsetValueAndNumRow(ProcCount, 0);
  std::vector<int> offsetStartPos(ProcCount, 0);
  for (int i = 1; i < ProcCount; ++i) {
    offsetValueAndNumRow[i] = offsetValueAndNumRow[i - 1] + countElem[i - 1];
    offsetStartPos[i] = offsetStartPos[i - 1] + countColInProc[i - 1];
  }

  MPI_Gatherv(locMatr.startPos.data(), countColInProc[ProcRank], MPI_INT,
    resMatr.startPos.data(), &countColInProc[0], &offsetStartPos[0], MPI_INT, 0, MPI_COMM_WORLD);
  MPI_Gatherv(locMatr.numberRow.data(), countElem[ProcRank], MPI_INT,
    resMatr.numberRow.data(), &countElem[0], &offsetValueAndNumRow[0], MPI_INT, 0, MPI_COMM_WORLD);
  MPI_Gatherv(locMatr.value.data(), countElem[ProcRank], MPI_DOUBLE,
    resMatr.value.data(), &countElem[0], &offsetValueAndNumRow[0], MPI_DOUBLE, 0, MPI_COMM_WORLD);

  if (ProcRank == 0) {
    int curStartPos = locMatr.startPos[countColInProc[0]];
    int curCol = countColInProc[0];
    for (int i = 1; i < ProcCount; ++i) {
      for (int j = 0; j < countColInProc[i]; ++j) {
        resMatr.startPos[curCol++] += curStartPos;
      }
      curStartPos = curStartPos + countElem[i];
    }
    resMatr.startPos[resMatr.countCol] = totalCountElem;
  }

  return resMatr;
}

double Sum(const ComprColStorage& first, const ComprColStorage& second, int curRow, int curCol, int size) {
  int distCol = 0;
  int startPosCol = second.startPos[curCol];
  int endPosCol = second.startPos[curCol + 1];
  double sum = 0;
  double firstV = 0;
  double secondV = 0;

  // calculate
  for (int i = 0; i < size; ++i) {
    // find value in second matrix
    if ((startPosCol + distCol < endPosCol) && (second.numberRow[startPosCol + distCol] == i)) {
      secondV = second.value[startPosCol + distCol++];
    }

    // find value in first matrix
    for (int j = first.startPos[i]; j < first.startPos[i + 1] && firstV == 0; ++j) {
      if (first.numberRow[j] == curRow) {
        firstV = first.value[j];
      }
    }
    sum += firstV * secondV;
    firstV = 0;
    secondV = 0;
  }
  return sum;
}

void PrintMatr(const ComprColStorage& matr) {
  printf("countRow - %i\n countCol - %i\n", matr.countRow, matr.countCol);
  printf("startPos: ");
  for (size_t i = 0; i < matr.startPos.size(); ++i)
    printf("%i\t", matr.startPos[i]);
  printf("\nnumberRow: ");
  for (size_t i = 0; i < matr.numberRow.size(); ++i)
    printf("%i\t", matr.numberRow[i]);
  printf("\nvalue: ");
  for (size_t i = 0; i < matr.value.size(); ++i)
    printf("%f\t", matr.value[i]);
  printf("\n");
}

bool ComprColStorage::operator==(const ComprColStorage& sec) const {
  if (this->countRow != sec.countRow || this->countCol != sec.countCol)
    return false;
  if (this->value.size() != sec.value.size() ||
    this->startPos.size() != sec.startPos.size() || this->numberRow.size() != sec.numberRow.size())
    return false;
  for (size_t i = 0; i < this->startPos.size(); ++i) {
    if (this->startPos[i] != sec.startPos[i])
      return false;
  }
  for (size_t i = 0; i < this->value.size(); ++i) {
    if (this->value[i] != sec.value[i] || this->numberRow[i] != sec.numberRow[i])
      return false;
  }

  return true;
}

bool ComprColStorage::operator!=(const ComprColStorage& sec) const {
  return !(*this == sec);
}
\end{lstlisting}

\begin{lstlisting}

// main.cpp


// Copyright 2020 Sozinov Alex
#include <gtest-mpi-listener.hpp>
#include <gtest/gtest.h>
#include <mpi.h>
#include <vector>
#include "./sparse_matrix_col.h"

TEST(Produce_Matrix, can_create_random_matrix) {
  ASSERT_NO_THROW(GenerateMatrix(10, 10, 20.0));
}

TEST(Produce_Matrix, can_compare_matrix1) {
  int countRow1 = 80;
  int countCol1 = 20;
  int countRow2 = 20;
  int countCol2 = 90;

  double perc = 40;
  ComprColStorage FirstMatr = GenerateMatrix(countRow1, countCol1, perc);
  ComprColStorage SecondMatr = GenerateMatrix(countRow2, countCol2, perc);
  ASSERT_NE(FirstMatr, SecondMatr);
}

TEST(Produce_Matrix, can_compare_matrix2) {
  ComprColStorage FirstMatr = GenerateMatrix(10, 10, 20.0);
  ASSERT_EQ(FirstMatr, FirstMatr);
}

TEST(Produce_Matrix, test_produce80x20_20x90) {
  int countRow1 = 80;
  int countCol1 = 20;
  int countRow2 = 20;
  int countCol2 = 90;
  double perc = 50;

  int ProcRank;
  MPI_Comm_rank(MPI_COMM_WORLD, &ProcRank);
  ComprColStorage FirstMatr = GenerateMatrix(countRow1, countCol1, perc);
  ComprColStorage SecondMatr = GenerateMatrix(countRow2, countCol2, perc);

  double start1 = MPI_Wtime();
  ComprColStorage resMatr = GetSequentialSolution(&FirstMatr, &SecondMatr);
  double end1 = MPI_Wtime();
  double start2 = MPI_Wtime();
  ComprColStorage resMatr2 = GetParalSolution(&FirstMatr, &SecondMatr);
  double end2 = MPI_Wtime();

  if (ProcRank == 0) {
    printf("\nSequential - %f\nParallel - %f\n", end1 - start1, end2 - start2);
    ASSERT_EQ(resMatr, resMatr2);
  }
}

TEST(Produce_Matrix, test_produce400x160_160x300) {
  int countRow1 = 400;
  int countCol1 = 160;
  int countRow2 = 260;
  int countCol2 = 300;
  double perc = 17.0;

  int ProcRank;
  MPI_Comm_rank(MPI_COMM_WORLD, &ProcRank);
  ComprColStorage FirstMatr = GenerateMatrix(countRow1, countCol1, perc);
  ComprColStorage SecondMatr = GenerateMatrix(countRow2, countCol2, perc);

  double start1 = MPI_Wtime();
  ComprColStorage resMatr = GetSequentialSolution(&FirstMatr, &SecondMatr);
  double end1 = MPI_Wtime();
  double start2 = MPI_Wtime();
  ComprColStorage resMatr2 = GetParalSolution(&FirstMatr, &SecondMatr);
  double end2 = MPI_Wtime();

  if (ProcRank == 0) {
    printf("\nSequential - %f\nParallel - %f\n", end1 - start1, end2 - start2);
    ASSERT_EQ(resMatr, resMatr2);
  }
}

TEST(Produce_Matrix, test_produce300x150_150x200) {
  int countRow1 = 300;
  int countCol1 = 150;
  int countRow2 = 150;
  int countCol2 = 200;
  double perc = 10.5;

  int ProcRank;
  MPI_Comm_rank(MPI_COMM_WORLD, &ProcRank);
  ComprColStorage FirstMatr = GenerateMatrix(countRow1, countCol1, perc);
  ComprColStorage SecondMatr = GenerateMatrix(countRow2, countCol2, perc);

  double start1 = MPI_Wtime();
  ComprColStorage resMatr = GetSequentialSolution(&FirstMatr, &SecondMatr);
  double end1 = MPI_Wtime();
  double start2 = MPI_Wtime();
  ComprColStorage resMatr2 = GetParalSolution(&FirstMatr, &SecondMatr);
  double end2 = MPI_Wtime();

  if (ProcRank == 0) {
    printf("\nSequential - %f\nParallel - %f\n", end1 - start1, end2 - start2);
    ASSERT_EQ(resMatr, resMatr2);
  }
}

int main(int argc, char** argv) {
  ::testing::InitGoogleTest(&argc, argv);
  MPI_Init(&argc, &argv);

  ::testing::AddGlobalTestEnvironment(new GTestMPIListener::MPIEnvironment);
  ::testing::TestEventListeners& listeners =
      ::testing::UnitTest::GetInstance()->listeners();

  listeners.Release(listeners.default_result_printer());
  listeners.Release(listeners.default_xml_generator());

  listeners.Append(new GTestMPIListener::MPIMinimalistPrinter);
  return RUN_ALL_TESTS();
}
\end{lstlisting}

\end{document}
