\documentclass{report}

\usepackage[T2A]{fontenc}
\usepackage[utf8]{luainputenc}
\usepackage[english, russian]{babel}
\usepackage[pdftex]{hyperref}
\usepackage[14pt]{extsizes}
\usepackage{listings}
\usepackage{color}
\usepackage{geometry}
\usepackage{enumitem}
\usepackage{multirow}
\usepackage{graphicx}
\usepackage{indentfirst}

\geometry{a4paper,top=2cm,bottom=3cm,left=2cm,right=1.5cm}
\setlength{\parskip}{0.5cm}
\setlist{nolistsep, itemsep=0.3cm,parsep=0pt}

\lstset{language=C++,
    basicstyle=\footnotesize,
    keywordstyle=\color{blue}\ttfamily,
    stringstyle=\color{red}\ttfamily,
    commentstyle=\color{green}\ttfamily,
    morecomment=[l][\color{magenta}]{\#},
    tabsize=4,
    breaklines=true,
    breakatwhitespace=true,
    title=\lstname,
}

\makeatletter
\renewcommand\@biblabel[1]{#1.\hfil}
\makeatother

\begin{document}

\begin{titlepage}

\begin{center}
Министерство науки и высшего образования Российской Федерации
\end{center}

\begin{center}
Федеральное государственное автономное образовательное учреждение высшего образования \\
Национальный исследовательский Нижегородский государственный университет им. Н.И. Лобачевского
\end{center}

\begin{center}
Институт информационных технологий, математики и механики
\end{center}

\vspace{4em}

\begin{center}
\textbf{\LargeОтчет по лабораторной работе} \\
\end{center}
\begin{center}
\textbf{\Large«Поразрядная сортировка для целых чисел с простым слиянием»} \\
\end{center}

\vspace{4em}

\newbox{\lbox}
\savebox{\lbox}{\hbox{text}}
\newlength{\maxl}
\setlength{\maxl}{\wd\lbox}
\hfill\parbox{7cm}{
\hspace*{5cm}\hspace*{-5cm}\textbf{Выполнил:} \\ студент группы 381808-2 \\ Булычев А. С.\\
\\
\hspace*{5cm}\hspace*{-5cm}\textbf{Проверил:}\\ доцент кафедры МОСТ, \\ кандидат технических наук \\ Сысоев А. В.\\}
\vspace{\fill}

\begin{center} Нижний Новгород \\ 2020 \end{center}

\end{titlepage}

\setcounter{page}{2}

\newpage

% Введение
\section*{Введение}
\addcontentsline{toc}{section}{Введение}
Поразрядная сортировка исходно был предназначен для сортировки целых чисел, записанных цифрами. Но так как в памяти компьютеров любая информация записывается целыми числами, алгоритм пригоден для сортировки любых объектов, запись которых можно поделить на «разряды», содержащие сравнимые значения. Например, так сортировать можно не только числа, записанные в виде набора цифр, но и строки, являющиеся набором символов, и вообще произвольные значения в памяти, представленные в виде набора байт.
\par
Сравнение производится поразрядно: сначала сравниваются значения одного крайнего разряда, и элементы группируются по результатам этого сравнения, затем сравниваются значения следующего разряда, соседнего, и элементы либо упорядочиваются по результатам сравнения значений этого разряда внутри образованных на предыдущем проходе групп, либо переупорядочиваются в целом, но сохраняя относительный порядок, достигнутый при предыдущей сортировке. Затем аналогично делается для следующего разряда, и так до конца.
\newpage

% Постановка задачи
\section*{Постановка задачи}
\addcontentsline{toc}{section}{Постановка задачи}
В рамках лабораторной работы нужно реализовать поразрядную сортировку для целых чисел с простым слиянием.
\par
Требуется выполнить следующее:
\par
1.	Последовательный алгоритм сортировки(пузырьковая сортировка).
\par
2.	Параллельный алгоритм сортировки.
\par
3.	Провести вычислительные эксперименты.
\par
4.	Сравнить время работы параллельной и последовательной реализаций.
\newpage

% Метод решения
\section*{Метод решения}
\addcontentsline{toc}{section}{Метод решения}
Предположим, что элементы линейного списка L есть k-разрядные десятичные числа, разрядность максимального числа известна заранее. Обозначим d(j,n) - j-ю справа цифра числа n, которую можно выразить как:
\par
d(j,n) = [ n / 10j-1 ] % 10
\par
Пусть L0, L1,..., L9 - вспомогательные списки (карманы), вначале пустые. Поразрядная сортировка состоит из двух процессов, называемых распределение и сборка и выполняемых для j=1,2,...,k..
\par 
Фаза распределения разносит элементы L по карманам: элементы li списка L последовательно добавляются в списки Lm, где m = d(j, li). Таким образом получаем десять списков, в каждом из которых j-тые разряды чисел одинаковы и равны m.
\par 
Фаза сборки состоит в объединении списков L0, L1,..., L9 в общий список L = L0 => L1 => L2 => ... => L
\par 
После сортировки мы сливаем  2 вектора разных процессов в один и получаем отсортированный вектор 
\newpage

% Схема распараллеливания
\section*{Схема распараллеливания}
\addcontentsline{toc}{section}{Схема распараллеливания}
В начале все процессы коммуникатора MPI COMM WORLD вычисляют свою часть вектора (начальную позицию и длину).  
\par
Деление вектора происходит следующим образом: делим количество всех элементов вектора на количество процессов, остатки от деления распределяем между процессами.
\par
После деления вектора каждый процесс (кроме нулевого) отправляет нулевому свои размеры и свой локальный кусочек вектора. Это осуществляется с помощью MPI Send и MPI Resv. Нулевой процесс собирает все данные в один вектор для дальнейшего удобства вставки вектор.
\par
Далее все процессы получают на вход вектор целиком. Это осуществляется с помощью MPI Bcast. Каждый процесс сортирует свои кусочки и отправляет их нулевому процессу (MPI Send и MPI Resv). В конце нулевой процесс все полученные результаты складывает в нужные позиции результирующего вектора.
\par
В итоге результирующий вектор будет расположен в нулевом процессе.
\newpage

% Описание программной реализации
\section*{Описание программной реализации}
\addcontentsline{toc}{section}{Описание программной реализации}
В программе реализованы методы:
\begin{itemize}
\item[]•	std::vector<int> getRandomVector(int size) – генерирование рандомного вектора размерности size
\end{itemize}
\begin{itemize}
\item[]•	int pow10(int power) – возведение числа 10 в степень power
\end{itemize}
\begin{itemize}
\item[]•	int pow2(int pow) – возведение числа 2 в степень power
\end{itemize}
\begin{itemize}
\item[]•	std::vector<int> bubble sort Vector(std::vector<int> vec) – сортировка пузырьком
\end{itemize}
\begin{itemize}
\item[]•	bool vector is sort(std::vector<int> vec) – проверка на отсортированность вектора
\end{itemize}
\begin{itemize}
\item[]•	int get max power(std::vector<int> vec) –  кол-во разрядов самого большого числа в векторе
\end{itemize}
\begin{itemize}
\item[]•	std::vector<int> RadixSort(std::vector<int> vec, int size) – поразрядная сортировка
\end{itemize}
\begin{itemize}
\item[]•	std::vector<int> Merge(std::vector<int> vec1, std::vector<int> vec2) – слияние двух векторов
\end{itemize}
\begin{itemize}
\item[]•	std::vector<int> ParallSort(std::vector<int> vec) – параллельный метод поразрядной сортировки для целых чисел с простым слиянием
\end{itemize}
\newpage
%Подтверждение корректности
\section*{Подтверждение корректности}
\addcontentsline{toc}{section}{Подтверждение корректности}
Для подтверждения корректности в программе реализован набор тестов с использованием библиотеки для модульного тестирования Google C++ Testing Framework. Тесты проверяют корректную работу последовательного и параллельного алгоритмов
\newpage
% Результаты экспериментов
\section*{Результаты экспериментов}
\addcontentsline{toc}{section}{Результаты экспериментов}
Характеристики ПК:
\begin{itemize}
    \item Процессор: Intel® Core™ i7-8750H CPU @ 2.20GHz 2.20 GHz
\end{itemize}
\begin{itemize}
    \item Оперативная память: 16,00 ГБ
\end{itemize}
\begin{itemize}
    \item Операционная система: Windows 10
\end{itemize}
\begin{table}[!h]
\caption{Результаты вычислительных экспериментов}
\centering
\begin{tabular}{lllll}
Количество процессов & Послед. время & Парал. время\\
4  & 1,74967 & 0,03714\\
7 & 1,80691 & 0,04013\\
11  & 1,77067 & 0,03504\\
\end{tabular}
\end{table}
\par
В рамках эксперимента было вычислено время работы последовательного и параллельного алгоритмов сортировки. Размер исходного вектора - 10000.
\par
В ходе эксперимента было доказано, что алгоритм реализован эффективно.
\newpage

% Заключение
\section*{Заключение}
\addcontentsline{toc}{section}{Заключение}
В ходе работы были реализованы последовательный и параллельный алгоритмы сортировки. Из результатов вычислительных экспериментов видим, что параллельный алгоритм эффективнее последовательного, так как последовательный алгоритм обрабатывает весь вектор целиком.
\newpage

% Список литературы
\begin{thebibliography}{1}
\addcontentsline{toc}{section}{Список литературы}
\bibitem{Gergel}
Гергель В. П., Стронгин Р. Г. Основы параллельных вычислений для многопроцессорных вычислительных систем. – 2003.
\bibitem{parallel} [Электронный ресурс] //http://algolist.ru/sort/radix sort.php
\bibitem{parallel} [Электронный ресурс] //http://www.hpcc.unn.ru/mskurs/RUS/DOC/ppr04.pdf
\end{thebibliography}
\newpage

% Приложение
\section*{Приложение}
\addcontentsline{toc}{section}{Приложение}
\begin{lstlisting}
radix_sort.h:

// Copyright 2020 Bulychev Andrey
#ifndef MODULES_TASK_3_BULYCHEV_A_RADIX_SORT_RADIX_SORT_H_
#define MODULES_TASK_3_BULYCHEV_A_RADIX_SORT_RADIX_SORT_H_

#include <mpi.h>
#include <string>
#include <vector>

std::vector<int> getRandomVector(int size);
int pow10(int power);
std::vector<int> bubble_sort_Vector(std::vector<int> vec);
int pow2(int pow);
bool vector_is_sort(std::vector<int> vec);
int get_max_power(std::vector<int> vec);
std::vector<int> RadixSort(std::vector<int> vec, int size);
std::vector<int> Merge(std::vector<int> vec1, std::vector<int> vec2);
std::vector<int> ParallSort(std::vector<int> vec);

#endif  // MODULES_TASK_3_BULYCHEV_A_RADIX_SORT_RADIX_SORT_H_

\end{lstlisting}
\begin{lstlisting}
radix_sort.cpp:

// Copyright 2020 Bulychev Andrey
#include "../../../modules/task_3/bulychev_a_radix_sort/radix_sort.h"

#include <math.h>
#include <mpi.h>

#include <algorithm>
#include <ctime>
#include <iostream>
#include <random>
#include <string>
#include <vector>

std::vector<int> getRandomVector(int size) {
  std::vector<int> vec(size);
  std::mt19937 gen;
  gen.seed(static_cast<unsigned int>(time(0)));
  for (int i = 0; i < size; i++) vec[i] = gen() % 10000;
  return vec;
}

int pow10(int power) {
  int tmp = 1;
  for (int i = 0; i < power; i++) {
    tmp = tmp * 10;
  }
  return tmp;
}

std::vector<int> bubble_sort_Vector(std::vector<int> vec) {
  int j = 0;
  int tmp = 0;
  int lengh = vec.size();
  for (int i = 0; i < lengh; i++) {
    j = i;
    for (int k = i; k < lengh; k++) {
      if (vec[j] > vec[k]) {
        j = k;
      }
    }
    tmp = vec[i];
    vec[i] = vec[j];
    vec[j] = tmp;
  }
  return vec;
}

int pow2(int pow) {
  int res = 1;
  for (int i = 0; i < pow; ++i) res = res * 2;

  return res;
}

bool vector_is_sort(std::vector<int> vec) {
  int size = vec.size();
  for (int i = 0; i < size - 1; i++) {
    if (vec[i] > vec[i + 1]) {
      return false;
    }
  }
  return true;
}

int get_max_power(std::vector<int> vec) {
  int max_power = 0;
  int tmp = vec[0];
  int size = vec.size();
  for (int i = 0; i < size; i++) {
    if (tmp < vec[i]) {
      tmp = vec[i];
    }
  }
  if (tmp == 0) {
    return 1;
  }
  while (tmp != 0) {
    tmp = tmp / 10;
    max_power++;
  }
  return max_power;
}

std::vector<int> RadixSort(std::vector<int> vec, int size) {
  std::vector<std::vector<int>> main_vec(10);
  std::vector<int> my_vec = vec;
  int tmp = 0;
  int power = 1;
  int max_power;
  if (vector_is_sort(vec)) {
    return vec;
  }
  max_power = get_max_power(vec);
  while (power <= max_power) {
    for (int i = 0; i < size; i++) {
      tmp = (my_vec[i] % static_cast<unsigned int>(pow10(power))) /
            static_cast<unsigned int>(pow(10, power - 1));
      main_vec[tmp].push_back(my_vec[i]);
    }
    my_vec.clear();
    for (int i = 0; i < 10; i++) {
      if (!main_vec[i].empty()) {
        int size = main_vec[i].size();
        for (int j = 0; j < size; j++) {
          my_vec.push_back(main_vec[i][j]);
        }
        main_vec[i].clear();
      }
    }
    power++;
  }
  return my_vec;
}

std::vector<int> Merge(std::vector<int> vec1, std::vector<int> vec2) {
  int size1 = vec1.size();
  int size2 = vec2.size();
  std::vector<int> tmp(size1 + size2);
  int i = 0, j = 0;
  for (int k = 0; k < size1 + size2; k++) {
    if (i > size1 - 1) {
      int a = vec2[j];
      tmp[k] = a;
      j++;
    } else {
      if (j > size2 - 1) {
        int a = vec1[i];
        tmp[k] = a;
        i++;
      } else {
        if (vec1[i] < vec2[j]) {
          int a = vec1[i];
          tmp[k] = a;
          i++;
        } else {
          int b = vec2[j];
          tmp[k] = b;
          j++;
        }
      }
    }
  }
  return tmp;
}

std::vector<int> ParallSort(std::vector<int> src) {
  int mpi_rank, mpi_size;
  MPI_Comm_size(MPI_COMM_WORLD, &mpi_size);
  MPI_Comm_rank(MPI_COMM_WORLD, &mpi_rank);
  int length, rem;
  if (mpi_rank == 0) {
    length = src.size() / mpi_size;
    rem = src.size() % mpi_size;
  }
  MPI_Bcast(&length, 1, MPI_INT, 0, MPI_COMM_WORLD);
  std::vector<int> local_vec(length);
  MPI_Scatter(src.data(), length, MPI_INT, local_vec.data(), length, MPI_INT, 0,
              MPI_COMM_WORLD);
  if (mpi_rank == 0 && rem != 0) {
    local_vec.insert(local_vec.end(), src.end() - rem, src.end());
  }
  local_vec = RadixSort(local_vec, local_vec.size());
  int i = mpi_size;
  int num_op = 1;
  int num_op_length = length;
  int part;
  int loc_size = length;
  if (mpi_rank == 0) {
    loc_size = loc_size + rem;
  }
  while (i > 1) {
    if (i % 2 == 1) {
      if (mpi_rank == 0) {
        MPI_Status status1;
        std::vector<int> tmp_vec(num_op_length);
        part = pow2(num_op - 1) * (i - 1);
        MPI_Recv(tmp_vec.data(), num_op_length, MPI_INT, part, 1,
                 MPI_COMM_WORLD, &status1);
        local_vec = Merge(local_vec, tmp_vec);
        loc_size += num_op_length;
      }
      if (mpi_rank == pow2(num_op - 1) * (i - 1)) {
        MPI_Send(local_vec.data(), num_op_length, MPI_INT, 0, 1,
                 MPI_COMM_WORLD);
        return local_vec;
      }
    }
    if (mpi_rank % pow2(num_op) == 0) {
      part = mpi_rank + pow2(num_op - 1);
      std::vector<int> tmp_vec(num_op_length);
      MPI_Status status3;
      MPI_Recv(tmp_vec.data(), num_op_length, MPI_INT, part, 3, MPI_COMM_WORLD,
               &status3);
      local_vec = Merge(local_vec, tmp_vec);
      loc_size += num_op_length;
    }
    if (mpi_rank % pow2(num_op) != 0) {
      part = mpi_rank - pow2(num_op - 1);
      MPI_Send(local_vec.data(), num_op_length, MPI_INT, part, 3,
               MPI_COMM_WORLD);
      return local_vec;
    }
    num_op++;
    i = i / 2;
    num_op_length = num_op_length * 2;
  }
  return local_vec;
}

\end{lstlisting}
\begin{lstlisting}
main.cpp:

// Copyright 2020 Bulychev Andrey
#include <gtest/gtest.h>

#include <gtest-mpi-listener.hpp>
#include <vector>

#include "./radix_sort.h"

TEST(Parallel_Operations_MPI, Test_incorrect_size) {
  int rank;
  int size = -2;
  MPI_Comm_rank(MPI_COMM_WORLD, &rank);
  std::vector<int> vec;
  if (rank == 0) {
    ASSERT_ANY_THROW(getRandomVector(size));
  }
}

TEST(Parallel_Operation_MPI, Test_work_radix_sort) {
  int rank;
  MPI_Comm_rank(MPI_COMM_WORLD, &rank);
  std::vector<int> vec = getRandomVector(15);
  std::vector<int> result(vec.size());
  if (rank == 0) {
    int size = vec.size();
    vec = RadixSort(vec, vec.size());
    result = bubble_sort_Vector(vec);
    for (int i = 0; i < size; ++i) EXPECT_EQ(vec[i], result[i]);
  }
}

TEST(Parallel_Operation_MPI, Test_work_merge) {
  int rank;
  MPI_Comm_rank(MPI_COMM_WORLD, &rank);
  if (rank == 0) {
    std::vector<int> vec1{3, 4, 10, 17};
    std::vector<int> vec2{0, 6, 7, 46};
    std::vector<int> sum{0, 3, 4, 6, 7, 10, 17, 46};
    std::vector<int> result = Merge(vec1, vec2);
    for (int i = 0; i < 8; ++i) EXPECT_EQ(result[i], sum[i]);
  }
}

TEST(Parallel_Operations_MPI, Test_const_ParallelRadixSort) {
  int rank;
  MPI_Comm_rank(MPI_COMM_WORLD, &rank);
  std::vector<int> vec{47, 39, 118, 5,   24, 7,  27, 22,  35, 48,
                       70, 56, 107, 461, 19, 32, 95, 206, 67, 11};
  std::vector<int> result = ParallSort(vec);
  MPI_Barrier(MPI_COMM_WORLD);
  if (rank == 0) {
    std::vector<int> simple_sort = bubble_sort_Vector(vec);
    for (int i = 0; i < 10; ++i) EXPECT_EQ(result[i], simple_sort[i]);
  }
}

int main(int argc, char** argv) {
  ::testing::InitGoogleTest(&argc, argv);
  MPI_Init(&argc, &argv);
  ::testing::AddGlobalTestEnvironment(new GTestMPIListener::MPIEnvironment);
  ::testing::TestEventListeners& listeners =
      ::testing::UnitTest::GetInstance()->listeners();

  listeners.Release(listeners.default_result_printer());
  listeners.Release(listeners.default_xml_generator());

  listeners.Append(new GTestMPIListener::MPIMinimalistPrinter);
  return RUN_ALL_TESTS();
}

\end{lstlisting}
    \end{document}