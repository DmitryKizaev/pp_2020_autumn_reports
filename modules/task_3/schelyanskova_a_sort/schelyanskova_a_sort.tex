\documentclass{report}

\usepackage[T2A]{fontenc}
\usepackage[utf8]{luainputenc}
\usepackage[english, russian]{babel}
\usepackage[pdftex]{hyperref}
\usepackage[14pt]{extsizes}
\usepackage{listings}
\usepackage{color}
\usepackage{geometry}
\usepackage{enumitem}
\usepackage{multirow}
\usepackage{graphicx}
\usepackage{indentfirst}

\geometry{a4paper,top=2cm,bottom=3cm,left=2cm,right=1.5cm}
\setlength{\parskip}{0.5cm}
\setlist{nolistsep, itemsep=0.3cm,parsep=0pt}

\lstset{language=C++,
basicstyle=\footnotesize,
keywordstyle=\color{blue}\ttfamily,
stringstyle=\color{red}\ttfamily,
commentstyle=\color{green}\ttfamily,
morecomment=[l][\color{magenta}]{\#},
tabsize=4,
breaklines=true,
breakatwhitespace=true,
title=\lstname,
}

\makeatletter
\renewcommand\@biblabel[1]{#1.\hfil}
\makeatother

\begin{document}

\begin{titlepage}

\begin{center}
Министерство науки и высшего образования Российской Федерации
\end{center}

\begin{center}
Федеральное государственное автономное образовательное учреждение высшего образования \\
Национальный исследовательский Нижегородский государственный университет им. Н.И. Лобачевского
\end{center}

\begin{center}
Институт информационных технологий, математики и механики
\end{center}

\vspace{4em}

\begin{center}
\textbf{\LargeОтчет по лабораторной работе} \\
\end{center}
\begin{center}
\textbf{\Large«Поразрядная сортировка для вещественных чисел (тип double) с простым слиянием.»} \\
\end{center}

\vspace{4em}

\newbox{\lbox}
\savebox{\lbox}{\hbox{text}}
\newlength{\maxl}
\setlength{\maxl}{\wd\lbox}
\hfill\parbox{7cm}{
\hspace*{5cm}\hspace*{-5cm}\textbf{Выполнил:} \\ студент группы 381808-2 \\ Щелянскова Анастасия\\
\\
\hspace*{5cm}\hspace*{-5cm}\textbf{Проверил:}\\ доцент кафедры МОСТ, \\ кандидат технических наук \\ Сысоев А.В.\\
}
\vspace{\fill}

\begin{center} Нижний Новгород \\ 2020 \end{center}

\end{titlepage}

\setcounter{page}{2}

\newpage

% Постановка задачи
\section*{Постановка задачи}
\addcontentsline{toc}{section}{Постановка задачи}
В рамках данной лабораторной работы необходимо разработать программу, использующую поразрядную сортировку вещественных чисел и простое слияние для сортировки массива.Требуется реализовать как параллельную, так и линейную версию программы.Массив векторов генерируется в начале работы программы на нулевом процессе. По завершении выполнения программа выводит на консоль время работы линейной и параллельной версий программы в секундах (в линейной версии программы используется только поразрядная сортировка).
\par
На основе разработанной программы необходимо провести вычислительные эксперименты, сравнив время работы параллельного и последовательного алгоритмов на разных входных данных и разном количестве процессов, затем сделать вывод об эффективности параллельной версии.
\par
Также следует экспериментально проверить корректность разработанных версий алгоритма с помощью Google тестов и проиллюстрировать работу на реальных изображениях.
\newpage

% Описание алгоритма
\section*{Описание алгоритма}
\addcontentsline{toc}{section}{Описание алгоритма}
Каждый процесс сортирует полученную часть исходного массива с помощью поразрядной сортировки. Затем каждый процесс отправляет свою отсортированную часть массива всем
остальным процессам (себе в том числе). Далее циклично происходит слияние. Сначала слияние происходит на половине процессов, затем на
четверти и т.д. до того момента, пока не получим две отсортированные
части, которые отсортируем слиянием на нулевом процессе
\par

В результате работы на нулевом процессе оказывается отсортированный
массив. (Его уже нет смысла отсылать всем)
\par

Так как каждая сортировка слиянием требует выделения памяти на
дополнительный массив, будем выделять массив максимальной необходимой
длины для каждого процесса в самом начале работы программы, это уменьшит
время работы программы.
\newpage

% Схема распараллеливания
\section*{Схема распараллеливания}
\addcontentsline{toc}{section}{Схема распараллеливания}
Нулевой процесс поровну разделяет массив между всеми процессами (остаток
равномерно распределяется между процессами). Затем каждый процесс
линейно сортирует полученную часть массива с помощью поразрядной
сортировки. Далее данные собираются на всех процессах, после чего,
фактически по схеме биномиального дерева, где корнем является нулевой
процесс, происходит простое слияние. После каждой итерации значения
массива отправляются остальным процессам, количество процессов,
участвующих в итерации сокращается вдвое
\par

Для написания параллельной программы используется библиотека MS-MPI.
\newpage

% Описание программной реализации
\section*{Описание программной реализации}
\addcontentsline{toc}{section}{Описание программной реализации}
Программа реализована в radix\_sort\_double\_simple\_merge.cpp
\par

radix\_sort\_double\_simple\_merge.h
\par

Тесты main.cpp
\par

Функции:\par

getRandomVector - функция,
генерирующая вектор, заполненный случайными значениями.
\par

bubbleSort - функция сортировки пузырьком, нужна для того чтобы в тестах
удобнее было сортировать тестовые массивы.
\par

numberOfPositiveRadix - функция получения кол-ва положительных разрядов
в числе.
\par

numberOfNegativeRadix - функция получения кол-ва отрицательных разрядов
в числе.
\par

merge - функция слияния двух отсортированных массивов.
\par

getDigit - функция получения цифры из указанного разряда числа.
\par

radixSort - функция сортировки чисел по указанному разряду.
\par

linearRadixSort - функция линейной поразрядной сортировки.
\par

parallelRadixSort - функция параллельной сортировки.
\par

Массивы и переменные:
\par

num- количество процессов
\par

rank - номер текущего процесса
\par

size - размер массива
\par

start - переменная для счета времени
\par

end - переменная для счета времени
\par

gen - переменная для генерирования случайных значений массива
\par

vec - сортируемый массив
\par

vec\_left - куски сортируемого массива
\par

vec\_right - куски сортируемого массива
\par

tmp - вспомогательный массива для сортировки с несчетными процессами
\newpage

\section*{Подтверждение корректности}
\addcontentsline{toc}{section}{Подтверждение корректности}
Для подтверждения корректности в программе реализована линейная и
параллельная версия. Показано время, с которым работает линейная и
параллельная версии, также результаты можно вывести и сравнить
результаты сортировок (при больших размерах массивов это делать
нецелесообразно )\\
Кроме этого запоминается отсортированный массив на линейной версии и
сравнивается с сортировкой на параллельной, после это выдается сообщение
о том, что ошибки нет, или о том, что найдены не одинаковые элементы
\newpage
% Описание экспериментов

\section*{Описание экспериментов}
\addcontentsline{toc}{section}{Описание экспериментов}
Эксперименты проводились на 4-х ядерном компьютере с операционной системой – Windows 10, который имеет 8 логических процессоров. Результаты экспериментов приведены в таблице:
\begin{table}[!h]
\caption{Результаты вычислительных экспериментов}
\centering
\begin{tabular}{lllll}
Размер & Количество процессов & Паралл. время & Послед. время\\
10000 & 4 & 0.811 & 0.739 \\
10000 & 6 & 0.79 & 0.89 \\
10000 & 8 & 0.8 & 1.050 \\

\end{tabular}
\end{table}
\par
По результатам эксперимента, можно сделать вывод о том, что параллельная версия работает примерно одинаково с последовательной. 
\par
Kорректность работы программы могут результаты гугл-тестов.
\newpage

% Заключение
\section*{Заключение}
\addcontentsline{toc}{section}{Заключение}
Мы не получили большого ускорения за счет распараллеливания программы.\\
Это могло произойти из-за того, что сортировка слиянием работает
недостаточно быстро

\newpage

% Список литературы
\begin{thebibliography}{1}
\addcontentsline{toc}{section}{Список литературы}
\bibitem{Gergel}
Гергель В. П., Стронгин Р. Г. Основы параллельных вычислений для многопроцессорных вычислительных систем. – 2003.
\bibitem{parallel} Parallel: Лаборатория Параллельных информационных технологий НИВЦ МГУ [Электронный ресурс] // URL: \url {https://parallel.ru/vvv/mpi.html} (дата обращения: 28.11.2020)
\bibitem{Algolist} Algolist: Алгоритмы, методы, исходники [Электронный ресурс] // URL: \url {http://algolist.manual.ru/sort/radix_sort.php} (дата обращения: 28.11.2020)
\end{thebibliography}
\newpage

% Приложение
\section*{Приложение}
\addcontentsline{toc}{section}{Приложение}
\begin{lstlisting}
						// main.cpp

// Copyright 2020 Schelyanskova Anastasia
#ifndef MODULES_TASK_3_SCHELYANSKOVA_A_SORT_RADIX_SORT_DOUBLE_SIMPLE_MERGE_H_
#define MODULES_TASK_3_SCHELYANSKOVA_A_SORT_RADIX_SORT_DOUBLE_SIMPLE_MERGE_H_
#include <vector>


std::vector<double> getRandomVector(int size);

std::vector<double> bubbleSort(const std::vector<double>& vec);

int numberOfPositiveRadix(int num);

int numberOfNegativeRadix(double num);

int getDigit(double num, int discharge);

std::vector<double> merge(const std::vector<double>& vec1, const std::vector<double>& vec2);
 
std::vector<double> linearRadixSort(const std::vector<double>& vect);
 
std::vector<double> radixSort(const std::vector<double>& vect, int radix);

std::vector<double> parallelRadixSort(const std::vector<double>& vect);

#endif  // MODULES_TASK_3_SCHELYANSKOVA_A_SORT_RADIX_SORT_DOUBLE_SIMPLE_MERGE_H_
\end{lstlisting}
\begin{lstlisting}
						// radix_sort_double_simple_merge.h

// Copyright 2020 Schelyanskova Anastasia
#ifndef MODULES_TASK_3_SCHELYANSKOVA_A_SORT_RADIX_SORT_DOUBLE_SIMPLE_MERGE_H_
#define MODULES_TASK_3_SCHELYANSKOVA_A_SORT_RADIX_SORT_DOUBLE_SIMPLE_MERGE_H_
#include <vector>


std::vector<double> getRandomVector(int size);

std::vector<double> bubbleSort(const std::vector<double>& vec);

int numberOfPositiveRadix(int num);

int numberOfNegativeRadix(double num);

int getDigit(double num, int discharge);

std::vector<double> merge(const std::vector<double>& vec1, const std::vector<double>& vec2);

std::vector<double> linearRadixSort(const std::vector<double>& vect);

std::vector<double> radixSort(const std::vector<double>& vect, int radix);

std::vector<double> parallelRadixSort(const std::vector<double>& vect);

#endif  // MODULES_TASK_3_SCHELYANSKOVA_A_SORT_RADIX_SORT_DOUBLE_SIMPLE_MERGE_H_

\end{lstlisting}
\begin{lstlisting}
						// radix_sort_double_simple_merge.cpp

// Copyright 2020 Schelyanskova Anastasia

#include "../../../modules/task_3/schelyanskova_a_sort/radix_sort_double_simple_merge.h"

#include <mpi.h>

#include <ctime>
#include <iostream>
#include <random>
#include <vector>
#include <list>
#include <utility>
#include <string>
#include <algorithm>

using std::vector;
using std::cout;
using std::endl;
using std::list;
using std::to_string;

vector<double> getRandomVector(int size) {
    vector<double> vec(size);
    std::mt19937 gen;
    gen.seed(static_cast<unsigned char>(time(0)));
    for (int i = 0; i < size; i++) {
        vec[i] = gen() / 10000;
    }
    return vec;
}

vector<double> bubbleSort(const std::vector<double>& vec) {
    std::vector<double> sorted(vec);
    for (int i = 0; i < static_cast<int>(sorted.size()); i++) {
        for (int j = i; j < static_cast<int>(sorted.size()); j++) {
            if (sorted[i] > sorted[j]) {
                std::swap(sorted[i], sorted[j]);
            }
        }
    }
    return sorted;
}

int numberOfPositiveRadix(int num) {
    int num_of_radix = 0;
    while (num> 0) {
        num /= 10;
        num_of_radix++;
    }
    return num_of_radix;
}

int numberOfNegativeRadix(double num) {
    std::string str_num = to_string(num);
    if (str_num.find('.')) {
        return -(static_cast<int>(str_num.find('.')) - static_cast<int>(str_num.size())) - 1;
    } else {
        return 0;
    }
}

vector<double> merge(const vector<double>& vec_left, const vector<double>& vec_right) {
    std::vector<double> result((vec_left.size() + vec_right.size()));

    int i = 0, j = 0, k = 0;
    while (i < static_cast<int>(vec_left.size()) && j < static_cast<int>(vec_right.size())) {
        if (vec_left[i] < vec_right[j])
            result[k] = vec_left[i++];
        else
            result[k] = vec_right[j++];
        k++;
    }
    while (i < static_cast<int>(vec_left.size())) {
        result[k++] = vec_left[i++];
    }
    while (j < static_cast<int>(vec_right.size())) {
        result[k++] = vec_right[j++];
    }

    return result;
}

int getDigit(double num, int radix) {
    if (radix > 0) {
        double mask = pow(10, radix);
        double tmp = num / mask;
        return static_cast<int>(tmp) % 10;
    }
    return uint64_t(num * pow(10, -radix)) % 10;
}

std::vector<double> radixSort(const std::vector<double>& vect, int rad) {
    vector<double> res;
    vector<list<double>> radix(10);
    for (auto i = radix.begin(); i < radix.end(); i++) {
        (*i) = list<double>();
    }
    for (int i = 0; i < static_cast<int>(vect.size()); i++) {
        radix[getDigit(vect[i], rad)].push_back(vect[i]);
    }
    for (auto vect : radix) {
        for (auto element : vect) {
            res.push_back(element);
        }
    }
    return res;
}

std::vector<double> linearRadixSort(const std::vector<double>& vect) {
    int radixNegativeZero = 0;
    int maxRadixNegativeZero = numberOfNegativeRadix(vect[0]);
    for (auto element : vect) {
        radixNegativeZero = numberOfNegativeRadix(element);
        if (radixNegativeZero > maxRadixNegativeZero) {
            maxRadixNegativeZero = radixNegativeZero;
        }
    }
    double max = vect[0];
    for (int i = 1; i < static_cast<int>(vect.size()); i++) {
        if (vect[i] > max) {
            max = vect[i];
        }
    }
    int maxRadixPositiveZero = numberOfPositiveRadix(max);
    vector<double> result(vect);
    for (int i = -maxRadixNegativeZero; i <= maxRadixPositiveZero; i++) {
        result = radixSort(result, i);
    }
    return result;
}

std::vector<double> parallelRadixSort(const std::vector<double>& vect) {
    int size, rank;
    MPI_Comm_size(MPI_COMM_WORLD, &size);
    MPI_Comm_rank(MPI_COMM_WORLD, &rank);

    if ((size == 1) || (static_cast<int>(vect.size()) < size)) {
        if (rank == 0) {
            return linearRadixSort(vect);
        }
        return vector<double>();
    }

    vector<double> local_vect;

    int num_of_local_numbers = static_cast<int>(vect.size()) / size;
    int count = num_of_local_numbers;
    if (rank == 0) {
        int begin;
        local_vect = vector<double>(num_of_local_numbers);
        std::copy(vect.begin(), vect.begin() + num_of_local_numbers, local_vect.begin());

        for (int process_num = 1; process_num < size; process_num++) {
            begin = num_of_local_numbers * process_num;
            if (process_num == size - 1) {
                count = static_cast<int>(vect.size()) - begin;
            }
            MPI_Send(&vect[0] + num_of_local_numbers * process_num, count,
                MPI_DOUBLE, process_num, 0, MPI_COMM_WORLD);
        }
        count = num_of_local_numbers;
    } else {
        MPI_Status st;
        if (rank == size - 1) {
            count = static_cast<int>(vect.size()) - num_of_local_numbers * (size - 1);
        }
        local_vect = vector<double>(count);
        MPI_Recv(&local_vect[0], count, MPI_DOUBLE, 0, 0, MPI_COMM_WORLD, &st);
    }
    vector<double> result = linearRadixSort(local_vect);

    vector<int> active_process(size);
    vector<int> process_deltas_map(size);
    for (int i = 0; i < size - 1; i++) {
        active_process[i] = i;
        process_deltas_map[i] = num_of_local_numbers;
    }
    active_process[size - 1] = size - 1;
    process_deltas_map[size - 1] = static_cast<int>(vect.size())
        - num_of_local_numbers * (size - 1);

    while (static_cast<int>(active_process.size()) > 1) {
        int index_of_cur_process = std::find(active_process.begin(), active_process.end(), rank)
            - active_process.begin();

        if ((index_of_cur_process == static_cast<int>(active_process.size()) - 1) &&
            (static_cast<int>(active_process.size()) % 2)) {
            vector<int> tmp_active_proc;
            vector<int> tmp_process_deltas_map;
            for (int i = 0; i < static_cast<int>(active_process.size()) - 1; i += 2) {
                tmp_active_proc.push_back(active_process[i]);
                tmp_process_deltas_map.push_back(process_deltas_map[i]);
            }
            tmp_active_proc.push_back(active_process[index_of_cur_process]);
            tmp_process_deltas_map.push_back(process_deltas_map[index_of_cur_process]);

            for (int i = 0; i < static_cast<int>(tmp_active_proc.size()); i++) {
                if (tmp_active_proc[i] != rank) {
                    tmp_process_deltas_map[i] *= 2;
                }
            }
            active_process = tmp_active_proc;
            process_deltas_map = tmp_process_deltas_map;
            continue;
        }
        if (index_of_cur_process % 2 == 1) {
            MPI_Send(&result[0], count, MPI_DOUBLE, active_process[index_of_cur_process - 1], 0,
                MPI_COMM_WORLD);
            return vector<double>();
        } else {
            MPI_Status st;

            vector<double> tmp(process_deltas_map[index_of_cur_process + 1]);
            MPI_Recv(&tmp[0], process_deltas_map[index_of_cur_process + 1], MPI_DOUBLE,
                active_process[index_of_cur_process + 1], 0, MPI_COMM_WORLD, &st);

            result = merge(result, tmp);
            count = static_cast<int>(result.size());

            vector<int> tmp_active_proc;
            vector<int> tmp_process_deltas_map;
            for (int i = 0; i < static_cast<int>(active_process.size()) - 1; i += 2) {
                tmp_active_proc.push_back(active_process[i]);
                tmp_process_deltas_map.push_back(process_deltas_map[i] + process_deltas_map[i + 1]);
            }

            if ((static_cast<int>(active_process.size()) % 2)) {
                tmp_active_proc.push_back(active_process[static_cast<int>(active_process.size()) - 1]);
                tmp_process_deltas_map.push_back(process_deltas_map[static_cast<int>(active_process.size()) - 1]);
            }

            active_process = tmp_active_proc;
            process_deltas_map = tmp_process_deltas_map;
        }
    }
    if (rank == 0) {
        return result;
    }
    return vector<double>();
}
\end{lstlisting}
    \end{document}
