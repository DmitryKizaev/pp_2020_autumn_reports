\documentclass{report}

\usepackage[T2A]{fontenc}
\usepackage[utf8]{luainputenc}
\usepackage[english, russian]{babel}
\usepackage[pdftex]{hyperref}
\usepackage[14pt]{extsizes}
\usepackage{listings}
\usepackage{color}
\usepackage{geometry}
\usepackage{enumitem}
\usepackage{multirow}
\usepackage{graphicx}
\usepackage{indentfirst}
\usepackage{amsmath}

\geometry{a4paper,top=2cm,bottom=3cm,left=2cm,right=1.5cm}
\setlength{\parskip}{0.5cm}
\setlist{nolistsep, itemsep=0.3cm,parsep=0pt}

\lstset{language=C++,
		basicstyle=\footnotesize,
		keywordstyle=\color{blue}\ttfamily,
		stringstyle=\color{red}\ttfamily,
		commentstyle=\color{green}\ttfamily,
		morecomment=[l][\color{magenta}]{\#}, 
		tabsize=4,
		breaklines=true,
  		breakatwhitespace=true,
  		title=\lstname,       
}

\makeatletter
\renewcommand\@biblabel[1]{#1.\hfil}
\makeatother

\begin{document}

\begin{titlepage}

\begin{center}
Министерство науки и высшего образования Российской Федерации
\end{center}

\begin{center}
Федеральное государственное автономное образовательное учреждение высшего образования \\
Национальный исследовательский Нижегородский государственный университет им. Н.И. Лобачевского
\end{center}

\begin{center}
Институт информационных технологий, математики и механики
\end{center}

\vspace{4em}

\begin{center}
\textbf{\LargeОтчет по лабораторной работе} \\
\end{center}
\begin{center}
\textbf{\Large«Линейная фильтрация изображений (вертикальное разбиение). Ядро Гаусса 3x3»} \\
\end{center}

\vspace{4em}

\newbox{\lbox}
\savebox{\lbox}{\hbox{text}}
\newlength{\maxl}
\setlength{\maxl}{\wd\lbox}
\hfill\parbox{7cm}{
\hspace*{5cm}\hspace*{-5cm}\textbf{Выполнил:} \\ студент группы 381806-1 \\ Макаров А. А.\\
\\
\hspace*{5cm}\hspace*{-5cm}\textbf{Проверил:}\\ доцент кафедры МОСТ, \\ кандидат технических наук \\ Сысоев А. В.\\
}
\vspace{\fill}

\begin{center} Нижний Новгород \\ 2020 \end{center}

\end{titlepage}

\setcounter{page}{2}

% Содержание
\tableofcontents
\newpage

% Введение
\section*{Введение}
\addcontentsline{toc}{section}{Введение}
Линейная фильтрация с использованием ядра Гаусса - один из наиболее частых способов для размытия изображения. Размытие изображения применяется для избавления от шума, что может использоваться как само по себе, так и в качестве предварительной подготовки изображения для получения лучшего результата работы других алгоритмов обработки. Однако размытие изображений высокой четкости предполагает вплоть до десятков миллионов операций свертки на одном изображении, из чего вытекает естественная потребность в распараллеливании проводимых вычислений.
\par В данной лабораторной работе будет рассмотрена линейная фильтрация изображения с применением ядра Гаусса размером 3х3, причем фильтр будет применяться по столбцам. Будут реализованы последовательная и параллельная версии линейной фильтрации, что предполагает разбиение изображения на столбцы для обработки на различных процессах.
\newpage

% Постановка задачи
\section*{Постановка задачи}
\addcontentsline{toc}{section}{Постановка задачи}
В рамках лабораторной работы необходимо реализовать последовательную и параллельную реализации линейной фильтрации изображения с ядром Гаусса 3х3, с проходом по столбцам, проверить корректность работы алгоритмов, провести эксперименты для оценки эффективности параллельной реализации. По полученным результатам сделать выводы.
\par Для реализации параллельной версии необходимо использовать библиотеку межпроцессорного взаимодействия MPI. Для проверки корректности работы алгоритмов требуется использовать Google C++ Testing Framework.
\newpage

% Метод решения
\section*{Метод решения}
\addcontentsline{toc}{section}{Метод решения}
Линейная фильтрация изображений заключается в вычислении линейной комбинации значений яркости пикселов в окне фильтрации с коэффициентами матрицы весов фильтра, называемой также маской или ядром линейного фильтра. Результат линейной фильтрации для данного окна радиуса R (для данного центрального пиксела) описывается следующей формулой:
\begin{gather}
\mbox{Im}'[x,y] = \sum\limits_{i= -\textrm{R}}^{\textrm{R}}~\sum\limits_{j= -\textrm{R}}^{\textrm{R}} {\mbox{Im}[x+i,~y+j]\cdot \mbox{Mask}[x+i,~y+j]}
\end{gather}

\par Эта операция выполняется для каждого пиксела исходного изображения по столбцам. Линейная фильтрация с использованием ядра Гаусса производит свертку по маске, которая выглядит следующим образом:

\begin{gather}
Mask = \frac{1}{\sum\limits_{i = -R}^{R}\sum\limits_{j = -R}^{R}e^{-\frac{1}{2}\frac{i^2 + j^2}{\sigma^{2}}}} \begin{bmatrix} e^{-\frac{1}{2}\frac{(-R)^2 + (-R)^2}{\sigma^{2}}}& ...& e^{-\frac{1}{2}\frac{(-R)^2 + R^2}{\sigma^{2}}}\\ \vdots& \ddots& \vdots\\ e^{-\frac{1}{2}\frac{R^2 + (-R)^2}{\sigma^{2}}}& ...& e^{-\frac{1}{2}\frac{R^2 + R^2}{\sigma^{2}}}\\ \end{bmatrix}\\ \\ where\quad R = radius
\end{gather}

\par Возможны несколько вариантов свертки на границах изображения, но в данной лабораторной работе значения, выходящие за пределы изображения, принимались равными ближайшему внутреннему пикселу.
\newpage

% Схема распараллеливания
\section*{Схема распараллеливания}
\addcontentsline{toc}{section}{Схема распараллеливания}
\par Для распараллеливания линейной фильтрации необходимо передать каждому процесу равное количество столбцов изображения, которые он должен обработать, а также $R$ столбцов справа и слева от них, где $R$ - это радиус ядра, в нашем случае - $R = 1$. Корневой процесс расширяет исходное изображение, добавляя ему границу шириной R пикселов, а затем транспонирует, для того чтобы можно было передавать столбцы.
\par Далее корневой процесс рассылает одинаковое количество столбцов всем процессам (рассылаемые области пересекаются, т.к. нужно отправлять граничные с областью обработки элементы), кроме первого - он получает еще и остаток. Каждый процесс создает ядро Гаусса и с помощью него проводит свертку на своей части изображения и результат отправляет корневому процессу. Корневой процесс собирает результаты фильтрации и склеивает их в одно изображение. Полученное изображение и будет результатом работы.
\newpage


% Описание программной реализации
\section*{Описание программной реализации}
\addcontentsline{toc}{section}{Описание программной реализации}
\parАлгоритм последовательной линейной фильтрации с использованием ядра Гаусса вызывается с помощью функции:
\begin{lstlisting}
std::vector<unsigned int> gaussFilterSequential(const std::vector<unsigned int>& image, unsigned int w, unsigned int h, double sigma, unsigned int radius);
\end{lstlisting}
\par Входные параметры:
\begin{itemize}
\item image - изображение для обработки в виде одномерного вектора
\item w - ширина изображения
\item h - высота изображения
\item sigma - параметр для функции Гаусса, применяющейся при создании ядра
\item radius - радиус ядра Гаусса
\end{itemize}

\par Функция возвращает изображение после линейной фильтрации

\par Алгоритм параллельной линейной фильтрации с использованием ядра Гаусса вызывается с помощью функции:
\begin{lstlisting}
std::vector<unsigned int> gaussFilterParallel(const std::vector<unsigned int>& image, unsigned int w, unsigned int h, double sigma, unsigned int radius);
\end{lstlisting}
\par Входные параметры:
\begin{itemize}
\item image - изображение для обработки в виде одномерного вектора
\item w - ширина изображения
\item h - высота изображения
\item sigma - параметр для функции Гаусса, применяющейся при создании ядра
\item radius - радиус ядра Гаусса
\end{itemize}

\par Функция возвращает изображение после линейной фильтрации

\par Функции для предобработки изображения - расширения его границ и транспонирования:
\begin{lstlisting}
std::vector<unsigned int> expand(const std::vector<unsigned int>& image, unsigned int w, unsigned int h, unsigned int radius);
\end{lstlisting}
\par Входные параметры:
\begin{itemize}
\item image - изображение для расширения
\item w - ширина изображения
\item h - высота изображения
\item radius - толщина добавляемых границ
\end{itemize}

\par Функция возвращает расширенное изображение

\begin{lstlisting}
std::vector<unsigned int> transpose(const std::vector<unsigned int>& image, unsigned int w, unsigned int h);
\end{lstlisting}
\par Входные параметры:
\begin{itemize}
\item image - изображение для транспонирования
\item w - ширина изображения
\item h - высота изображения
\end{itemize}

\par Функция возвращает транспонированное изображение

\par Сама фильтрация Гаусса над расширенным изображением, с данным радиусом и сигмой для функции Гаусса вызывается в последовательном случае для всего изображения, а в параллельном - для полученных от корневого процесса частей с помощью функции:

\begin{lstlisting}
std::vector<unsigned int> gaussFilter(const std::vector<unsigned int>& exp_image, unsigned int w, unsigned int h, double sigma, unsigned int radius);
\end{lstlisting}
\par Входные параметры:
\begin{itemize}
\item exp-image - расширенное изображение для транспонирования
\item w - ширина изображения
\item h - высота изображения
\item sigma - параметр для функции Гаусса, применяющейся при создании ядра
\item radius - радиус ядра Гаусса
\end{itemize}

\par Функция возвращает изображение после применения линейной фильтрации с использованием ядра Гаусса

\par Внутри последней функции вызывается функция генерация ядра Гаусса с заданным радиусом и сигмой с помощью функции:

\begin{lstlisting}
std::vector<double> createGaussianKernel(double sigma, unsigned int radius);
\end{lstlisting}
\par Входные параметры:
\begin{itemize}
\item sigma - параметр для функции Гаусса, применяющейся при создании ядра
\item radius - радиус ядра Гаусса
\end{itemize}

\par Функция возвращает ядро Гаусса с заданной сигмой и радиусом

\newpage

% Подтверждение корректности
\section*{Подтверждение корректности}
\addcontentsline{toc}{section}{Подтверждение корректности}
Для подтверждения корректности в программе представлен набор тестов, разработанных с использованием Google C++ Testing Framework.
\par Набор представляет из себя тесты, которые генерируют случайные изображения разных размеров и проверяют корректность вычислений (сравнивается изображение, полученное с помощью последовательной фильтрации, с изображением, полученным с помощью параллельной фильтрации), а также эффективность (сравнивается время выполнения последовательной и параллельной реализации).
\par Успешное прохождение всех тестов доказывает корректность работы программного комплекса, а по времени выполнения можно судить об эффективности параллельной реализации
\newpage

% Результаты экспериментов
\section*{Результаты экспериментов}
\addcontentsline{toc}{section}{Результаты экспериментов}
Эксперименты для оценки эффективности параллельной реализации линейной фильтрации с использованием ядра Гаусса проводились на машине со следующей конфигурацией:

\begin{itemize}
\item Процессор: Intel Core i3-7020U, 2.30 GHz, количество ядер: 4;
\item Оперативная память: 8 ГБ;
\item ОС: Microsoft Windows 10 Home, версия 1909 сборка 18363.1256.
\end{itemize}

\par Для проведения экспериментов производилась фильтрация изображений размером 5000x5000. 
\par Результаты экспериментов представлены в таблице.

\begin{table}[!h]
\caption{Результаты вычислительных экспериментов}
\centering
\begin{tabular}{lllll}
Процессы & Последовательно & Параллельно & Ускорение  \\
1        & 1.51845          & 1.50244     & 1.01       \\
2        & 1.49318         & 1.7602     & 0.85       \\
3        & 1.6784         & 1.12344     & 1.49       \\
4        & 1.47147        & 1.04673     & 1.4       \\
5        & 1.48035         & 1.00177     & 1.48       \\
6        & 1.47162         & 1.00192     & 1.47     \\
7        & 1.48062         & 1.00296     & 1.48       \\
8        & 1.47564        & 0.97019    & 1.52
\end{tabular}
\end{table}

\par По замерам времени экспериментов, можно сделать вывод о том, что параллельный случай действительно работает быстрее, чем последовательный. 
\par Кроме того, можно сделать вывод, что при данной аппаратной конфигурации наиболее эффективным будет использование 4-5 процессов для параллельной поразрядной сортировки.
\par Также можно сделать вывод, что на данной машине наиболее эффективно будет использовать 3-4 процесса для параллельной реализации, дальнейшее увеличение числа процессов не дает заметного прироста производительности.
\newpage

% Заключение
\section*{Заключение}
\addcontentsline{toc}{section}{Заключение}
В результате лабораторной работы были разработаны последовательная и параллельная реализации линейной фильтрации изображения с использованием ядра Гаусса.
\par Задача данной лабораторной работы заключалась в параллельной реализации алгоритма, которая была бы более эффективная, чем последовательная. Задача была успешно выполнена, что следует из вышеприведенных результатов экспериментов.
\par Кроме того, были разработаны и доведены до успешного выполнения тесты, созданные для данного программного проекта с использованием Google C++ Testing Framework и необходимые для подтверждения корректности работы программы.
\newpage

% Список литературы
\begin{thebibliography}{1}
\addcontentsline{toc}{section}{Список литературы}
\bibitem{parallel} Профессиональный Wiki ресурс «Техническое зрение» [Электронный ресурс] // URL: \url {http://wiki.technicalvision.ru/} (дата обращения: 26.12.2020)
\bibitem{Algolist} Algolist: Алгоритмы, методы, исходники [Электронный ресурс] // URL: \url {https://en.wikipedia.org/wiki/Gaussian_filter} (дата обращения: 26.12.2020)
\end{thebibliography}
\newpage

% Приложение
\section*{Приложение}
\addcontentsline{toc}{section}{Приложение}
В данном разделе находится листинг всего кода, написанного в рамках лабораторной работы.
\begin{lstlisting}
// Copyright 2020 Makarov Alexander
#ifndef MODULES_TASK_3_MAKAROV_A_GAUSS_FILTER_COL_GAUSS_FILTER_COL_H_
#define MODULES_TASK_3_MAKAROV_A_GAUSS_FILTER_COL_GAUSS_FILTER_COL_H_

#include <vector>

std::vector<unsigned int> generate_image(unsigned int w, unsigned int h);

std::vector<double> createGaussianKernel(double sigma, unsigned int radius);

std::vector<unsigned int> gaussFilter(
                               const std::vector<unsigned int>& exp_image,
                               unsigned int w, unsigned int h, double sigma,
                               unsigned int radius);

std::vector<unsigned int> transpose(
                                   const std::vector<unsigned int>& image,
                                   unsigned int w, unsigned int h);

std::vector<unsigned int> expand(
                                   const std::vector<unsigned int>& image,
                                   unsigned int w, unsigned int h,
                                   unsigned int radius);

std::vector<unsigned int> gaussFilterSequential(
                                   const std::vector<unsigned int>& image,
                                   unsigned int w, unsigned int h,
                                   double sigma, unsigned int radius);
std::vector<unsigned int> gaussFilterParallel(
                                   const std::vector<unsigned int>& image,
                                   unsigned int w, unsigned int h,
                                   double sigma, unsigned int radius);

#endif  // MODULES_TASK_3_MAKAROV_A_GAUSS_FILTER_COL_GAUSS_FILTER_COL_H_
\end{lstlisting}
\begin{lstlisting}
// Copyright 2020 Makarov Alexander
#include <mpi.h>
#include <vector>
#include <random>
#include <iostream>
#include "../../../modules/task_3/makarov_a_gauss_filter_col/gauss_filter_col.h"


std::vector<unsigned int> generate_image(unsigned int w, unsigned int h) {
    unsigned int size = w * h;
    if (size == 0) return std::vector<unsigned int>();
    std::random_device rd;
    std::mt19937 gen(rd());
    size = w * h;
    std::vector<unsigned int> image(size);
    for (unsigned int i = 0; i < h; i++) {
        for (unsigned int j = 0; j < w; j++)
            image[i * w + j] = static_cast<unsigned int>(gen() % 256);
    }
    return image;
}

std::vector<double> createGaussianKernel(double sigma, unsigned int radius) {
    int m_radius = static_cast<int>(radius);
    unsigned int size = 2 * m_radius + 1;
    std::vector<double> result(size * size);
    double norm = 0;

    for (int i = -m_radius; i <= m_radius; i++)
        for (int j = -m_radius; j <= m_radius; j++) {
            int idx = (i + m_radius) * size + (j + m_radius);
            result[idx] = exp(-static_cast<double>(i * i + j * j) /
                         static_cast<double>(2. * sigma * sigma));
            norm += result[idx];
        }
    for (int i = -m_radius; i <= m_radius; i++)
        for (int j = -m_radius; j <= m_radius; j++) {
            int idx = (i + m_radius) * size + (j + m_radius);
            result[idx] /= norm;
        }
    return result;
}

std::vector<unsigned int> gaussFilter(
                               const std::vector<unsigned int>& exp_image,
                               unsigned int w, unsigned int h, double sigma,
                               unsigned int radius) {
    std::vector<double> gauss_kernel = createGaussianKernel(sigma, radius);
    int diam = static_cast<int>(2 * radius);
    int kern_radius = static_cast<int>(radius);
    int kern_size = kern_radius * 2 + 1;
    int result_h = static_cast<int>(h);
    int result_w = static_cast<int>(w);
    std::vector<unsigned int> result(result_h * result_w);
    for (int i = 0; i < result_h; i++)
        for (int j = 0; j < result_w; j++) {
            double conv = 0.;
            for (int k = -kern_radius; k <= kern_radius; k++)
                for (int t = -kern_radius; t <= kern_radius; t++)
                    conv += (static_cast<double>(exp_image[
                                   (kern_radius + i + k) * (result_w + diam) +
                                   kern_radius + j + t]) *
                                   gauss_kernel[
                                   (k + kern_radius) * kern_size + t +
                                   kern_radius]);
            result[i * result_w + j] = static_cast<unsigned int>(conv);
        }
    return result;
}

std::vector<unsigned int> transpose(
                                   const std::vector<unsigned int>& image,
                                   unsigned int w, unsigned int h) {
    std::vector<unsigned int> result(w * h);
    for (unsigned int i = 0; i < h; i++)
        for (unsigned int j = 0; j < w; j++)
            result[j * h + i] = image[i * w + j];
    return result;
}

std::vector<unsigned int> expand(
                                   const std::vector<unsigned int>& image,
                                   unsigned int w, unsigned int h,
                                   unsigned int radius) {
    unsigned int diam = radius * 2;
    std::vector<unsigned int> result((w + diam) * (h + diam));
    // transpose
    for (unsigned int i = 0; i < h; i++)
        for (unsigned int j = 0; j < w; j++)
            result[(i + radius) * (w + diam) + (j + radius)] =
                                                             image[i * w + j];
    // frames
    for (unsigned int i = 0; i < h; i++) {
        for (unsigned int j = 0; j < radius; j++) {
            result[(i + radius) * (w + diam) + j] = result[
                                          (i + radius) * (w + diam) + radius];
            result[(i + radius) * (w + diam) + (w + radius) + j] =
                         result[(i + radius) * (w + diam) + (w + radius - 1)];
        }
    }
    for (unsigned int i = 0; i < w + diam; i++) {
        for (unsigned int j = 0; j < radius; j++) {
            result[(w + diam) * j + i] = result[(w + diam) * radius + i];
            result[(h + radius + j) * (w + diam) + i] = result[
                                           (h + radius - 1) * (w + diam) + i];
        }
    }
    return result;
}

std::vector<unsigned int> gaussFilterSequential(
                                   const std::vector<unsigned int>& image,
                                   unsigned int w, unsigned int h,
                                   double sigma, unsigned int radius) {
    std::vector<unsigned int> expanded_img = expand(
                                                    transpose(image, w, h), h,
                                                    w, radius);
    std::vector<unsigned int> result = transpose(
                               gaussFilter(expanded_img, h, w, sigma, radius),
                               h, w);
    return result;
}

std::vector<unsigned int> gaussFilterParallel(
                                   const std::vector<unsigned int>& image,
                                   unsigned int w, unsigned int h,
                                   double sigma, unsigned int radius) {
    int rank;
    MPI_Comm_rank(MPI_COMM_WORLD, &rank);
    int proc_count;
    MPI_Comm_size(MPI_COMM_WORLD, &proc_count);
    if (proc_count <= 1) return gaussFilterSequential(image, w, h, sigma,
                                                      radius);
    unsigned int diam = radius * 2;
    unsigned int delta = w / (proc_count - 1);
    unsigned int remain = w % (proc_count - 1);

    unsigned int first_count = (h + diam) * (remain + delta + diam);
    unsigned int col_count = (h + diam) * (delta + diam);
    if (rank == 0) {
        std::vector<unsigned int> t_image = expand(transpose(image, w, h),
                                                       h, w, radius);
        MPI_Send(t_image.data(), first_count, MPI_INT, 1, 0,
                 MPI_COMM_WORLD);
        for (int i = 2; i < proc_count; i++) {
            // start_pos = (h + 2) * (1 + remain + (i - 1) * delta - 1);
            unsigned int start_pos = (h + diam) * (remain + (i - 1) * delta);
            MPI_Send(t_image.data() + start_pos, col_count, MPI_INT, i, 0,
                     MPI_COMM_WORLD);
        }
        MPI_Status status;
        std::vector<unsigned int> result(h * w);
        std::vector<unsigned int> tmp((remain + delta) * h);
        MPI_Recv(tmp.data(), (remain + delta) * h, MPI_INT, 1, MPI_ANY_TAG,
                 MPI_COMM_WORLD, &status);
        for (unsigned int i = 0; i < h; i++) {
            for (unsigned int j = 0; j < remain + delta; j++)
                result[i * w + j] = tmp[i * (remain + delta) + j];
        }
        for (int i = 2; i < proc_count; i++) {
            unsigned int start_col = remain + delta * (i - 1);
            MPI_Recv(tmp.data(), delta * h, MPI_INT, i, MPI_ANY_TAG,
                     MPI_COMM_WORLD, &status);
            for (unsigned int j = 0; j < h; j++) {
                for (unsigned int k = 0; k < delta; k++)
           // result[w * k + ((i - 1) * delta + remain + j)] = tmp[j * h + k];
                    result[j * w + start_col + k] = tmp[j * delta + k];
            }
        }
        return result;
    } else {
        std::vector<unsigned int> local_image;
        std::vector<unsigned int> local_result;
        MPI_Status status;
        if (rank == 1) {
            local_image.resize(first_count);
            MPI_Recv(local_image.data(), first_count, MPI_INT, 0,
                     MPI_ANY_TAG, MPI_COMM_WORLD, &status);
            local_result = transpose(gaussFilter(local_image, h,
                                     remain + delta, sigma, radius), h,
                                     remain + delta);
            MPI_Send(local_result.data(), (remain + delta) * h, MPI_INT, 0,
                     0, MPI_COMM_WORLD);
        } else {
            local_image.resize(col_count);
            MPI_Recv(local_image.data(), col_count, MPI_INT, 0, MPI_ANY_TAG,
                                                     MPI_COMM_WORLD, &status);
            local_result = transpose(
                            gaussFilter(local_image, h, delta, sigma, radius),
                            h, delta);
            MPI_Send(local_result.data(), delta * h, MPI_INT, 0, 0,
                                                       MPI_COMM_WORLD);
        }
        return local_result;
    }
}
\end{lstlisting}
\begin{lstlisting}
// Copyright 2020 Makarov Alexander
#include <gtest-mpi-listener.hpp>
#include <gtest/gtest.h>
#include <vector>
#include <iostream>
#include "./gauss_filter_col.h"

TEST(GaussFilter, 5x10_generation_test) {
    int rank;
    MPI_Comm_rank(MPI_COMM_WORLD, &rank);
    double sigma = 1.;
    unsigned int radius = 1;
    unsigned int w = 5;
    unsigned int h = 10;
    double start_time, end_time;
    std::vector<unsigned int> image;
    if (rank == 0) {
        image = generate_image(w, h);
    }
    if (rank == 0) start_time = MPI_Wtime();
    std::vector<unsigned int> result_par = gaussFilterParallel(image, w, h,
                                                                sigma, radius);
    if (rank == 0) end_time = MPI_Wtime();
    if (rank == 0) {
        double parr_time = end_time - start_time;
        start_time = MPI_Wtime();
        std::vector<unsigned int> result_seq = gaussFilterSequential(image,
                                                        w, h, sigma, radius);
        end_time = MPI_Wtime();
        double seq_time = end_time - start_time;
        std::cout << "Sequential time = " << seq_time << " s" << std::endl;
        std::cout << "Parallel time = " << parr_time << " s" << std::endl;
        ASSERT_EQ(result_par, result_seq);
    }
}

TEST(GaussFilter, 100x200_generation_test) {
    int rank;
    MPI_Comm_rank(MPI_COMM_WORLD, &rank);
    double sigma = 1.;
    unsigned int radius = 1;
    unsigned int w = 100;
    unsigned int h = 200;
    double start_time, end_time;
    std::vector<unsigned int> image;
    if (rank == 0) {
        image = generate_image(w, h);
    }
    if (rank == 0) start_time = MPI_Wtime();
    std::vector<unsigned int> result_par = gaussFilterParallel(image, w, h,
                                                                sigma, radius);
    if (rank == 0) end_time = MPI_Wtime();
    if (rank == 0) {
        double parr_time = end_time - start_time;
        start_time = MPI_Wtime();
        std::vector<unsigned int> result_seq = gaussFilterSequential(image,
                                                        w, h, sigma, radius);
        end_time = MPI_Wtime();
        double seq_time = end_time - start_time;
        std::cout << "Sequential time = " << seq_time << " s" << std::endl;
        std::cout << "Parallel time = " << parr_time << " s" << std::endl;
        ASSERT_EQ(result_par, result_seq);
    }
}

TEST(GaussFilter, 300x200_generation_test) {
    int rank;
    MPI_Comm_rank(MPI_COMM_WORLD, &rank);
    double sigma = 1.;
    unsigned int radius = 1;
    unsigned int w = 300;
    unsigned int h = 200;
    double start_time, end_time;
    std::vector<unsigned int> image;
    if (rank == 0) {
        image = generate_image(w, h);
    }
    if (rank == 0) start_time = MPI_Wtime();
    std::vector<unsigned int> result_par = gaussFilterParallel(image, w, h,
                                                                sigma, radius);
    if (rank == 0) end_time = MPI_Wtime();
    if (rank == 0) {
        double parr_time = end_time - start_time;
        start_time = MPI_Wtime();
        std::vector<unsigned int> result_seq = gaussFilterSequential(image,
                                                        w, h, sigma, radius);
        end_time = MPI_Wtime();
        double seq_time = end_time - start_time;
        std::cout << "Sequential time = " << seq_time << " s" << std::endl;
        std::cout << "Parallel time = " << parr_time << " s" << std::endl;
        ASSERT_EQ(result_par, result_seq);
    }
}

TEST(GaussFilter, 1000x2000_generation_test) {
    int rank;
    MPI_Comm_rank(MPI_COMM_WORLD, &rank);
    double sigma = 1.;
    unsigned int radius = 1;
    unsigned int w = 1000;
    unsigned int h = 2000;
    double start_time, end_time;
    std::vector<unsigned int> image;
    if (rank == 0) {
        image = generate_image(w, h);
    }
    if (rank == 0) start_time = MPI_Wtime();
    std::vector<unsigned int> result_par = gaussFilterParallel(image, w, h,
                                                                sigma, radius);
    if (rank == 0) end_time = MPI_Wtime();
    if (rank == 0) {
        double parr_time = end_time - start_time;
        start_time = MPI_Wtime();
        std::vector<unsigned int> result_seq = gaussFilterSequential(image,
                                                        w, h, sigma, radius);
        end_time = MPI_Wtime();
        double seq_time = end_time - start_time;
        std::cout << "Sequential time = " << seq_time << " s" << std::endl;
        std::cout << "Parallel time = " << parr_time << " s" << std::endl;
        ASSERT_EQ(result_par, result_seq);
    }
}

TEST(GaussFilter, 3000x2000_generation_test) {
    int rank;
    MPI_Comm_rank(MPI_COMM_WORLD, &rank);
    double sigma = 1.;
    unsigned int radius = 1;
    unsigned int w = 3000;
    unsigned int h = 2000;
    double start_time, end_time;
    std::vector<unsigned int> image;
    if (rank == 0) {
        image = generate_image(w, h);
    }
    if (rank == 0) start_time = MPI_Wtime();
    std::vector<unsigned int> result_par = gaussFilterParallel(image, w, h,
                                                                sigma, radius);
    if (rank == 0) end_time = MPI_Wtime();
    if (rank == 0) {
        double parr_time = end_time - start_time;
        start_time = MPI_Wtime();
        std::vector<unsigned int> result_seq = gaussFilterSequential(image,
                                                        w, h, sigma, radius);
        end_time = MPI_Wtime();
        double seq_time = end_time - start_time;
        std::cout << "Sequential time = " << seq_time << " s" << std::endl;
        std::cout << "Parallel time = " << parr_time << " s" << std::endl;
        ASSERT_EQ(result_par, result_seq);
    }
}

TEST(GaussFilter, 5000x5000_generation_test) {
    int rank;
    MPI_Comm_rank(MPI_COMM_WORLD, &rank);
    double sigma = 1.;
    unsigned int radius = 1;
    unsigned int w = 5000;
    unsigned int h = 5000;
    double start_time, end_time;
    std::vector<unsigned int> image;
    if (rank == 0) {
        image = generate_image(w, h);
    }
    if (rank == 0) start_time = MPI_Wtime();
    std::vector<unsigned int> result_par = gaussFilterParallel(image, w, h,
                                                                sigma, radius);
    if (rank == 0) end_time = MPI_Wtime();
    if (rank == 0) {
        double parr_time = end_time - start_time;
        start_time = MPI_Wtime();
        std::vector<unsigned int> result_seq = gaussFilterSequential(image,
                                                        w, h, sigma, radius);
        end_time = MPI_Wtime();
        double seq_time = end_time - start_time;
        std::cout << "Sequential time = " << seq_time << " s" << std::endl;
        std::cout << "Parallel time = " << parr_time << " s" << std::endl;
        ASSERT_EQ(result_par, result_seq);
    }
}

int main(int argc, char** argv) {
    ::testing::InitGoogleTest(&argc, argv);
    MPI_Init(&argc, &argv);

    ::testing::AddGlobalTestEnvironment(new GTestMPIListener::MPIEnvironment);
    ::testing::TestEventListeners& listeners =
        ::testing::UnitTest::GetInstance()->listeners();

    listeners.Release(listeners.default_result_printer());
    listeners.Release(listeners.default_xml_generator());

    listeners.Append(new GTestMPIListener::MPIMinimalistPrinter);
    return RUN_ALL_TESTS();
}

\end{lstlisting}

\end{document}
