\documentclass{report}

\usepackage[T2A]{fontenc}
\usepackage[utf8]{luainputenc}
\usepackage[english, russian]{babel}
\usepackage[pdftex]{hyperref}
\usepackage[14pt]{extsizes}
\usepackage{listings}
\usepackage{color}
\usepackage{geometry}
\usepackage{enumitem}
\usepackage{multirow}
\usepackage{graphicx}
\usepackage{indentfirst}

\geometry{a4paper,top=2cm,bottom=3cm,left=2cm,right=1.5cm}
\setlength{\parskip}{0.5cm}
\setlist{nolistsep, itemsep=0.3cm,parsep=0pt}

\lstset{language=C++,
basicstyle=\footnotesize,
keywordstyle=\color{blue}\ttfamily,
stringstyle=\color{red}\ttfamily,
commentstyle=\color{green}\ttfamily,
morecomment=[l][\color{magenta}]{\#},
tabsize=4,
breaklines=true,
breakatwhitespace=true,
title=\lstname,
}

\makeatletter
\renewcommand\@biblabel[1]{#1.\hfil}
\makeatother

\begin{document}

\begin{titlepage}

\begin{center}
Министерство науки и высшего образования Российской Федерации
\end{center}

\begin{center}
Федеральное государственное автономное образовательное учреждение высшего образования \\
Национальный исследовательский Нижегородский государственный университет им. Н.И. Лобачевского
\end{center}

\begin{center}
Институт информационных технологий, математики и механики
\end{center}

\vspace{4em}

\begin{center}
\textbf{\LargeОтчет по лабораторной работе} \\
\end{center}
\begin{center}
\textbf{\Large«Построение выпуклой оболочки для компонент бинарного изображения»} \\
\end{center}

\vspace{4em}

\newbox{\lbox}
\savebox{\lbox}{\hbox{text}}
\newlength{\maxl}
\setlength{\maxl}{\wd\lbox}
\hfill\parbox{7cm}{
\hspace*{5cm}\hspace*{-5cm}\textbf{Выполнил:} \\ студент группы 381806-3 \\ Таширев И.Э.\\
\\
\hspace*{5cm}\hspace*{-5cm}\textbf{Проверил:}\\ доцент кафедры МОСТ, \\ кандидат технических наук \\ Сысоев А.В.\\
}
\vspace{\fill}

\begin{center} Нижний Новгород \\ 2020 \end{center}

\end{titlepage}

\setcounter{page}{2}

\newpage

% Введение
\section*{Введение}
\addcontentsline{toc}{section}{Введение}
Выпуклой оболочкой конечного числа точек N называется наименьший, выпуклый многоугольник, содержащий все точки из N (некоторые из точек могут быть внутри многоугольника, некоторые на его сторонах, а некоторые будут его вершинами). Главной особенностью минимальной выпуклой оболочки множества точек A является то, что эта оболочка представляет собой выпуклый многоугольник, вершинами которого являются некоторые точки из A. У данной задачи существует множество применений, большая их часть связана с распознаванием образов, кластеризацией. В данной лабораторной работе представлен алгоритм посторения минимальной выпуклой оболчоки по методу Грэхема на языке C++ с использованием MPI.

\newpage

% Постановка задачи
\section*{Постановка задачи}
\addcontentsline{toc}{section}{Постановка задачи}
В данной работе необходимо реализовать последовательный и параллельный алгоритмы построения выпуклой оболочки по алгоритму Грэхема. 
\par
На основе разработанной программы необходимо провести вычислительные эксперименты, сравнив время работы параллельного и последовательного алгоритмов на разных входных данных и разном количестве процессов, затем сделать вывод об эффективности параллельной версии.
\par
Для реализации параллельной версии использовалась технология MPI. Для проверки корректности работы алгоритмов использовался Google C++ Testing Framework.
\newpage

% Описание алгоритма
\section*{Описание алгоритма}
\addcontentsline{toc}{section}{Описание алгоритма}
Данный алгоритм является трехшаговым. На первом шаге ищется любая точка в A, гарантированно входящая в минимальную выпуклую оболочку.Такой точкой будет точка с наименьшей x-координатой (самая левая точка в A). Эту точку (будем называть ее стартовой) перемещаем в начало списка, вся дальнейшая работа будет производиться с оставшимися точками. По некоторым соображениям, исходный массив точек A нами меняться не будет, для всех манипуляций с точками будем использовать косвенную адресацию: заведем список P, в котором будут хранится номера точек (их позиции в массиве A). Итак, первый шаг алгоритма заключается в том, чтобы первой точкой в P оказалась точка с наименьшей x-координатой.
\par 
Второй шаг в алгоритме Грэхема заключается в сортировке всех точек (кроме P[0]-ой), по степени их левизны относительно стартовой точки R=AP[0]. Будем говорить, что B<C, если точка С находится по левую сторону от вектора RB.
\par 
Для выпонения такого упорядочивания можно применять любой алгоритм сортировки, основанный на попарном сравнении элементов в данной лабораторной работе приведена быстрая сортировка. После этого можно сединить точки, однако они не будут представлять собой выпуклую оболочку.
\par
Третье действие заключается в срезании углов. Для этого нужно пройтись по всем вершинам и удалить те из них, в которых выполняется правый поворот угол в такой вершине оказывается больше развернутого. Заводим стек S и помещаем в него первые две вершины, они гарантированно входят в минимальную выпуклую оболочку. Затем просматриваем все остальные вершины, и отслеживаем направление поворота в них с точки зрения последних двух вершин в стеке S: если это направление отрицательно, то можно срезать угол удалением из стека последней вершины. Как только поворот оказывается положительным, срезание углов завершается, текущая вершина заносится в стек. В итоге в стеке S оказывается искомая последовательность вершин, причем в нужной ориентации, определяющая минимальную выпуклую оболочку заданного множества точек A.
\newpage

% Схема распараллеливания
\section*{Схема распараллеливания}
\addcontentsline{toc}{section}{Схема распараллеливания}
Первым шагом является распределение всего массива заданых точек на равные блоки. Каждому блоку присваевается свой процесс. Если массив точек не делится нацело на число процессов, то останутся точки которые не войдут в блок, их количество равно остатку от деления. Эти точки будут учитываться в корневом процессе.
\par
На втором шаге каждый процесс ищет самую крйнюю точку слева, после передаёт её в корневой процесс.
\par
На третьем шаге для каждого процесса выполняется проход Грэхема. Далее процессы отправляют найденную точку на корневой процесс, после этого в корневом еще раз запускается проход Грэхема. Таким образом, получаем точки, которые принадлежат выпуклой оболочке.  
\newpage

% Описание программной реализации
\section*{Описание программной реализации}
\addcontentsline{toc}{section}{Описание программной реализации}
В программе имеются следующие функции:
\begin{itemize}
\item Функция, выполняющая последовательный проход Грэхема: 
\end{itemize}
\begin{lstlisting}
	std::vector<pixel> greh_sequential(std::vector <pixel> pixels);
\end{lstlisting}
\begin{itemize}
\item Функция, выполняющая параллельный проход Грэхема: 
\end{itemize}
\begin{lstlisting}
	std::vector<pixel> greh_parallel(std::vector <pixel> pixels, int size);
\end{lstlisting}
\begin{itemize}
\item Функция создания случайных точек:
\end{itemize}
\begin{lstlisting}
	std::vector <pixel> get_random_image(int width, int height, int* size);
\end{lstlisting}
\begin{itemize}
\item Функция, находящая крайнюю правую точку: 
\end{itemize}
\begin{lstlisting}
	bool dist(pixel p1, pixel p2, pixel p3);
\end{lstlisting}
\begin{itemize}
\item Функция, находящая расстояние между точками: 
\end{itemize}
\begin{lstlisting}
	int64_t rot(pixel p1, pixel p2, pixel p3);
\end{lstlisting}
\newpage

% Описание экспериментов
\section*{Результаты экспериментов}
\addcontentsline{toc}{section}{Результаты экспериментов}
Контрольные эксперименты проводились на компьютере с операционной системой Windows 10 с процессором i5-3340 частотой 3.10GHz, оперативная память - 8192 Мегабайт:
\begin{table}[!h]
\caption{Результаты вычислительных экспериментов}
\centering
\caption{Результаты вычислительных экспериментов}
\centering
\begin{tabular}{llllll}
Количество процессов & Параллельная & Последовательная & Ускорение  \\
1        & 5.64834        & 5.70978      & 0.98                     \\
2        & 2.41635        & 5.11423      & 2.11                     \\
3        & 2.31384        & 4.93600      & 2.13                     \\
4        & 2.30855        & 5.12827      & 2,22                     \\
5        & 2.12874        & 4.18153      & 1.96                     \\
6        & 2.03917        & 3.84051      & 2.03                     \\
7        & 2.12585        & 4.33114      & 2.04                     \\
8        & 2.00752        & 4.22901      & 2.01      
\end{tabular}
\end{table}

\par
Результаты вычислений показывают эффективнось параллельных вычислений в этой задаче. Самый эффективный результат - 4 процесса.
\newpage

% Заключение
\section*{Заключение}
\addcontentsline{toc}{section}{Заключение}
В ходе выполнения работы было успешно реализовано последовательное и праллельное построение выпуклой оболочки по алгоритму грэхема. Эффективность параллельной реализации доказана экспериментально.
\newpage

% Список литературы
\begin{thebibliography}{1}
\addcontentsline{toc}{section}{Список литературы}
\bibitem{Gergel}
Гергель В. П., Стронгин Р. Г. Основы параллельных вычислений для многопроцессорных вычислительных систем. – 2003.
\bibitem{Algolist} Algolist: Алгоритмы, методы, исходники [Электронный ресурс] // URL: \url {http://algolist.manual.ru/sort/radix_sort.php} (дата обращения: 28.11.2020)
\end{thebibliography}
\newpage

% Приложение
\section*{Приложение}
\addcontentsline{toc}{section}{Приложение}
\begin{lstlisting}
						// main.cpp

// Copyright 2020 Tashirev Ivan
#include <gtest-mpi-listener.hpp>
#include <gtest/gtest.h>
#include <iostream>
#include <algorithm>
#include <vector>
#include "./tashirev_i_graham.h"
using namespace std;

TEST(Parallel_Operations_MPI, Test1) {
    int proc_size, proc;
    MPI_Comm_rank(MPI_COMM_WORLD, &proc);
    MPI_Comm_size(MPI_COMM_WORLD, &proc_size);
    std::vector<pixel> input_image, p_out_points, s_out_points;
    int side = 10;
    int points_in_image = 0;

    if (proc == 0) {
        input_image = get_random_image(side, side, &points_in_image);
    }
    double start = MPI_Wtime();
    MPI_Bcast(&points_in_image, 1, MPI_INT, 0, MPI_COMM_WORLD);
    p_out_points = greh_parallel(input_image, points_in_image);
    double end = MPI_Wtime();

    if (proc == 0) {
        cout << end - start << endl;
        start = MPI_Wtime();
        s_out_points = greh_sequential(input_image);
        end = MPI_Wtime();
        cout << end - start << endl;
        ASSERT_EQ(p_out_points, s_out_points);
    }
    MPI_Barrier(MPI_COMM_WORLD);
}

TEST(Parallel_Operations_MPI, Test2) {
    int proc_size, proc;
    MPI_Comm_rank(MPI_COMM_WORLD, &proc);
    MPI_Comm_size(MPI_COMM_WORLD, &proc_size);
    std::vector<pixel> input_image, p_out_points, s_out_points;
    int side = 50;
    int points_in_image = 0;

    if (proc == 0) {
        input_image = get_random_image(side, side, &points_in_image);
    }

    MPI_Bcast(&points_in_image, 1, MPI_INT, 0, MPI_COMM_WORLD);
    p_out_points = greh_parallel(input_image, points_in_image);

    if (proc == 0) {
        s_out_points = greh_sequential(input_image);
        ASSERT_EQ(p_out_points, s_out_points);
    }
    MPI_Barrier(MPI_COMM_WORLD);
}

TEST(Parallel_Operations_MPI, Test3) {
    int proc_size, proc;
    MPI_Comm_rank(MPI_COMM_WORLD, &proc);
    MPI_Comm_size(MPI_COMM_WORLD, &proc_size);
    std::vector<pixel> input_image, p_out_points, s_out_points;
    int side = 100;
    int points_in_image = 0;

    if (proc == 0) {
        input_image = get_random_image(side, side, &points_in_image);
    }

    MPI_Bcast(&points_in_image, 1, MPI_INT, 0, MPI_COMM_WORLD);
    p_out_points = greh_parallel(input_image, points_in_image);

    if (proc == 0) {
        s_out_points = greh_sequential(input_image);
        ASSERT_EQ(p_out_points, s_out_points);
    }
    MPI_Barrier(MPI_COMM_WORLD);
}

TEST(Parallel_Operations_MPI, Test4) {
    int proc_size, proc;
    MPI_Comm_rank(MPI_COMM_WORLD, &proc);
    MPI_Comm_size(MPI_COMM_WORLD, &proc_size);
    std::vector<pixel> input_image, p_out_points, s_out_points;
    int side = 500;
    int points_in_image = 0;

    if (proc == 0) {
        input_image = get_random_image(side, side, &points_in_image);
    }

    MPI_Bcast(&points_in_image, 1, MPI_INT, 0, MPI_COMM_WORLD);
    p_out_points = greh_parallel(input_image, points_in_image);

    if (proc == 0) {
        s_out_points = greh_sequential(input_image);
        ASSERT_EQ(p_out_points, s_out_points);
    }
    MPI_Barrier(MPI_COMM_WORLD);
}

TEST(Parallel_Operations_MPI, Test5) {
    int proc_size, proc;
    MPI_Comm_rank(MPI_COMM_WORLD, &proc);
    MPI_Comm_size(MPI_COMM_WORLD, &proc_size);
    std::vector<pixel> input_image, p_out_points, s_out_points;
    int side = 1000;
    int points_in_image = 0;

    if (proc == 0) {
        input_image = get_random_image(side, side, &points_in_image);
    }

    MPI_Bcast(&points_in_image, 1, MPI_INT, 0, MPI_COMM_WORLD);
    p_out_points = greh_parallel(input_image, points_in_image);

    if (proc == 0) {
        s_out_points = greh_sequential(input_image);
        ASSERT_EQ(p_out_points, s_out_points);
    }
    MPI_Barrier(MPI_COMM_WORLD);
}

int main(int argc, char** argv) {
    ::testing::InitGoogleTest(&argc, argv);
    MPI_Init(&argc, &argv);

    ::testing::AddGlobalTestEnvironment(new GTestMPIListener::MPIEnvironment);
    ::testing::TestEventListeners& listeners =
        ::testing::UnitTest::GetInstance()->listeners();

    listeners.Release(listeners.default_result_printer());
    listeners.Release(listeners.default_xml_generator());

    listeners.Append(new GTestMPIListener::MPIMinimalistPrinter);
    return RUN_ALL_TESTS();
}

						// tashirev_i_graham.h

// Copyright 2020 Tashirev Ivan
#ifndef MODULES_TASK_3_TASHIREV_I_GRAHAM_TASHIREV_I_GRAHAM_H_
#define MODULES_TASK_3_TASHIREV_I_GRAHAM_TASHIREV_I_GRAHAM_H_

#include <vector>

struct pixel {
    pixel() = default;
    pixel(int64_t x, int64_t y) : x(x), y(y) {}

    friend bool operator!=(pixel p1, pixel p2) {
        return !operator==(p1, p2);
    }
    friend bool operator==(pixel p1, pixel p2) {
        return p2.x == p1.x && p2.y == p1.y;
    }

    int64_t x;
    int64_t y;
};

int64_t rot(pixel a, pixel b, pixel c);
bool dist(pixel a, pixel b, pixel c);

std::vector<pixel> get_random_image(int width, int height, int* size);

std::vector<pixel> greh_sequential(std::vector <pixel> pixels);
std::vector<pixel> greh_parallel(std::vector <pixel> pixels, int size);

#endif  // MODULES_TASK_3_TASHIREV_I_GRAHAM_TASHIREV_I_GRAHAM_H_

						// convex_hull_for_binary.cpp

/// Copyright 2020 Tashirev Ivan
#include <mpi.h>
#include <string>
#include <random>
#include <ctime>
#include <algorithm>
#include <iostream>
#include <utility>
#include <vector>
#include <cstring>
#include "../../../modules/task_3/tashirev_i_graham/tashirev_i_graham.h"

std::vector<pixel> greh_sequential(std::vector <pixel> pixels) {
    std::vector <int> point_pos({0, 1});
    pixel f;
    int size = pixels.size();
    int it, pos = 0;
    f = pixels[0];
    for (it = 1; it < size; ++it)
        if (f.x > pixels[it].x || (f.x == pixels[it].x && f.y > pixels[it].y)) {
            pos = it;
            f = pixels[it];
        }

    std::swap(pixels[0], pixels[pos]);
    sort(pixels.begin() + 1, pixels.end(), [&](pixel a, pixel b) {
        if (rot(f, a, b) > 0)
            return true;
        else if (rot(f, a, b) < 0)
            return false;
        else
            return dist(f, b, a);
    });

    it = 2;
    while (it < size) {
        if (point_pos.size() < 2) {
            point_pos.push_back(it);
            ++it;
        }
        if (rot(pixels[point_pos[point_pos.size() - 2]], pixels[point_pos[point_pos.size() - 1]], pixels[it]) <= 0) {
            point_pos.pop_back();
        } else {
            point_pos.push_back(it);
            ++it;
        }
    }

    std::vector<pixel> hull_points(point_pos.size());
    for (size_t i = 0; i < point_pos.size(); i++) {
        hull_points[i] = pixels[point_pos[i]];
    }

    return hull_points;
}

std::vector<pixel> greh_parallel(std::vector <pixel> pixels, int size) {
    MPI_Datatype MPI_POINT;
    MPI_Type_contiguous(sizeof(int64_t) / sizeof(int) * 2, MPI_INT, &MPI_POINT);
    MPI_Type_commit(&MPI_POINT);

    int proc_s, proc_n;
    MPI_Comm_size(MPI_COMM_WORLD, &proc_s);
    MPI_Comm_rank(MPI_COMM_WORLD, &proc_n);

    int* sendc = new int[proc_s];
    int* sendd = new int[proc_s];

    if (proc_n == 0) {
        int distance = 0;
        for (int i = 0; i < proc_s; i++) {
            sendd[i] = distance;
            if (i < size % proc_s) {
                sendc[i] = size / proc_s + 1;
                distance += size / proc_s + 1;
            } else {
                sendc[i] = size / proc_s;
                distance += size / proc_s;
            }
        }
    }

    int l_image_size;
    if (proc_n < size % proc_s)
        l_image_size = size / proc_s + 1;
    else
        l_image_size = size / proc_s;

    std::vector<pixel> l_image(l_image_size);

    MPI_Scatterv(pixels.data(), sendc, sendd, MPI_POINT,
        l_image.data(), l_image_size, MPI_POINT, 0, MPI_COMM_WORLD);

    l_image = greh_sequential(l_image);

    l_image_size = l_image.size();

    int counter = proc_s, swipe = 1;

    while (counter > 1) {
        counter = counter / 2 + counter % 2;
        if ((proc_n - swipe) % (2 * swipe) == 0) {
            MPI_Send(&l_image_size, 1, MPI_INT, proc_n - swipe, 0, MPI_COMM_WORLD);

            MPI_Send(l_image.data(), l_image_size, MPI_POINT, proc_n - swipe, 0, MPI_COMM_WORLD);
        }
        if ((proc_n % (2 * swipe) == 0) && (proc_s - proc_n > swipe)) {
            MPI_Status stat;
            int r_size;

            MPI_Recv(&r_size, 1, MPI_INT, proc_n + swipe, MPI_ANY_TAG, MPI_COMM_WORLD, &stat);

            std::vector<pixel> r_image(r_size);

            MPI_Recv(r_image.data(), r_size, MPI_POINT, proc_n + swipe, MPI_ANY_TAG,
                MPI_COMM_WORLD, &stat);

            l_image.insert(l_image.end(), r_image.begin(), r_image.end());

            l_image = greh_sequential(l_image);

            l_image_size = l_image.size();
        }
        swipe = 2 * swipe;
    }
    return l_image;
}

std::vector <pixel> get_random_image(int width, int height, int* size) {
    std::mt19937 gen;
    gen.seed(static_cast<unsigned int>(time(0)));

    std::vector <pixel> out_image;
    for (int i = 0; i < height; i++) {
        for (int j = 0; j < width; j++) {
            if (gen() % 4 == 0) {
                out_image.push_back(pixel(j, i));
                ++(*size);
            }
        }
    }
    return out_image;
}

bool dist(pixel p1, pixel p2, pixel p3) {
    return (p1.x - p2.x) * (p1.x - p2.x) + (p1.y - p2.y) * (p1.y - p2.y) >=
    (p1.x - p3.x) * (p1.x - p3.x) + (p1.y - p3.y) * (p1.y - p3.y);
}

int64_t rot(pixel p1, pixel p2, pixel p3) {
    return (p2.x - p1.x) * (p3.y - p1.y) - (p2.y - p1.y) * (p3.x - p1.x);
}

\end{lstlisting}
    \end{document}
