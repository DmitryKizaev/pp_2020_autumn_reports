\documentclass{report}

\usepackage[T2A]{fontenc}
\usepackage[utf8]{luainputenc}
\usepackage[english, russian]{babel}
\usepackage[pdftex]{hyperref}
\usepackage{listings}
\usepackage{color}
\usepackage{geometry}
\usepackage{enumitem}
\usepackage{multirow}
\usepackage{graphicx}
\usepackage{indentfirst}
\usepackage{hyphenat}

\geometry{a4paper,top=2cm,bottom=3cm,left=2cm,right=1.5cm}
\setlength{\parskip}{0.5cm}
\setlist{nolistsep, itemsep=0.3cm,parsep=0pt}

\lstset{language=C++,
		basicstyle=\footnotesize,
		keywordstyle=\color{blue}\ttfamily,
		stringstyle=\color{red}\ttfamily,
		commentstyle=\color{green}\ttfamily,
		morecomment=[l][\color{magenta}]{\#}, 
		tabsize=4,
		breaklines=true,
  		breakatwhitespace=true,
  		title=\lstname,       
}

\makeatletter
\renewcommand\@biblabel[1]{#1.\hfil}
\makeatother

\begin{document}

\begin{titlepage}

\begin{center}
Министерство науки и высшего образования Российской Федерации
\end{center}

\begin{center}
Федеральное государственное автономное образовательное учреждение высшего образования \\
Национальный исследовательский Нижегородский государственный университет им. Н.И. Лобачевского
\end{center}

\begin{center}
Институт информационных технологий, математики и механики
\end{center}

\vspace{4em}

\begin{center}
\textbf{\LargeОтчет по лабораторной работе} \\
\end{center}
\begin{center}
\textbf{\LargeУмножение плотных матриц. Элементы типа double. Алгоритм Штрассена.} \\
\end{center}

\vspace{4em}

\newbox{\lbox}
\savebox{\lbox}{\hbox{text}}
\newlength{\maxl}
\setlength{\maxl}{\wd\lbox}
\hfill\parbox{7cm}{
\hspace*{5cm}\hspace*{-5cm}\textbf{Выполнил:} \\ студент группы 381806-1 \\ Стрельцова Я. Д.\\
\hspace*{5cm} \\
\hspace*{5cm}\hspace*{-5cm}\textbf{Проверил:}\\ доцент кафедры МОСТ, \\ кандидат технических наук \\ Сысоев А. В.
}
\vspace{\fill}

\begin{center} Нижний Новгород \\ 2020 \end{center}

\end{titlepage}

\setcounter{page}{2}

% Содержание
\tableofcontents
\newpage

% Введение
\section*{Введение}
\addcontentsline{toc}{section}{Введение}
Алгоритм Штрассена предназначен для быстрого умножения матриц. Он был разработан Фолькером Штрассеном в 1969 году и является обобщением метода умножения Карацубы на матрицы.

В отличие от традиционного алгоритма умножения матриц (по формуле ${\displaystyle c_{i,k}=\sum a_{i,j}b_{j,k}}$), работающего за время ${\displaystyle \Theta (n^{\log _{2}8})=\Theta (n^{3})}$), алгоритм Штрассена умножает матрицы за время ${\displaystyle \Theta (n^{\log _{2}7})=O(n^{2.81})}$, что даёт выигрыш на больших плотных матрицах начиная, примерно, от 64×64.

Несмотря на то, что алгоритм Штрассена является асимптотически не самым быстрым из существующих алгоритмов быстрого умножения матриц, он проще программируется и эффективнее при умножении матриц относительно малого размера, поэтому именно он чаще используется на практике. 

Применяя метод Штрассена для умножения чисел с плавающей точкой, нужно иметьв виду, что тип данных float не образует кольца, и потому алгоритм не обязан вычислятьправильные значения. На практике это может вылиться в накопление погрешности.
\newpage

% Постановка задачи
\section*{Постановка задачи}
\addcontentsline{toc}{section}{Постановка задачи}
С ипользованием средств MPI необходимо разработать параллельную и последовательную реализации алгоритма Штрассена, которые будут выполнять умножение плотных матриц с элементами типа double. Также необходимо провести сравнение времени работы, реализованных алгоритмов, и подтвердить корректность работы, используя Google Testing Framework.
\newpage

% Метод решения
\section*{Метод решения}
\addcontentsline{toc}{section}{Метод решения}
Пусть A, B — две квадратные матрицы над кольцом R. Матрица C получается по формуле:

    ${\displaystyle \mathbf {C} =\mathbf {A} \mathbf {B} \qquad \mathbf {A} ,\mathbf {B} ,\mathbf {C} \in R^{2^{n}\times 2^{n}}}$

Алгоритм Штрассена для вещественных чисел, применяемый для квадратных матриц, заключается в следующем:
\begin{itemize}
    \item Проверяем размер матрицы, если n < 64, то используем обычный метод.
    \item Дополняем исходные матрицы нулевыми строками и столбцами, если размер умножаемых матриц n не является натуральной степенью двойки. 
    \item Разделяем A и B на четыре (n/2)×(n/2) подматрицы.
    \item Сформировываем множители с помощью сложения и вычитания подматриц.
    \item Рекурсивно выполняем 7 умножений подматриц.
    \item Получаем результирующую матрицу C с помощью сложения и вычитания подматриц.
\end{itemize}
\newpage

% Схема распараллеливания
\section*{Схема распараллеливания}
\addcontentsline{toc}{section}{Схема распараллеливания}
Чтобы реализовать параллельную реализацию алгоритма необходимо:
\begin{itemize}
    \item Передать размер матриц n из корневого процесса всем процессам группы, включая себя;
    \item Проверить размер матриц, если n < 64, то вызвать функцию обычного последовательного умножения матриц;
    \item Если размер умножаемых матриц n не является натуральной степенью двойки, дополнить исходные матрицы дополнительными нулевыми строками и столбцами;
    \item Разделить A и B на четыре (n/2)×(n/2) подматрицы каждую: а11, а12, а21, а22, b11, b12, b21, b22;
    \item Сформировать множители с помощью параллельного сложения и вычитания подматриц;
    \item Рекурсивно вызвать 7 умножений подматриц: Р1, Р2, Р3, Р4, Р5, Р6, Р7;
    \item Получить результирующую матрицу C с помощью параллельного сложения и вычитания подматриц:
    
        ${\displaystyle \mathbf {C}_{1,1}=\mathbf {P}_{1}+\mathbf {P}_{4}-\mathbf {P}_{5}+\mathbf {P}_{7}}$
    
        ${\displaystyle \mathbf {C} _{1,2}=\mathbf {P} _{3}+\mathbf {P} _{5}}$
    
        ${\displaystyle \mathbf {C} _{2,1}=\mathbf {P} _{2}+\mathbf {P} _{4}}$
    
        ${\displaystyle \mathbf {C} _{2,2}=\mathbf {P} _{1}-\mathbf {P} _{2}+\mathbf {P} _{3}+\mathbf {P} _{6}}$
    
\end{itemize}
\newpage

% Описание программной реализации
\section*{Описание программной реализации}
\begin {enumerate}
\addcontentsline{toc}{section}{Описание программной реализации}
\item Реализация алгоритма Штрассена содержит следующий класс:

Класс для хранения матриц (содержит количество строк, столбцов и вектор вещественных значений элементов матрицы), который включает в себя такие методы, как конструкторы по умолчанию, с параметрами, с параметрами для генерации случайной матрицы, деструктор, перегруженные операторы присваивания и унарного минуса, а также метод print() для печати матрицы.
\begin{lstlisting}
class Matrix {
 public:
    int rows, cols;
    std::vector<double> m;

    Matrix();
    Matrix(int _rows, int _cols);
    Matrix(int _rows, int _cols, std::mt19937 _gen);
    ~Matrix();
    Matrix operator-() const;
    const Matrix& operator=(const Matrix& a);
    void print() const;
}; 
\end{lstlisting}
\item Реализация алгоритма Штрассена содержит следующие функции:
\end{enumerate}
\begin{itemize}
\itemФункция, которая дополняет исходную марицу нулевыми строками и столбцами до размера, равного степени двойки.
\begin{lstlisting}
Matrix* get_square_matrix(const Matrix& _a, int size);
\end{lstlisting}
\item Функция, которая удаляет лишние строки и столбцы, обрезая матрицу до исходного размера.
\begin{lstlisting}
Matrix* get_orig_size_matrix(const Matrix& _a, int size);
\end{lstlisting}
\item Функция, которая разделяет матрицу на 4 равных по размеру блока.
\begin{lstlisting}
void get_four_matrix(const Matrix& _a, Matrix** _a11, Matrix** _a12, Matrix** _a21, Matrix** _a22);
\end{lstlisting}
\item Функция, которая собирает из четырёх равных по размеру подматриц одну.
\begin{lstlisting}
Matrix* get_one_matrix(const Matrix& _a11, const Matrix& _a12, const Matrix& _a21, const Matrix& _a22);
\end{lstlisting}
\item Функция, выполняющая параллельное сложение двух матриц.
\begin{lstlisting}
Matrix* parallel_sum(const Matrix& _a, const Matrix& _b);
\end{lstlisting}
\item Функция, выполняющая последовательное сложение двух матриц.
\begin{lstlisting}
Matrix* sequential_sum(const Matrix& _a, const Matrix& _b);
\end{lstlisting}
\item Функция, выполняющая последовательное умножение двух матриц.
\begin{lstlisting}
Matrix* sequential_mul(const Matrix& _a, const Matrix& _b);
\end{lstlisting}
\item Функция, реализующая рекурсивный алгоритм Штрассена.
\begin{lstlisting}
Matrix* Strassen_alg(const Matrix& _a, const Matrix& _b);
\end{lstlisting}
\end{itemize}

\newpage

% Подтверждение корректности
\section*{Подтверждение корректности}
\addcontentsline{toc}{section}{Подтверждение корректности}
Для подтверждения корректности в программе представлен набор тестов, разработанных с помощью использования Google C++ Testing Framework.
\par Набор представляет из себя тесты, которые проверяют корректность вычислений (сравниваются матрицы, полученные благодаря параллельной и последовательной реализации алгоритма Штрассена), а также эффективность (вычисление времени, занимаемого последовательной и параллельной реализацией) на разных размерах матриц.
\par Успешное прохождение всех тестов доказывает корректность работы программного комплекса.
\newpage

% Результаты экспериментов
\section*{Результаты экспериментов}
\addcontentsline{toc}{section}{Результаты экспериментов}
Вычислительные эксперименты для оценки эффективности параллельной реализации метода Штрассена проводились на оборудовании со следующей аппаратной конфигурацией:

\begin{itemize}
\item Процессор: Intel(R) Core(TM) i5-10210U CPU @ 1.60GHz - 4.2GHz, 4 ядра, кол-во потоков: 8;
\item Оперативная память: 8,00 ГБ, 2666 MHz;
\item ОС: Microsoft Windows 10 Home, версия 20H2
\end{itemize}

\par Для проведения экспериментов алгоритм Штрассена применяем на матрицах размерами 128$\times$128 и 1024$\times$1024.
\par Результаты экспериментов представлены в таблицах.

\begin{table}[!h]
\caption{Результаты вычислительных экспериментов 128x128}
\centering
\begin{tabular}{lllll}
Процессы & Последовательно & Параллельно & Ускорение   \\
2        & 0.07010         & 0.08045     & 0.8713        \\
3        & 0.06936         & 0.08534     & 0.8127       \\
4        & 0.07091         & 0.09737     & 0.7282       \\
5        & 0.07580         & 0.10842     & 0.6991       \\
\end{tabular}
\end{table}

\begin{table}[!h]
\caption{Результаты вычислительных экспериментов 1024x1024}
\centering
\begin{tabular}{lllll}
Процессы & Последовательно & Параллельно & Ускорение   \\
2        & 101.814         & 30.396      & 3.349        \\
3        & 106.101         & 38.697      & 2.741       \\
4        & 107.05          & 46.80       & 2.287       \\
5        & 104.436         & 54.593      & 1.912       \\
\end{tabular}
\end{table}
\par По данным, полученным в результате экспериментов, можно сделать вывод о том, что параллельный случай работает действительно быстрее, чем последовательный, на размерах матриц примерно 512$\times$512. 
\par Кроме того, можно сделать вывод, что наиболее эффективным будет использование небольшого количества процессов для параллельного алгоритма Штрассена, т.к. много времени уходит на пересылку данных.
\newpage

% Заключение
\section*{Заключение}
\addcontentsline{toc}{section}{Заключение}
В результате лабораторной работы были разработаны последовательная и параллельная реализации алгоритма Штрассена.
\par Основной задачей данной лабораторной работы была реализация параллельной версии с использованием средств MPI, которая должна быть эффективнее последовательной. Эта задача была успешно выполнена, о чём и этом говорят результаты экспериментов, проведенных в ходе работы. 
\par Кроме того, были разработаны и доведены до успешного выполнения тесты, созданные для данного программного проекта с использованием Google C++ Testing Framework и необходимые для подтверждения корректности работы программы.
\newpage

% Список литературы
\begin{thebibliography}{1}
\addcontentsline{toc}{section}{Список литературы}
\bibitem{Levitin} Левитин А.В. Глава 4. Метод декомпозиции: Умножение больших целых чисел и алгоритм умножения матриц Штрассена // Алгоритмы: введение в разработку и анализ. — М.: «Вильямс», 2006. — С. 189-195. 
Левитин А.В. Алгоритмы: введение в разработку и анализ.
\bibitem{Wikipedia} Wikipedia: Алгоритм Штрассена - Википедия [Электронный ресурс] // URL: \url {https://en.wikipedia.org/wiki/Strassen_algorithm} (дата обращения: 05.12.2020)

\end{thebibliography}
\newpage

% Приложение
\section*{Приложение}
\addcontentsline{toc}{section}{Приложение}
В данном разделе находится листинг всего кода, написанного в рамках лабораторной работы.
\begin{lstlisting}
// Strassens_algorithm.h
// Copyright 2020 Streltsova Yana
#ifndef MODULES_TASK_3_STRELTSOVA_Y_STRASSENS_ALGORITHM_STRASSENS_ALGORITHM_H_
#define MODULES_TASK_3_STRELTSOVA_Y_STRASSENS_ALGORITHM_STRASSENS_ALGORITHM_H_
#include <vector>

class Matrix {
 public:
    int rows, cols;
    std::vector<double> m;

    Matrix();
    Matrix(int _rows, int _cols);
    Matrix(int _rows, int _cols, std::mt19937 _gen);
    ~Matrix();
    Matrix operator-() const;
    const Matrix& operator=(const Matrix& a);
    void print() const;
};

Matrix* get_square_matrix(const Matrix& _a, int size);
Matrix* get_orig_size_matrix(const Matrix& _a, int size);
void get_four_matrix(const Matrix& _a, Matrix** _a11, Matrix** _a12, Matrix** _a21, Matrix** _a22);
Matrix* get_one_matrix(const Matrix& _a11, const Matrix& _a12, const Matrix& _a21, const Matrix& _a22);
Matrix* parallel_sum(const Matrix& _a, const Matrix& _b);
Matrix* sequential_sum(const Matrix& _a, const Matrix& _b);
Matrix* sequential_mul(const Matrix& _a, const Matrix& _b);
Matrix* Strassen_alg(const Matrix& _a, const Matrix& _b);

#endif  // MODULES_TASK_3_STRELTSOVA_Y_STRASSENS_ALGORITHM_STRASSENS_ALGORITHM_H_
\end{lstlisting}
\begin{lstlisting}
// Strassens_algorithm.cpp
// Copyright 2020 Streltsova Yana
#include <mpi.h>
#include <random>
#include <ctime>
#include <algorithm>
#include <cmath>
#include <iostream>
#include <vector>
#include "../../modules/task_3/streltsova_y_Strassens_algorithm/Strassens_algorithm.h"

Matrix::Matrix() {
    rows = 0;
    cols = 0;
}
Matrix::Matrix(int _rows, int _cols) {
    rows = _rows;
    cols = _cols;
    m = std::vector<double>(rows * cols, 0.);
}
Matrix::Matrix(int _rows, int _cols, std::mt19937 _gen) {
    rows = _rows;
    cols = _cols;
    m = std::vector<double>(_rows * _cols);
    for (size_t i = 0; i < m.size(); i++) {
        std::uniform_real_distribution<> urd(0, 100);
            m[i] = urd(_gen);
    }
}
Matrix::~Matrix() {}
Matrix Matrix::operator-() const {
    Matrix tmp(rows, cols);
    for (size_t i = 0; i < m.size(); i++)
            tmp.m[i] = -m[i];
    return tmp;
}
const Matrix& Matrix :: operator=(const Matrix& _a) {
    if (this == &_a)
        return *this;
    rows = _a.rows;
    cols = _a.cols;
    m = _a.m;
    return *this;
}
void Matrix::print() const {
    for (int i = 0; i < rows; i++) {
        for (int j = 0; j < cols; j++)
            std::cout << m[i * cols + j] << " ";
        std::cout << std::endl;
    }
}

Matrix* get_square_matrix(const Matrix& _a, int size) {
    Matrix* tmp = new Matrix();
    int n = 2;
    while (size > pow(2, n)) n++;
    if (_a.rows == _a.cols && size == pow(2, n)) {
        *tmp = _a;
        return tmp;
    }
    size = pow(2, n);
    *tmp = Matrix(size, size);
    for (int i = 0; i < _a.rows; i++)
        for (int j = 0; j < _a.cols; j++)
            tmp->m[i * size + j] = _a.m[i * _a.cols + j];
    return tmp;
}

Matrix* get_orig_size_matrix(const Matrix& _a, int size) {
    Matrix* tmp = new Matrix(size, size);
    for (int i = 0; i < size; i++)
        for (int j = 0; j < size; j++)
            tmp->m[i * size + j] = _a.m[i * _a.cols + j];
    return tmp;
}

void get_four_matrix(const Matrix& _a, Matrix** _a11, Matrix** _a12, Matrix** _a21, Matrix** _a22) {
    int n = _a.rows / 2;
    *_a11 = new Matrix(n, n);
    *_a12 = new Matrix(n, n);
    *_a21 = new Matrix(n, n);
    *_a22 = new Matrix(n, n);
    for (int i = 0; i < n; i++)
        for (int j = 0; j < n; j++) {
            (*_a11)->m[i * n + j] = _a.m[i * _a.cols + j];
            (*_a12)->m[i * n + j] = _a.m[i * _a.cols + n + j];
            (*_a21)->m[i * n + j] = _a.m[(n + i) * _a.cols + j];
            (*_a22)->m[i * n + j] = _a.m[(n + i) * _a.cols + n + j];
        }
}

Matrix* get_one_matrix(const Matrix& _a11, const Matrix& _a12, const Matrix& _a21, const Matrix& _a22) {
    int n = _a11.rows;
    MPI_Bcast(&n, 1, MPI_INT, 0, MPI_COMM_WORLD);
    Matrix* result = new Matrix(n * 2, n * 2);
    for (int i = 0; i < n; i++)
        for (int j = 0; j < n; j++) {
            result->m[i * result->cols + j] = _a11.m[i * n + j];
            result->m[i * result->cols + n + j] = _a12.m[i * n + j];
            result->m[(n + i) * result->cols + j] = _a21.m[i * n + j];
            result->m[(n + i) * result->cols + n + j] = _a22.m[i * n + j];
        }
    return result;
}

Matrix* sequential_sum(const Matrix& _a, const Matrix& _b) {
    Matrix* result = new Matrix(_a.rows, _a.cols);
    for (size_t i = 0; i < _a.m.size(); i++)
        result->m[i] = _a.m[i] + _b.m[i];
    return result;
}

Matrix* parallel_sum(const Matrix& _a, const Matrix& _b) {
    int size, rank;
    MPI_Comm_size(MPI_COMM_WORLD, &size);
    MPI_Comm_rank(MPI_COMM_WORLD, &rank);

    int n = _a.rows;
    int part = n / size;
    std::vector<int> scounts(size, part * n);
    int remain = n % size;
    if (rank < remain)
        scounts[rank] += n;
    std::vector<int> displs(size, 0);
    for (int i = 1; i < size; i++)
        displs[i] = displs[i - 1] + scounts[i - 1];

    std::vector<double> local_vec1(scounts[rank]), local_vec2(scounts[rank]);
    MPI_Scatterv(_a.m.data(), scounts.data(), displs.data(), MPI_DOUBLE,
        local_vec1.data(), scounts[rank], MPI_DOUBLE, 0, MPI_COMM_WORLD);
    MPI_Scatterv(_b.m.data(), scounts.data(), displs.data(), MPI_DOUBLE,
        local_vec2.data(), scounts[rank], MPI_DOUBLE, 0, MPI_COMM_WORLD);
    for (int i = 0; i < scounts[rank]; i++)
        local_vec1[i] += local_vec2[i];

    Matrix* result = new Matrix(n, n);
    MPI_Gatherv(local_vec1.data(), scounts[rank], MPI_DOUBLE,
        result->m.data(), scounts.data(), displs.data(), MPI_DOUBLE, 0, MPI_COMM_WORLD);
    return result;
}

Matrix* sequential_mul(const Matrix& _a, const Matrix& _b) {
    if (_a.cols != _b.rows)
        throw "Matrices are not consistent";
    Matrix* C = new Matrix(_a.rows, _b.cols);
    for (int i = 0; i < _a.rows; i++)
        for (int j = 0; j < _b.cols; j++)
            for (int k = 0; k < _a.cols; k++)
                C->m[i * _b.cols + j] += _a.m[i * _a.cols + k] * _b.m[k * _b.cols + j];
    return C;
}

Matrix* Strassen_alg(const Matrix& _a, const Matrix& _b) {
    int rank;
    MPI_Comm_rank(MPI_COMM_WORLD, &rank);
    int n = 0;
    if (rank == 0)
        n = _a.rows;
    MPI_Bcast(&n, 1, MPI_INT, 0, MPI_COMM_WORLD);
    Matrix a(n, n), b(n, n);
    if (rank == 0) {
        a = _a;
        b = _b;
    }
    MPI_Bcast(a.m.data(), 1, MPI_INT, 0, MPI_COMM_WORLD);
    MPI_Bcast(b.m.data(), 1, MPI_INT, 0, MPI_COMM_WORLD);
    if (n <= 64)
        return sequential_mul(a, b);
    Matrix* A = get_square_matrix(a, n);
    Matrix* B = get_square_matrix(b, n);
    Matrix *a11, *a12, *a21, *a22, *b11, *b12, *b21, *b22;
    get_four_matrix(*A, &a11, &a12, &a21, &a22);
    get_four_matrix(*B, &b11, &b12, &b21, &b22);

    Matrix* P1 = Strassen_alg(*parallel_sum(*a11, *a22), *parallel_sum(*b11, *b22));
    Matrix* P2 = Strassen_alg(*parallel_sum(*a21, *a22), *b11);
    Matrix* P3 = Strassen_alg(*a11, *parallel_sum(*b12, -*b22));
    Matrix* P4 = Strassen_alg(*a22, *parallel_sum(*b21, -*b11));
    Matrix* P5 = Strassen_alg(*parallel_sum(*a11, *a12), *b22);
    Matrix* P6 = Strassen_alg(*parallel_sum(*a21, -*a11), *parallel_sum(*b11, *b12));
    Matrix* P7 = Strassen_alg(*parallel_sum(*a12, -*a22), *parallel_sum(*b21, *b22));

    Matrix* c11 = parallel_sum(*parallel_sum(*P1, *P4), *parallel_sum(-*P5, *P7));
    Matrix* c12 = parallel_sum(*P3, *P5);
    Matrix* c21 = parallel_sum(*P2, *P4);
    Matrix* c22 = parallel_sum(*parallel_sum(*P1, -*P2), *parallel_sum(*P3, *P6));
    return get_one_matrix(*c11, *c12, *c21, *c22);
}
\end{lstlisting}
\begin{lstlisting}
// main.cpp
// Copyright 2020 Streltsova Yana
#include <gtest-mpi-listener.hpp>
#include <gtest/gtest.h>
#include <vector>
#include <random>
#include <ctime>
#include "./Strassens_algorithm.h"

TEST(Parallel_Operations_MPI, Test_7x7) {
    int size, rank;
    MPI_Comm_size(MPI_COMM_WORLD, &size);
    MPI_Comm_rank(MPI_COMM_WORLD, &rank);
    int n = 7;
    std::mt19937 gen(time(0));
    Matrix a, b;
    if (rank == 0) {
        a = Matrix(n, n, gen);
        b = Matrix(n, n, gen);
    }
    double start = MPI_Wtime();
    Matrix* c_parallel = Strassen_alg(a, b);
    double end = MPI_Wtime();
    if (rank == 0) {
        std::cout << "Parallel time: " << end - start << std::endl;
        start = MPI_Wtime();
        Matrix* A = get_square_matrix(a, n);
        Matrix* B = get_square_matrix(b, n);
        Matrix* c_seq = sequential_mul(*A, *B);
        c_seq = get_orig_size_matrix(*c_seq, n);
        end = MPI_Wtime();
        std::cout << "Sequential time: " << end - start << std::endl;
        c_parallel = get_orig_size_matrix(*c_parallel, n);
        for (int i = 0; i < n * n; i ++)
            ASSERT_DOUBLE_EQ(round(c_seq->m[i] * 10) / 10, round(c_parallel->m[i] * 10) / 10);
    }
}
TEST(Parallel_Operations_MPI, Test_16x16) {
    int size, rank;
    MPI_Comm_size(MPI_COMM_WORLD, &size);
    MPI_Comm_rank(MPI_COMM_WORLD, &rank);
    int n = 16;
    std::mt19937 gen(time(0));
    Matrix a, b;
    if (rank == 0) {
        a = Matrix(n, n, gen);
        b = Matrix(n, n, gen);
    }
    double start = MPI_Wtime();
    Matrix* c_parallel = Strassen_alg(a, b);
    double end = MPI_Wtime();
    if (rank == 0) {
        std::cout << "Parallel time: " << end - start << std::endl;
        start = MPI_Wtime();
        Matrix* A = get_square_matrix(a, n);
        Matrix* B = get_square_matrix(b, n);
        Matrix* c_seq = sequential_mul(*A, *B);
        c_seq = get_orig_size_matrix(*c_seq, n);
        end = MPI_Wtime();
        std::cout << "Sequential time: " << end - start << std::endl;
        c_parallel = get_orig_size_matrix(*c_parallel, n);
        for (int i = 0; i < n * n; i++)
            ASSERT_DOUBLE_EQ(round(c_seq->m[i] * 10) / 10, round(c_parallel->m[i] * 10) / 10);
    }
}
TEST(Parallel_Operations_MPI, Test_64x64) {
    int size, rank;
    MPI_Comm_size(MPI_COMM_WORLD, &size);
    MPI_Comm_rank(MPI_COMM_WORLD, &rank);
    int n = 64;
    std::mt19937 gen(time(0));
    Matrix a, b;
    if (rank == 0) {
        a = Matrix(n, n, gen);
        b = Matrix(n, n, gen);
    }
    double start = MPI_Wtime();
    Matrix* c_parallel = Strassen_alg(a, b);
    double end = MPI_Wtime();
    if (rank == 0) {
        std::cout << "Parallel time: " << end - start << std::endl;
        start = MPI_Wtime();
        Matrix* A = get_square_matrix(a, n);
        Matrix* B = get_square_matrix(b, n);
        Matrix* c_seq = sequential_mul(*A, *B);
        c_seq = get_orig_size_matrix(*c_seq, n);
        end = MPI_Wtime();
        std::cout << "Sequential time: " << end - start << std::endl;
        c_parallel = get_orig_size_matrix(*c_parallel, n);
        for (int i = 0; i < n * n; i++)
            ASSERT_DOUBLE_EQ(round(c_seq->m[i] * 10) / 10, round(c_parallel->m[i] * 10) / 10);
    }
}
TEST(Parallel_Operations_MPI, Test_128x128) {
    int size, rank;
    MPI_Comm_size(MPI_COMM_WORLD, &size);
    MPI_Comm_rank(MPI_COMM_WORLD, &rank);
    int n = 128;
    std::mt19937 gen(time(0));
    Matrix a, b;
    if (rank == 0) {
        a = Matrix(n, n, gen);
        b = Matrix(n, n, gen);
    }
    double start = MPI_Wtime();
    Matrix* c_parallel = Strassen_alg(a, b);
    double end = MPI_Wtime();
    if (rank == 0) {
        std::cout << "Parallel time: " << end - start << std::endl;
        start = MPI_Wtime();
        Matrix* A = get_square_matrix(a, n);
        Matrix* B = get_square_matrix(b, n);
        Matrix* c_seq = sequential_mul(*A, *B);
        c_seq = get_orig_size_matrix(*c_seq, n);
        end = MPI_Wtime();
        std::cout << "Sequential time: " << end - start << std::endl;
        c_parallel = get_orig_size_matrix(*c_parallel, n);
        for (int i = 0; i < n * n; i++)
            ASSERT_DOUBLE_EQ(round(c_seq->m[i] * 10) / 10, round(c_parallel->m[i] * 10) / 10);
    }
}
TEST(Parallel_Operations_MPI, Test_689x689) {
    int size, rank;
    MPI_Comm_size(MPI_COMM_WORLD, &size);
    MPI_Comm_rank(MPI_COMM_WORLD, &rank);
    int n = 689;
    std::mt19937 gen(time(0));
    Matrix a, b;
    if (rank == 0) {
        a = Matrix(n, n, gen);
        b = Matrix(n, n, gen);
    }
    double start = MPI_Wtime();
    Matrix* c_parallel = Strassen_alg(a, b);
    double end = MPI_Wtime();
    if (rank == 0) {
        std::cout << "Parallel time: " << end - start << std::endl;
        start = MPI_Wtime();
        Matrix* A = get_square_matrix(a, n);
        Matrix* B = get_square_matrix(b, n);
        Matrix* c_seq = sequential_mul(*A, *B);
        c_seq = get_orig_size_matrix(*c_seq, n);
        end = MPI_Wtime();
        std::cout << "Sequential time: " << end - start << std::endl;
        c_parallel = get_orig_size_matrix(*c_parallel, n);
        for (int i = 0; i < n * n; i++)
            ASSERT_DOUBLE_EQ(round(c_seq->m[i] * 10) / 10, round(c_parallel->m[i] * 10) / 10);
    }
}

int main(int argc, char** argv) {
    ::testing::InitGoogleTest(&argc, argv);
    MPI_Init(&argc, &argv);

    ::testing::AddGlobalTestEnvironment(new GTestMPIListener::MPIEnvironment);
    ::testing::TestEventListeners& listeners =
        ::testing::UnitTest::GetInstance()->listeners();

    listeners.Release(listeners.default_result_printer());
    listeners.Release(listeners.default_xml_generator());

    listeners.Append(new GTestMPIListener::MPIMinimalistPrinter);
    return RUN_ALL_TESTS();
}
\end{lstlisting}

\end{document}
